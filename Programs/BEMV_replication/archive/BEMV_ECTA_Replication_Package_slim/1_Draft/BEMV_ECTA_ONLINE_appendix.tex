\documentclass[letterpaper,11pt]{article}

\usepackage{amsmath}
\usepackage{etex}
\usepackage{natbib}
\usepackage{amsfonts}
\usepackage{mathpazo}
\usepackage{hyperref}
\usepackage{multimedia}
\usepackage{graphicx, color}
\usepackage{epsfig}
\usepackage{amsthm}
\usepackage{mathtools}
\usepackage{esint}
\usepackage{amssymb}
\usepackage{url}
\usepackage{graphicx}
\usepackage{relsize}
\usepackage{amsfonts}
\usepackage{bm}
\usepackage{fancyheadings}
\usepackage{float}
\usepackage{color}
\usepackage{mathrsfs}
\usepackage{setspace}
\usepackage[mathscr]{euscript}
\usepackage{caption}
\usepackage{subcaption}
\usepackage{pdflscape}
\usepackage{booktabs}
\usepackage[utf8]{inputenc}
\usepackage[T1]{fontenc}
\usepackage{geometry}
\usepackage[dvipsnames]{xcolor}
\usepackage{tikz}
\usepackage{caption}
\usepackage{subcaption}
\usepackage[bottom]{footmisc}
\usepackage{comment}
\usepackage{xr} % to reference external documents

% External documents
\externaldocument{BEMV_Princeton_AB_v13}
\externaldocument{BEMV_Appendix_AB_v13}
\externaldocument{BEMV_WageDetNote_AB_v13}

\setcounter{MaxMatrixCols}{10}

\hypersetup{
    colorlinks,
    linkcolor={red!75!black},
    citecolor={blue!75!black},
}
\pdfminorversion 4
\setlength{\topmargin}{-0.4in}
\setlength{\textheight}{8.85in}
\setlength{\oddsidemargin}{-0.2in}
\setlength{\evensidemargin}{0.0in}
\setlength{\textwidth}{6.93in}
\renewcommand{\baselinestretch}{1.44}%{1.44}
\defcitealias{GMV}{GMV}
\defcitealias{DFH}{DFH}
\setcounter{page}{1}
\usetikzlibrary{decorations.pathreplacing}
\usetikzlibrary{shapes}
\usetikzlibrary{arrows.meta,arrows}
\usetikzlibrary{fit,positioning}
\tikzset{>=latex}
\definecolor{redSM}{RGB}{226,20,61}
\definecolor{greenSM}{RGB}{0,125,0}
\definecolor{blueSM}{RGB}{0,0,178}
\definecolor{hotpinkSM}{RGB}{191,0,127}
\newcommand{\whiteSM}{\textcolor[rgb]{1,1,1}}           % Used for fillers
\newcommand{\alertblue}{\textcolor{blue}}
\newcommand{\alertroyalblue}{\textcolor{RoyalBlue}}
\newcommand{\purple}{\textcolor{purple}}
\newcommand{\hyperlinkblue}{\textcolor{hyperlinkblue}}
\newcommand{\lightblue}{\textcolor{lightblue}}
\newcommand{\alert}{\textcolor{red}}
\newcommand{\tr}{\textcolor{red}}

%\input{tcilatex}

\usepackage{nicefrac}

%~~~~~~~~~~~~~~~~~~~~~~~~~~~~~~~~~~~~~~~~~~~~~~~~~~~~~~~~~~~~~~~~~~~~~~~~~~~~~~~~~~~~
% DIRECTORIES FOR ALL FIGURES AND TABLES IN PAPER
\newcommand{\MATLABfigureDir}{../2_MATLAB_code/Created_figure_files}       % Figures created by MATLAB
\newcommand{\MATLABtableDir}{../2_MATLAB_code/Created_table_files}         % Tables created by MATLAB
\newcommand{\DRAWNfigureDir}{./Figures_without_data}              % Figures drawn 'by hand' and imported as png or tex files (no data or code used)
%~~~~~~~~~~~~~~~~~~~~~~~~~~~~~~~~~~~~~~~~~~~~~~~~~~~~~~~~~~~~~~~~~~~~~~~~~~~~~~~~~~~~








\setlength{\topmargin}{-0.6in}
\setlength{\textheight}{9.05in}
\setlength{\oddsidemargin}{-0.3in}
\setlength{\evensidemargin}{-0.2in}
\setlength{\textwidth}{7.23in}

\begin{document}

\appendix
\begin{center}
\textsc{\huge Appendix II - Not for Publication}\vspace*{.5cm} \\
\textbf{\Large Firm and Worker Dynamics in a Frictional Labor Market}\vspace*{.5cm} \\
\emph{\Large Adrien Bilal, Niklas Engbom, Simon Mongey, Gianluca Violante}\vspace*{.5cm} \\
\end{center}

\small
\setstretch{1.1}

\noindent This Appendix contains the proof of the joint surplus representation in the fully dynamic model, the main result of Section 3 in the main text, and computational details.
Section \ref{appx:dynamicmodel} lays out the notation for the fully dynamic model.
Section \ref{appx:jointvalue_derivation} develops the proof.
Section \ref{appx:computation} contains details on the computation of the stationary equilibrium of the model.

\section{Notation for dynamic model}\label{appx:dynamicmodel}

We first specify the value function of an
individual worker $i$ in a firm with arbitrary state $x$: $V(x,i)$. We then
specify the value function of the firm: $J(x)$. Combining all workers' value
functions with that of the firm we define the joint value: $\Omega(x)$. We
then apply the assumptions from Section 2 which allow us
to reduce $(x)$ to only the number of workers and productivity of the firm, $%
(z,n)$. Finally we take the continuous work force limit to derive a
Hamilton-Jacobi-Bellman (HJB) equation for $\Omega(z,n)$ Applying the
definition of total surplus used above, we obtain a HJB equation in $S(z,n)$
which we use to construct the equilibrium.

\subsection{Worker value function: $V$}

Let $U$ be the value of unemployment. Define separately worker $i$'s value when employed at firm $x$
\emph{before} the quit, layoff and exit decisions, $\bm{V}%
\left(x,i\right) $, and their value \emph{after} these decisions, $V\left(
x,i\right)$.

\paragraph{Value of unemployment.}

Let $h_{U}\left( x\right)$ denote how the state of firm $x$ is updated when
it hires an unemployed worker.\footnote{%
For example, size would be update from $n$ to $n+1$ and possibly some of the
incumbent wages would be bargained down.} Let $\mathcal{A}$ denote the set
of firms making job offers that an unemployed worker would accept. The value of
unemployment $U$ therefore satisfies
\begin{equation*}
\rho U=b+\lambda^U(\theta)\int_{x\in \mathcal{A}}\left[ \bm{V}\left(
h_{U}\left( x\right) ,i\right) -U\right] dH_v\left(x\right)
\end{equation*}%
where $H_v$ is the vacancy-weighted distribution of firms. If $x\notin $ $%
\mathcal{A}$, then the worker remains unemployed.

\paragraph{Stage I.}

To relate the value of the worker pre separation, $\bm{V}(x,i)$, to that
post separation, $V(x,i)$, we require the following notation regarding firm
and co-worker actions. Since workers do not form `unions' within the firm,
all of these actions are taken as given by worker $i$.

\begin{itemize}
\item[-] Let $\epsilon(x)\in \left\{ 0,1\right\}$ denote the exit decision
of firm, and $\mathcal{E=}\left\{ x:\epsilon (x)=1\right\}$ the set of $x$'s
for which the firm exits.

\item[-] Let $\ell(x)\in\{0,1\}^{n(x)}$ be a vector of zeros and ones of length $n(x)$, with generic entry
$\ell_{i}(x)$, that characterizes the
firm's decision to lay off incumbent worker $i\in\{1,\dots,n(x)\}$, and $\mathcal{L=}%
\left\{\left( x,i\right) :\ell _{i}(x)=1\right\}$ the set of $(x,i)$ such
that worker $(x,i)$ is laid off.

\item[-] Let $q^U\left( x\right) \in\{0,1\}^{n(x)}$ be a vector of length $%
n(x)$, with generic entry $q^U_i(x)$ that characterizes an incumbent
workers' decisions to quit, and $\mathcal{Q}^{U}=\left\{ \left(
x,i\right):q^U_i(x)=1\right\} $ the set of $(x,i)$ such that worker $(x,i)$
quits into unemployment.

\item[-] Let $\kappa\left( x\right) =\left( 1-\bm{\ell}\left( x\right)
\right) \circ \left( 1-q_{U}\left( x\right) \right)$ be an element-wise
product vector that identifies workers that are kept in the firm, and $%
\mathcal{S}=\mathcal{L}\cup\mathcal{Q}^{U}=\left\{\left(x,i\right):%
\kappa_{i}\left( x\right) =0\right\}$, the set of $(x,i)$ such that worker $%
(x,i)$ separates into unemployment.

\item[-] Let $s(x,\kappa(x))$ denote how the state of firm $x$ is updated
when workers identified by $\kappa(x)$ are kept. This includes any renegotiation.
\end{itemize}

\noindent Given these sets and functions, the pre separation value $\bm{V}\left(
x,i\right) $ satisfies:
\begin{equation*}
\bm{V}\left( x,i\right) =\underbrace{\Big.\epsilon(x)U\Big.}_{\text{Exit}%
} + (1-\epsilon(x))\Big[\underbrace{\Big.\mathbb{I}_{\{(x,i)\notin \mathcal{S}%
\}} V(s(x,\kappa(x)),i)\Big.}_{\text{Continuing employment}} + \underbrace{%
\Big.\mathbb{I}_{\{(x,i)\in \mathcal{S}\}}U\Big.}_{\text{Separations and
Quits}}\Big]
\end{equation*}

\paragraph{Stage II.}

It is helpful to characterize the continuation value of employment post separation
decisions, $V(x,i)$, in terms of the three distinct types of events
described in Figure . First, the value changes due to
\emph{`Direct'} labor markets shocks to worker $i$, $V_{D}(x,i)$. These
include her match being destroyed exogenously or meeting a new potential
employer. Second, the value changes due to labor market shocks hitting
other workers in the firm, $V_{I}(x,i)$, including their matches being
exogenously destroyed or them meeting new potential employers. These events
have an \emph{`Indirect'} impact on worker $i$. Third, the value changes due to
events on the \emph{`Firm'} side, $V_{F}(x,i)$, including the firm
contacting new workers and receiving productivity shocks. Combining
events and exploiting the fact that in continuous time they are mutually
exclusive, we obtain the following, where  $w\left( x,i\right)$ is the wage paid to worker $i$:
\begin{equation*}
\rho V\left( x,i\right) =w\left( x,i\right) +\rho V_{D}\left( x,i\right)
+\rho V_{I}\left( x,i\right) +\rho V_{F}\left( x,i\right).
\end{equation*}

The wage function $w(x,i)$ includes the transfers between worker $i$ and the firm that may occur at the stage of vacancy posting (after separations and before the labor market opens), as discussed in the Appendix Section A in the context of the static example. These transfers can depend on the entire wage distribution inside the firm which is subsumed in the state vector $x$.

\paragraph{Direct events.}

We first characterize changes in value due to labor market shocks directly to worker $i$ in firm $x$, $V_{D}(x,i)$. Exogenous separation shocks arrive at rate $\delta $ and draws of outside offers arrive at rate $\lambda^E(\theta)$ from the vacancy-weighted distribution of firms $H_v$. If worker $i$ receives a sufficiently good outside offer from $x^{\prime }$, she quits to the new firm. We denote by $\mathcal{Q}^{E}(x,i)$ the set of such quit-firms $x^{\prime}$
for $i$. Otherwise, the worker remains with the current firm but with an updated contract. Therefore $V_D(x,i)$ satisfies
\vspace{-0.1in}

{\small
\begin{eqnarray*}
\rho V_{D}\left( x,i\right) = \underbrace{\Bigg.\delta \left[ U-V\left(
x,i\right)\right]}_{\text{Exogenous separation}} &+& \lambda^E(\theta)
\underbrace{\Bigg.\int_{x^{\prime }\in \mathcal{Q}^{E}\left(x,i\right) }%
\left[ \bm{V}\left( h_{E}\left( x,i,x^{\prime }\right),i\right) -V\left(
x,i\right) \right] dH_v\left( x^{\prime }\right)}_{EE \text{ $\mathcal{Q}$uit}}
\\
&+& \lambda^E(\theta) \underbrace{\Bigg.\int_{x^{\prime }\notin \mathcal{Q}%
^{E}\left( x,i\right) }\left[ \bm{V}\left( r\left( x,i,x^{\prime
}\right) ,i\right) -V\left( x,i\right) \right] dH_v\left( x^{\prime }\right)}_{%
\text{Retention}},
\end{eqnarray*}%
}\noindent where $h_{E}\left( x,i,x^{\prime }\right) $ describes how the
state of a poaching firm $x^{\prime}$ gets updated when it hires worker $i$
from firm $x$. Similarly, $r\left( x,i,x^{\prime}\right)$ updates $x$ when---after meeting firm $x^\prime$---worker $i$ in firm $x$ is
retained and renegotiates its value.
In all functions with three arguments $\left( x,i,x^{\prime}\right)$, the
first argument denotes the origin firm, the second identifies the worker,
and the third the potential destination firm.

\paragraph{Indirect events.}

We next characterize changes in value due to the same labor market shocks hitting other workers in firm $x$, $V_{I}(x,i)$. The value $V_{I}(x,i)$ satisfies
\vspace{-0.1in}

{\small
\begin{eqnarray*}
\rho V_{I}\left( x,i\right) = \sum_{j\neq i}^{n\left( x\right)} \Bigg\{
\underbrace{\delta \left[ \bm{\ V}\left( d(x,j),i\right) -V\left(
x,i\right) \right]\Bigg.}_{\text{Exogenous separation}} &+& \lambda^E(\theta)%
\underbrace{\Bigg.\int_{x^{\prime }\in \mathcal{Q}^{E}\left( x,j\right) } %
\left[\bm{V} \left( q_{E}\left( x,j,x^{\prime }\right) ,i\right)
-V\left(x,i\right)\right] dH_v\left( x^{\prime }\right)}_{\text{$EE$ Quit}} \\
&+& \lambda^E(\theta)\underbrace{\Bigg.\int_{x^{\prime }\notin \mathcal{Q}%
^{E}\left( x,j\right) }\left[ \bm{V}\left( r\left( x,j,x^{\prime
}\right) ,i\right) -V\left( x,i\right) \right] dH_v\left( x^{\prime }\right)}_{%
\text{Retention}} \Bigg\},
\end{eqnarray*}%
}where $d(x,j)$ updates $x$ when worker $j$ exogenously separates, and $%
q_{E}\left( x,j,x^{\prime }\right) $ when worker $j$ quits to firm $%
x^{\prime}$.

\paragraph{Firm events.}
Finally, we characterize changes in value due to events that directly impact the firm and hence
indirectly its workers, $V_F(x,i)$.
Taking as given the firm's vacancy posting policy $v(x)$ and
other actions, $V_F(x,i)$ satisfies
\begin{eqnarray*}
\rho V_{F}\left( x,i\right) &=& \\
\text{$UE$ Hire} &&\phi q(\theta)v\left( x\right) \left[ \bm{V}\left(
h_{U}\left( x\right) ,i\right) -V\left( x,i\right) \right] \cdot \mathbb{I}%
_{\left\{ x\in \mathcal{A}\right\} } \\
\text{$UE$ Threat} &&+\phi q(\theta)v\left( x\right) \left[ \bm{V}\left(
t_{U}\left( x\right) ,i\right) -V\left( x,i\right) \right] \cdot \mathbb{I}%
_{\left\{ x\notin \mathcal{A}\right\} } \\
\text{$EE$ Hire} &&+\left( 1-\phi \right)q(\theta)v\left( x\right)
\int_{x\in \mathcal{Q}^{E}\left( x^{\prime },i^{\prime }\right) }\left[
\bm{V}\left( h_{E}\left( x^{\prime },i^{\prime },x\right) ,i\right)
-V\left( x,i\right) \right] dH_n\left( x^{\prime },i^{\prime }\right) \\
\text{$EE$ Threat} &&+\left( 1-\phi \right)q(\theta)v\left( x\right)
\int_{x\notin \mathcal{Q}^{E}\left( x^{\prime},i^{\prime}\right) }\left[
\bm{V}\left( t_{E}\left( x^{\prime },i^{\prime },x\right) ,i\right)
-V\left( x,i\right) \right] dH_n\left( x^{\prime },i^{\prime }\right) \\
\text{Shock} &&+\Gamma _{z}\left[ \bm{V},V\right] \left( x,i\right)
\end{eqnarray*}%
where $t_{U}\left( x\right) $ updates $x$ when an
unemployed worker is met and not hired, but could be possibly used as a
threat in firm $x$. Similarly, $t_{E}(x^{\prime },i^{\prime },x)$ updates
$x$ when worker $i^{\prime }$ employed at firm $x^{\prime }$
is met, not hired, but could be used as a threat. And, with a slight abuse
of notation, $H_n(x^\prime,i^\prime)$ gives the joint distribution of firms $%
x^\prime$ and worker types within firms $i^\prime$.

Finally, $\Gamma _{z}\left[ \bm{V},V\right] \left( x,i\right) $
identifies the contribution of productivity shocks $z$ to the Bellman
equation. At this stage we only require that the productivity process is
Markovian with an infinitesimal generator. Later we will specialize this to
a diffusion process $dz_t = \mu(z_t)dt + \sigma(z_t) dW_t$ such that
\begin{eqnarray}
\Gamma _{z}\left[ \bm{V},V\right] \left( x,i\right) &=&\mu \left( z\right)
\lim_{dz\rightarrow 0}\frac{\bm{V}\left( \left( x,z+dz\right) ,i\right)
-V\left( x,z,i\right) }{dz} \notag \\
&+&\frac{\sigma ^{2}\left( z\right) }{2}%
\lim_{dz\rightarrow 0}\frac{\bm{V}\left( \left( x,z+dz\right) ,i\right) +%
\bm{V}\left( \left( x,z-dz\right) ,i\right) -2V\left( x,z,i\right) }{%
dz^{2}}  \label{eq:D_diffusion}
\end{eqnarray}
In the case that $\bm{V}=V$, this becomes the standard expression for
a diffusion featuring the first and second derivatives of $V$ with respect to $z$: $\Gamma_z[V](x,i) =
\mu(z)V_z(x,z,i) + \frac{1}{2}\sigma(z)^2V_{zz}(x,z,i)$.\footnote{%
Note that in $(\ref{eq:D_diffusion})$ we abuse notation and write the state
as $(x,z)$ with some redundancy since $z$ is clearly a member of $x$. We
also note that we are not constrained to a diffusion process. We could also
consider a Poisson process where, at exogenous rate $\eta$, $z$ jumps
according to the transition density $\Pi(z,z^\prime)$: $\Gamma _{z}\left[
\bm{V},V\right] \left( x,i\right) = \eta[\sum_{z^{\prime }\in Z}\bm{V%
}\left( \left( x,z^{\prime }\right) ,i\right) \Pi \left( z^{\prime
},z\right) -V\left( x,z,i\right)] $.}

In the event productivity changes or $n\left( x\right)$ changes because of
exogenous labor market events, the worker will want to reassess whether to
stay with the firm or not. Additionally, the firm may want to reassess
whether to exit or fire some workers. Bold values $\bm{V}$ capture any
case where the state changes.

\subsection{Firm value function: $J$}

Consistent with the notation we used for workers' values, let $\bm{J}(x)$
and $J(x)$ be the values of the firm at the corresponding points of an interval $dt$.
For now, we take the vacancy creation decision $v\left(
x\right)$ as given. At the end of the section we describe the expected value
of an entrant firm.

\paragraph{Stage I.}
Consistent with the first stage worker value function, we define the firm
value before the exit/layoff/quit decision, where we recall that $\vartheta$ is the firm's value of exit, or scrap value:
\begin{equation*}
\bm{J}\left( x\right) =\epsilon \left( x\right)\vartheta + \left[
1-\epsilon \left( x\right) \right] J\left( s\left( x,\kappa \left( x\right)
\right) \right).
\end{equation*}%

\paragraph{Stage II.}
Given a vacancy policy $v\left( x\right)$, let $J\left( x\right)$ be the
value of a firm with state $x$ \emph{after} the layoff/quit, exit. It is
convenient to split the value of the firm, as we did for the
worker, into three components
\begin{equation*}
\rho J\left( x\right) = \underbrace{\Bigg.y\left( x\right)
-\sum\limits_{i=1}^{n\left( x\right)}w_{i}\left( x,i\right)\Bigg.}_{\text{%
Flow profits}} \quad+\quad \underbrace{\Bigg.\rho J_{W}\left( x\right)\Bigg.}%
_{\text{Workforce events}} +\quad \underbrace{\Bigg.\rho J_{F}\left(
x\right)-c\left( v\left( x\right) ,x\right)\Bigg.}_{\text{Firm events net of
vacancy costs}}.
\end{equation*}%

For a given policy $v(x)$ there is a set of associated transfers between workers and the firm which, as for the worker value function, are implicit in the wage function $w(x,i)$.

The component $J_{W}\left( x\right) $ is given by
\begin{eqnarray*}
\rho J_{W}\left( x\right) &=& \\
\text{Destruction} &&\delta \sum\limits_{i=1}^{n\left( x\right) }\left[
\bm{J}\left( d(x,i)\right) -J\left( x\right) \right] \\
\text{$EE$ Quit} &&+\:\:\lambda^E(\theta)\sum\limits_{i=1}^{n\left( x\right)
}\int_{x^{\prime }\in \mathcal{Q}^{E}\left( x,i\right) }\left[ \bm{J}%
\left( q_{E}\left( x,i,x^{\prime }\right) \right) -J\left( x\right) \right]
dH_v\left( x^{\prime }\right) \\
\text{Retention} &&+\:\:\lambda^E(\theta)\sum\limits_{i=1}^{n\left( x\right)
}\int_{x^{\prime }\notin \mathcal{Q}^{E}\left( x,i\right) }\left[ \bm{J}%
\left( r\left( x,i,x^{\prime }\right) \right) -J\left( x\right) \right]
dH_v\left( x^{\prime }\right).
\end{eqnarray*}%
The component $J_{F}\left( x\right) $ is given by
\begin{eqnarray*}
\rho J_{F}\left( x\right) &=& \\
\text{$UE$ Hire} && \phi q(\theta)v\left( x\right) \left[ \bm{J}\left(
h_{U}\left( x\right) \right) -J\left( x\right) \right] \cdot \mathbb{I}%
_{\left\{ x\in \mathcal{A}\right\} } \\
\text{$UE$ Threat} &&+\:\:\phi q(\theta)v\left( x\right) \left[ \bm{J}%
\left( t_{U}\left( x\right) \right) -J\left( x\right) \right] \cdot \mathbb{I%
}_{\left\{ x\notin \mathcal{A}\right\} } \\
\text{$EE$ Hire} &&+\:\:\left( 1-\phi \right)q(\theta)v\left( x\right)
\int_{x\in \mathcal{Q}^{E}\left( x^{\prime },i^{\prime }\right) }\left[
\bm{J}\left( h_{E}\left( x^{\prime },i^{\prime },x\right) \right)
-J\left( x\right) \right] dH_n\left( x^{\prime },i^{\prime }\right) \\
\text{$EE$ Threat} &&+\:\:\left(1-\phi \right)q(\theta)v\left( x\right)
\int_{x\notin \mathcal{Q}^{E}\left( x^{\prime },i^{\prime }\right) }\left[
\bm{J}\left( t_{E}\left( x^{\prime },i^{\prime },x\right) \right)
-J\left( x\right) \right] dH_n\left( x^{\prime },i^{\prime }\right) \\
\text{Shock} &&+\:\Gamma _{z}\left[ \bm{J},J\right] \left( x\right)
\end{eqnarray*}

Recall that, in continuous time at most one contact is made
per instant. Either one worker is exogenously separated, or one
worker is contacted by another firm, or one worker is met by posting
vacancies (at rate $q(\theta)v(x)$), or a shock hits the firm. We have bold $\bm{J}$'s in each line since after any of these events, the firm may want to layoff some workers or exit, and workers may want to quit.

\paragraph{Entry.}
The expected value of an entrant firm is
\begin{equation}
J_0=-c_{0}+\int \bm{J}\left( x_{0}\right) d\Pi _{0}\left( z_0\right)
\label{eq:D_J0}
\end{equation}%
where $x_{0}$ is the state of the entrant firm which includes only the
random productivity value $z_0$ drawn from $\Pi _{0}$ since we assumed the
initial number of workers is $0$. The argument of the integral is $\bm{J}
$, which incorporates the firm's decision to exit or operate after
observing $z_0$. Entry occurs when $J_0 >0$.


\section{Derivation of the joint value function $\Omega$}
\label{appx:jointvalue_derivation}
We define the \textbf{joint value} of the firm and
its employed workers $\Omega \left( x\right) :=J\left( x\right) +\sum_{i=1}^{n\left(
x\right)}V\left( x,i\right) $. We also define the joint value before exit/quit/layoff
decisions: $\bm{\Omega }\left( x\right) :=\bm{J}\left(
x\right)+\sum_{i=1}^{n\left( x\right) }\bm{V}\left( x,i\right) $.


\subsection{Combinining worker and firm values}

In this section, we show that summing firm and worker
values, then applying these definitions delivers the following Bellman
equation for the joint value:
\begin{eqnarray}
\rho \Omega \left( x\right) &=&y\left( x\right) -c\left( v\left( x\right)
,x\right)  \label{eq:D_omega1} \\
\text{Destruction} &&+\:\:\sum\limits_{i=1}^{n\left( x\right) }\delta \left[
\bm{\Omega }\left( d(x,i)\right) +U-\Omega \left( x\right) \right]
\notag \\
\text{Retention} &&+\:\:\lambda^E(\theta)\sum\limits_{i=1}^{n\left( x\right)
}\int_{x^{\prime }\notin \mathcal{Q}^{E}\left( x,i\right) }\left[ \bm{%
\Omega }\left( r\left( x,i,x^{\prime }\right) \right) -\Omega \left(
x\right) \right] dH_v\left( x^{\prime }\right)  \notag \\
\text{$EE$ Quit} &&+\:\:\lambda^E(\theta)\sum\limits_{i=1}^{n\left( x\right)
}\int_{x^{\prime }\in \mathcal{Q}^{E}\left( x,i\right) }\left[ \bm{%
\Omega }\left( q_{E}\left( x,i,x^{\prime }\right) \right) +\bm{V}\left(
h_{E}\left( x,i,x^{\prime }\right) ,i\right) -\Omega \left( x\right) \right]
dH_v\left( x^{\prime }\right)  \notag \\
\text{$UE$ Hire} &&+\:\:\phi q(\theta)v\left( x\right) \left[ \bm{\Omega
}\left( h_{U}\left( x\right) \right) -U-\Omega \left( x\right) \right] \cdot
\mathbb{I}_{\left\{ x\in \mathcal{A}\right\} }  \notag \\
\text{$UE$ Threat} &&+\:\:\phi q(\theta)v\left( x\right) \left[ \bm{%
\Omega }\left( t_{U}\left( x\right) \right) -\Omega \left( x\right) \right]
\cdot \mathbb{I}_{\left\{ x\notin \mathcal{A}\right\} }  \notag \\
\text{$EE$ Hire} &&+\:\:(1-\phi)q(\theta)v\left( x\right) \int_{x\in
\mathcal{Q}^{E}\left( x^{\prime },i^{\prime }\right) }\left[ \bm{\Omega }%
\left( h_{E}\left( x^{\prime },i^{\prime },x\right) \right) -\bm{V}%
\left( h_{E}\left( x^{\prime },i^{\prime },x\right) ,i^{\prime }\right)
-\Omega \left( x\right) \right] dH_n\left( x^{\prime },i^{\prime }\right)
\notag \\
\text{$EE$ Threat} &&+\:\:(1-\phi)q(\theta)v\left( x\right) \int_{x\notin
\mathcal{Q}^{E}\left( x^{\prime },i^{\prime }\right) }\left[ \bm{\Omega }%
\left( t_{E}\left( x^{\prime },i^{\prime },x\right) \right) -\Omega \left(
x\right) \right] dH_n\left( x^{\prime },i^{\prime }\right)  \notag \\
\text{Shock} &&+\:\:{\Gamma }_{z}\left[ \bm{\Omega },\Omega \right]
\left( x\right) .  \notag
\end{eqnarray}%
This joint value is only written in terms of other joint values
and worker values. However, it involves both firm and worker decisions
through the sets $\mathcal{A}, \mathcal{Q}^{E}$ and the vacancy policy, $v(x)$.

\paragraph{Derivation.}

We start by computing the sum of the workers' values at a particular firm.
Summing values of all the employed workers
\begin{eqnarray*}
\rho \sum\limits_{i=1}^{n\left( x\right) }V\left( x,i\right)
&=&\sum\limits_{i=1}^{n\left( x\right) }w\left( x,i\right) \\
\text{Destructions} &&+\sum\limits_{i=1}^{n\left( x\right) }\delta \left[
U-V\left( x,i\right) \right] \\
\text{Retentions} &&+\lambda ^{E}\sum\limits_{i=1}^{n\left( x\right)
}\int_{x^{\prime }\notin \mathcal{Q}^{E}\left( x,i\right) }\left[ \boldsymbol{V}%
\left( r\left( x,i,x^{\prime }\right) ,i\right) -V\left( x,i\right) \right]
dH_v\left( x^{\prime }\right) \\
\text{$EE$ Quits} &&+\lambda ^{E}\sum\limits_{i=1}^{n\left( x\right)
}\int_{x^{\prime }\in \mathcal{Q}^{E}\left( x,i\right) }\left[ \boldsymbol{V}%
\left( h_{E}\left( x,i,x^{\prime }\right) \right) -V\left( x,i\right) \right]
dH_v\left( x^{\prime }\right) \\
\text{Incumbents} &&+\sum\limits_{i=1}^{n\left( x\right) }\rho V_{I}(x,i) \\
\text{Firm} &&+\sum\limits_{i=1}^{n\left( x\right) }\rho V_{D}(x,i)
\end{eqnarray*}

where the indirect terms due to incumbents and the firm can be written as:
\begin{eqnarray*}
\sum\limits_{i=1}^{n\left( x\right) }\rho V_{I}(x,i) &=& \\
\text{Destructions} &&\sum\limits_{i=1}^{n\left( x\right) }\sum_{j\neq
i}^{n\left( x\right) }\delta \left[ \boldsymbol{V}\left( d(x,j),i\right)
-V\left( x,i\right) \right] \\
\text{Retentions} &&+\sum\limits_{i=1}^{n\left( x\right) }\sum_{j\neq
i}^{n\left( x\right) }\lambda ^{E}\int_{x^{\prime }\notin \mathcal{Q}%
^{E}\left( x,j\right) }\left[ \boldsymbol{V}\left( r\left( x,j,x^{\prime
}\right) ,i\right) -V\left( x,i\right) \right] dH_v\left( x^{\prime }\right) \\
\text{$EE$ Quits} &&+\sum\limits_{i=1}^{n\left( x\right) }\sum_{j\neq
i}^{n\left( x\right) }\lambda ^{E}\int_{x^{\prime }\in \mathcal{Q}^{E}\left(
x,j\right) }\left[ \boldsymbol{V}\left( q_{E}\left( x,j,x^{\prime }\right)
,i\right) -V\left( x,i\right) \right] dH_v\left( x^{\prime }\right) \ , \\
\sum\limits_{i=1}^{n\left( x\right) }\rho V_{F}(x,i) &=& \\
\text{$UE$ Hires} &&qv\left( x\right) \phi \sum\limits_{i=1}^{n\left(
x\right) }\left[ \boldsymbol{V}\left( h_{U}\left( x\right) ,i\right) -V\left(
x,i\right) \right] \cdot \mathbb{I}_{\left\{ x\in \mathcal{A}\right\} } \\
\text{$UE$ Threats} &&+qv\left( x\right) \phi \sum\limits_{i=1}^{n\left(
x\right) }\left[ \boldsymbol{V}\left( t_{U}\left( x\right) ,i\right) -V\left(
x,i\right) \right] \cdot \mathbb{I}_{\left\{ x\notin \mathcal{A}\right\} } \\
\text{$EE$ Hires} &&+qv\left( x\right) \left( 1-\phi \right)
\sum\limits_{i=1}^{n\left( x\right) }\int_{x\in \mathcal{Q}^{E}\left(
x^{\prime },i^{\prime }\right) }\left[ \boldsymbol{V}\left( h_{E}\left(
x^{\prime },i^{\prime },x\right) ,i\right) -V\left( x,i\right) \right]
dH_n\left( x^{\prime },i^{\prime }\right) \\
\text{$EE$ Threats} &&+qv\left( x\right) \left( 1-\phi \right)
\sum\limits_{i=1}^{n\left( x\right) }\int_{x\notin \mathcal{Q}^{E}\left(
x^{\prime },i^{\prime }\right) }\left[ \boldsymbol{V}\left( t_{E}\left(
x^{\prime },i^{\prime },x\right) ,i\right) -V\left( x,i\right) \right]
dH_n\left( x^{\prime },i^{\prime }\right) \\
\text{Shocks} &&+\sum\limits_{i=1}^{n\left( x\right) } \Gamma_z[\boldsymbol{V}%
,V](x,i)
\end{eqnarray*}%
We now collect terms.

\paragraph{Destructions.}

When worker $i$ separates from firm $x$, the sum of the changes in values of
all employed workers at its own firm is given by:
\begin{eqnarray*}
\text{Destructions} &=&\delta \left[ U-V\left( x,i\right) \right] +\delta
\sum_{j\neq i}^{n\left( x\right) }\left[ \boldsymbol{V}\left( d(x,i),j\right)
-V\left( x,j\right) \right] =\delta \left[ U+\sum_{j\neq i}^{n\left( x\right) }\boldsymbol{V}\left(
d(x,i),j\right) -\sum_{j=1}^{n\left( x\right) }V\left( x,j\right) \right]
\end{eqnarray*}

\paragraph{Retentions.}

When $i$ renegotiates at firm $x$, the sum of the changes in values of all
employed workers at its own firm is given by:
\begin{eqnarray*}
\hspace*{-1cm}\text{Retentions} &=&\lambda ^{E}\int_{x^{\prime }\notin \mathcal{Q}%
^{E}\left( x,i\right) }\left[ \mathbf{V}\left( r\left( x,i,x^{\prime
}\right) ,i\right) -V\left( x,i\right) \right] dH_v\left( x^{\prime }\right) ++\lambda ^{E}\int_{x^{\prime }\notin \mathcal{Q}^{E}\left( x,i\right)
}\sum_{j\neq i}^{n\left( x\right) }\left[ \mathbf{V}\left( r\left(
x,i,x^{\prime }\right) ,j\right) -V\left( x,j\right) \right] dH_v\left(
x^{\prime }\right) \\
&=&\lambda ^{E}\int_{x^{\prime }\notin \mathcal{Q}^{E}\left( x,i\right) }%
\left[ \mathbf{V}\left( r\left( x,i,x^{\prime }\right) ,i\right)
+\sum_{j\neq i}^{n\left( x\right) }\mathbf{V}\left( r\left( x,i,x^{\prime
}\right) ,j\right) -\sum_{j=1}^{n\left( x\right) }V\left( x,j\right) \right]
dH_v\left( x^{\prime }\right) \\
&=&\lambda ^{E}\int_{x^{\prime }\notin \mathcal{Q}^{E}\left( x,i\right) }%
\left[ \sum_{j=1}^{n\left( x\right) }\mathbf{V}\left( r\left( x,i,x^{\prime
}\right) ,j\right) -\sum_{j=1}^{n\left( x\right) }V\left( x,j\right) \right]
dH_v\left( x^{\prime }\right)
\end{eqnarray*}

\paragraph{Quits.}

Similarly, when $i$ quits firm $x$, the sum of the changes in values of all
employed workers at its own firm is given by:
\begin{equation*}
\text{$EE$ Quits}=\lambda ^{E}\int_{x^{\prime }\in Q\left( x,i\right) }\left[
\boldsymbol{V}\left( h_{E}\left( x,i,x^{\prime }\right) ,i\right) +\sum_{j\neq
i}^{n\left( x\right) }\boldsymbol{V}\left( q_{E}\left( x,i,x^{\prime }\right)
,j\right) -\sum_{j=1}^{n\left( x\right) }V\left( x,j\right) \right] dH_v\left(
x^{\prime }\right)
\end{equation*}

\paragraph{Combining terms.}

Before summing up all these terms, define for convenience the total worker value:
\begin{eqnarray*}
\rho \overline{V}\left( x\right) &=&\sum\limits_{i=1}^{n\left( x\right) }w\left(
x,i\right) \\
\text{Destructions} &&+\sum\limits_{i=1}^{n\left( x\right) }\delta \left[
U+\sum_{j\neq i}^{n\left( x\right) }\boldsymbol{V}\left( d(x,i),j\right)
-\sum_{j=1}^{n\left( x\right) }V\left( x,j\right) \right] \\
\text{Retentions} &&+\lambda ^{E}\sum\limits_{i=1}^{n\left( x\right)
}\int_{x^{\prime }\notin \mathcal{Q}^{E}\left( x,i\right) }\left[
\sum_{j=i}^{n\left( x\right) }\boldsymbol{V}\left( r\left( x,i,x^{\prime
}\right) ,j\right) -\sum_{j=1}^{n\left( x\right) }V\left( x,j\right) \right]
dH_v\left( x^{\prime }\right) \\
\text{$EE$ Quits} &&+\lambda ^{E}\sum\limits_{i=1}^{n\left( x\right)
}\int_{x^{\prime }\in \mathcal{Q}^{E}\left( x,i\right) }\left[ \boldsymbol{V}%
\left( h_{E}\left( x,i,x^{\prime }\right) ,i\right) +\sum_{j\neq i}^{n\left(
x\right) }\boldsymbol{V}\left( q_{E}\left( x,i,x^{\prime }\right) ,j\right)
-\sum_{j=1}^{n\left( x\right) }V\left( x,j\right) \right] dH_v\left( x^{\prime
}\right) \\
\text{$UE$ Hires} &&+qv\left( x\right) \phi \sum\limits_{i=1}^{n\left(
x\right) }\left[ \boldsymbol{V}\left( h_{U}\left( x\right) ,i\right) -V\left(
x,i\right) \right] \cdot \mathbb{I}_{\left\{ x\in \mathcal{A}\right\} } \\
\text{$UE$ Threats} &&+qv\left( x\right) \phi \sum\limits_{i=1}^{n\left(
x\right) }\left[ \boldsymbol{V}\left( t_{U}\left( x\right) ,i\right) -V\left(
x,i\right) \right] \cdot \mathbb{I}_{\left\{ x\notin \mathcal{A}\right\} } \\
\text{$EE$ Hires} &&+qv\left( x\right) \left( 1-\phi \right)
\sum\limits_{i=1}^{n\left( x\right) }\int_{x\in \mathcal{Q}^{E}\left(
x^{\prime },i^{\prime }\right) }\left[ \boldsymbol{V}\left( h_{E}\left(
x^{\prime },i^{\prime },x\right) ,i\right) -V\left( x,i\right) \right]
dH_n\left( x^{\prime },i^{\prime }\right) \\
\text{$EE$ Threats} &&+qv\left( x\right) \left( 1-\phi \right)
\sum\limits_{i=1}^{n\left( x\right) }\int_{x\notin \mathcal{Q}^{E}\left(
x^{\prime },i^{\prime }\right) }\left[ \boldsymbol{V}\left( t_{E}\left(
x^{\prime },i^{\prime },x\right) ,i\right) -V\left( x,i\right) \right]
dH_n\left( x^{\prime },i^{\prime }\right) \\
\text{Shocks} &&+\sum\limits_{i=1}^{n\left( x\right) }\Gamma_z[\boldsymbol{V}%
,V](x,i)
\end{eqnarray*}%
Now sum, up all the previous terms, collect terms and use the definition of $%
\overline{V}\left( x\right) $:
\begin{eqnarray*}
\rho \overline{V}\left( x\right) &=&\sum\limits_{i=1}^{n\left( x\right) }w\left(
x,i\right) \\
\text{Destructions} &&+\sum\limits_{i=1}^{n\left( x\right) }\delta \left[
U+\sum_{j\neq i}^{n\left( x\right) }\boldsymbol{V}\left( d(x,i),j\right) -\overline{V}%
\left( x\right) \right] \\
\text{Retentions} &&+\lambda ^{E}\sum\limits_{i=1}^{n\left( x\right)
}\int_{x^{\prime }\notin \mathcal{Q}^{E}\left( x,i\right) }\left[
\sum_{j=i}^{n\left( x\right) }\boldsymbol{V}\left( r\left( x,i,x^{\prime
}\right) ,j\right) -\overline{V}\left( x\right) \right] dH_v\left( x^{\prime
}\right) \\
\text{$EE$ Quits} &&+\lambda ^{E}\sum\limits_{i=1}^{n\left( x\right)
}\int_{x^{\prime }\in \mathcal{Q}^{E}\left( x,i\right) }\left[ \boldsymbol{V}%
\left( h_{E}\left( x,i,x^{\prime }\right) ,i\right) +\sum_{j\neq i}^{n\left(
x\right) }\boldsymbol{V}\left( q_{E}\left( x,i,x^{\prime }\right) ,j\right) -%
\overline{V}\left( x\right) \right] dH_v\left( x^{\prime }\right) \\
\text{$UE$ Hires} &&+qv\left( x\right) \phi \left[ \sum\limits_{i=1}^{n%
\left( x\right) }\boldsymbol{V}\left( h_{U}\left( x\right) ,i\right) -\overline{V}%
\left( x\right) \right] \cdot \mathbb{I}_{\left\{ x\in \mathcal{A}\right\} }
\\
\text{$UE$ Threats} &&+qv\left( x\right) \phi \left[ \sum\limits_{i=1}^{n%
\left( x\right) }\boldsymbol{V}\left( t_{U}\left( x\right) ,i\right) -\overline{V}%
\left( x\right) \right] \cdot \mathbb{I}_{\left\{ x\notin \mathcal{A}%
\right\} } \\
\text{$EE$ Hires} &&+qv\left( x\right) \left( 1-\phi \right) \int_{x\in
\mathcal{Q}^{E}\left( x^{\prime },i^{\prime }\right) }\left[
\sum\limits_{i=1}^{n\left( x\right) }\boldsymbol{V}\left( h_{E}\left( x^{\prime
},i^{\prime },x\right) ,i\right) -\overline{V}\left( x\right) \right] dH_n\left(
x^{\prime },i^{\prime }\right) \\
\text{$EE$ Threats} &&+qv\left( x\right) \left( 1-\phi \right) \int_{x\notin
\mathcal{Q}^{E}\left( x^{\prime },i^{\prime }\right) }\left[
\sum\limits_{i=1}^{n\left( x\right) }\boldsymbol{V}\left( t_{E}\left( x^{\prime
},i^{\prime },x\right) ,i\right) -\overline{V}\left( x\right) \right] dH_n\left(
x^{\prime },i^{\prime }\right) \\
\text{Shocks} &&+\Gamma_z[\boldsymbol{\overline{V}},\overline{V}](x)
\end{eqnarray*}
Adding this last equation to the Bellman equation for $J(x)$ yields
{\small
\begin{eqnarray*}
\rho \Omega \left( x\right) &=&y\left( x\right) -c\left( v\left( x\right)
,x\right) \\
\text{Destructions} &&+\sum\limits_{i=1}^{n\left( x\right) }\delta \left[
\boldsymbol{J}\left( d(x,i)\right) +U+\sum_{j\neq i}^{n\left( x\right) }\boldsymbol{V%
}\left( d(x,i),j\right) -J\left( x\right) -\overline{V}\left( x\right) \right] \\
\text{Retentions} &&+\lambda ^{E}\sum\limits_{i=1}^{n\left( x\right)
}\int_{x^{\prime }\notin \mathcal{Q}^{E}\left( x,i\right) }\left[ \boldsymbol{J}%
\left( r\left( x,i,x^{\prime }\right) \right) +\sum_{j=i}^{n\left( x\right) }%
\boldsymbol{V}\left( r\left( x,i,x^{\prime }\right) ,j\right) -J\left( x\right) -%
\overline{V}\left( x\right) \right] dH_v\left( x^{\prime }\right) \\
\text{$EE$ Quits} &&+\lambda ^{E}\sum\limits_{i=1}^{n\left( x\right)
}\int_{x^{\prime }\in \mathcal{Q}^{E}\left( x,i\right) }\Bigg[ \boldsymbol{J}%
\left( q_{E}\left( x,i,x^{\prime }\right) \right) +\boldsymbol{V}\left(
h_{E}\left( x,i,x^{\prime }\right) ,i\right) +\sum_{j\neq i}^{n\left(
x\right) }\boldsymbol{V}\left( q_{E}\left( x,i,x^{\prime }\right) ,j\right)
-J\left( x\right) -\overline{V}\left( x\right) \Bigg] dH_v\left( x^{\prime }\right)
\\
\text{$UE$ Hires} &&+qv\left( x\right) \phi \left[ \boldsymbol{J}\left(
h_{U}\left( x\right) \right) +\sum\limits_{i=1}^{n\left( x\right) }\boldsymbol{V}%
\left( h_{U}\left( x\right) ,i\right) -J\left( x\right) -\overline{V}\left(
x\right) \right] \cdot \mathbb{I}_{\left\{ x\in \mathcal{A}\right\} } \\
\text{$UE$ Threats} &&+qv\left( x\right) \phi \left[ \boldsymbol{J}\left(
t_{U}\left( x\right) \right) +\sum\limits_{i=1}^{n\left( x\right) }\boldsymbol{V}%
\left( t_{U}\left( x\right) ,i\right) -J\left( x\right) -\overline{V}\left(
x\right) \right] \cdot \mathbb{I}_{\left\{ x\notin \mathcal{A}\right\} } \\
\text{$EE$ Hires} &&+qv\left( x\right) \left( 1-\phi \right) \int_{x\in
\mathcal{Q}^{E}\left( x^{\prime },i^{\prime }\right) }\left[ \boldsymbol{J}%
\left( h_{E}\left( x^{\prime },i^{\prime },x\right) \right)
+\sum\limits_{i=1}^{n\left( x\right) }\boldsymbol{V}\left( h_{E}\left( x^{\prime
},i^{\prime },x\right) ,i\right) -J\left( x\right) -\overline{V}\left( x\right) %
\right] dH_n\left( x^{\prime },i^{\prime }\right) \\
\text{$EE$ Threats} &&+qv\left( x\right) \left( 1-\phi \right) \int_{x\notin
\mathcal{Q}^{E}\left( x^{\prime },i^{\prime }\right) }\left[ \boldsymbol{J}%
\left( t_{E}\left( x^{\prime },i^{\prime },x\right) \right)
+\sum\limits_{i=1}^{n\left( x\right) }\boldsymbol{V}\left( t_{E}\left( x^{\prime
},i^{\prime },x\right) ,i\right) -J\left( x\right) -\overline{V}\left( x\right) %
\right] dH_n\left( x^{\prime },i^{\prime }\right) \\
\text{Shocks} &&+\Gamma_z[\boldsymbol{J} + \boldsymbol{\overline{V}},J + \overline{V}](x)
-J\left( x\right) -\overline{V}\left( x\right)
\end{eqnarray*}}
Collecting terms and using the definition of $\Omega$ :
{\small
\begin{eqnarray*}
\rho \Omega \left( x\right) &=&y\left( x\right) -c\left( v\left( x\right)
,x\right) \\
\text{Destructions} &&+\sum\limits_{i=1}^{n\left( x\right) }\delta \left[
\boldsymbol{\Omega}\left( d(x,i)\right) +U -\Omega \left( x\right) \right] \\
\text{Retentions} &&+\lambda ^{E}\sum\limits_{i=1}^{n\left( x\right)
}\int_{x^{\prime }\notin \mathcal{Q}^{E}\left( x,i\right) }\left[ \boldsymbol{%
\Omega}\left( r\left( x,i,x^{\prime }\right) \right) -\Omega \left( x\right) %
\right] dH_v\left( x^{\prime }\right) \\
\text{$EE$ Quits} &&+\lambda ^{E}\sum\limits_{i=1}^{n\left( x\right)
}\int_{x^{\prime }\in \mathcal{Q}^{E}\left( x,i\right) }\left[ \boldsymbol{\Omega%
}\left( q_{E}\left( x,i,x^{\prime }\right) \right) +\boldsymbol{V}\left(
h_{E}\left( x,i,x^{\prime }\right) ,i\right) -\Omega \left( x\right) \right]
dH_v\left( x^{\prime }\right) \\
\text{$UE$ Hires} &&+qv\left( x\right) \phi \left[ \boldsymbol{\Omega}\left(
h_{U}\left( x\right) \right) - U -\Omega \left( x\right) \right] \cdot
\mathbb{I}_{\left\{ x\in \mathcal{A}\right\} } \\
\text{$UE$ Threats} &&+qv\left( x\right) \phi \left[ \boldsymbol{\Omega}\left(
t_{U}\left( x\right) \right) -\Omega \left( x\right) \right] \cdot \mathbb{I}%
_{\left\{ x\notin \mathcal{A}\right\} } \\
\text{$EE$ Hires} &&+qv\left( x\right) \left( 1-\phi \right) \int_{x\in
\mathcal{Q}^{E}\left( x^{\prime },i^{\prime }\right) }\left[ \boldsymbol{\Omega}%
\left( h_{E}\left( x^{\prime },i^{\prime },x\right) \right) -\boldsymbol{V}%
\left( h_{E}\left( x^{\prime },i^{\prime },x\right) ,i^{\prime}\right)
-\Omega \left( x\right) \right] dH_n\left( x^{\prime },i^{\prime }\right) \\
\text{$EE$ Threats} &&+qv\left( x\right) \left( 1-\phi \right) \int_{x\notin
\mathcal{Q}^{E}\left( x^{\prime },i^{\prime }\right) }\left[ \boldsymbol{\Omega}%
\left( t_{E}\left( x^{\prime },i^{\prime },x\right) \right) -\Omega \left(
x\right) \right] dH_n\left( x^{\prime },i^{\prime }\right) \\
\text{Shocks} &&+\Gamma_z[\boldsymbol{\overline{\Omega}},\overline{\Omega}](x)
\end{eqnarray*}
}


\subsection{Value sharing}

To make progress on $(\ref{eq:D_omega1})$, we begin by stating seven intermediate results, conditions \textbf{(C-RT)-(C-E)} which we prove from
the assumptions listed in Section 2.2. These results establish how worker values \textbf{$\bm{V}$} in $(\ref{eq:D_omega1})$ evolve in the six cases of hiring, retention, layoff, quits, exit and vacancy creation. Next, we apply conditions \textbf{(C-RT)-(C-E)} to $(\ref{eq:D_omega1})$.

To highlight the structure of the argument, we note a key implication our zero-sum game assumption \textbf{(A-IN)}: during internal negotiation, any value lost to one party must accrue to the other. This feature is obvious in the static model, and extends readily to our dynamic environment. 
In other words, the joint value of the firm plus its incumbent workers is invariant during the negotiation. We use this property extensively in the proof. This generalizes pairwise efficient bargaining---commonly used in one-worker firm models with linear production---to an environment with multi-worker firms and decreasing returns in production.

We now state the seven conditions that we apply to $(\ref{eq:D_omega1})$. In section \ref{sec:proofC} below, we prove how each of them is implied by the
assumptions of Section $2.2$.

\begin{enumerate}
\item[\textbf{(C-RT)}] \textbf{Retentions and Threats.} First, if firm $x$
meets an unemployed worker and the worker is not hired but only used as a
threat, then the joint value of coalition $x$ does not change since threats
only redistribute value within the coalition. Second, when firm $x$ uses employed worker $i^{\prime }$ from firm $x^{\prime }$ as a threat, the joint value of coalition $x$ does not change. Third, when firm $x$ meets worker $i^{\prime }$ at $x^{\prime }$ and the worker is retained by
firm $x^{\prime }$, the joint value of coalition $x^{\prime }$ does not change. Formally,
\begin{equation*}
\bm{\Omega }\left( r\left( x^{\prime },i^{\prime },x\right)
\right)=\Omega (x^{\prime })\quad,\quad \bm{\Omega }\left( t_{U}\left(
x\right) \right) =\Omega (x)\quad,\quad \bm{\Omega }\left( t_{E}\left(
x^{\prime },i^{\prime },x\right) \right)=\Omega (x).
\end{equation*}
Respectively, these imply that the \emph{Retention}, \emph{$UE$
Threat} and \emph{$EE$ Threat} components of $(\ref{eq:D_omega1})$ are equal
to zero.
\item[\textbf{(C-UE)}] \textbf{$UE$ Hires.} An unemployed worker that meets
firm $x$ is hired when $x\in\mathcal{A}$. This set consists of firms that
have a joint value after hiring that is higher than the pre-hire joint value
plus the outside value of the hired worker.
Due to the take-leave offer, the new hire receives her outside value, which is the value of unemployment:
\begin{equation*}
\mathcal{A}=\{x|\bm{\Omega}(h_{U}(x))-\Omega(x)\geq U\}
\quad,\quad
\bm{V}\left( h_{U}\left( x\right) ,i\right) =U.
\end{equation*}

\item[\textbf{(C-EE)}] \textbf{$EE$ Hires.} An employed worker $i^{\prime }$
at firm $x^{\prime }$ that meets firm $x$ is hired when $x\in \mathcal{Q}%
^{E}\left(x^{\prime },i^{\prime }\right)$. This set consists of firms that
have a higher marginal joint value than that of the current firm:
\begin{equation*}
\mathcal{Q}^{E}\left( x^{\prime },i^{\prime }\right) =\left\{ x\Big|\bm{%
\Omega }\left( h_{E}\left( x^{\prime },i^{\prime },x\right) \right) -\Omega
\left( x\right) \geq \Omega (x^{\prime })-\bm{\Omega }\left( q_{E}\left(
x^{\prime },i^{\prime },x\right) \right) \right\}.
\end{equation*}
Due to the take-leave offer, the new hire receives her
outside value, which is the
marginal joint value at her current firm:
\begin{equation*}
\bm{V}\left( h_{E}\left( x^{\prime },i^{\prime },x\right) \right)
=\Omega \left( x^{\prime }\right) -\bm{\Omega }\left( q_{E}\left(
x^{\prime },i^{\prime },x\right) \right) .
\end{equation*}

\item[\textbf{(C-EU)}] $EU$\textbf{\ Quits and Layoffs.} An employed worker $%
i$ at firm $x$ quits to unemployment when $\left( x,i\right) \in \mathcal{Q}%
^{U}$. This set consist of states $x$ such that the marginal joint value is
less than the value of unemployment:
\begin{eqnarray*}
\mathcal{Q}^{U}&=&\left\{ (x,i)\Big|\Omega \left( \widehat{s}_{q1}\left(
x,i\right) \right) +U>\Omega \left( \widehat{s}_{q0}\left( x,i\right) \right)
\right\}, \\
%&& \\
\text{where}\quad\quad\widehat{s}_{q1}\left( x,i\right) &=&s\left( x,\left( 1-\left[ q_{U,-i}\left(
x\right) ;q_{U,i}\left( x\right) =1\right] \right) \circ \left( 1-\ell
\left( x\right) \right) \right), \\
\widehat{s}_{q0} \left( x,i\right) &=&s\left( x,\left( 1-\left[ q_{U,-i}\left(
x\right) ;q_{U,i}\left( x\right) =0\right] \right) \circ\left( 1-\ell \left(
x\right) \right)\right).
\end{eqnarray*}%
The first expression captures when worker $i$ quits, and the second where
worker $i$ does not.
Similarly, an $EU$ layoff will be chosen by the firm
when $\left( x,i\right) \in \mathcal{L}$:
\begin{eqnarray*}
\mathcal{L}&=&\left\{ (x,i)\Big|\Omega \left( \widehat{s}_{\ell 1}\left(
x,i\right) \right) +U>\Omega \left( \widehat{s}_{\ell 0}\left( x,i\right)
\right) \right\}, \\
%&& \\
\text{where}\quad\quad\widehat{s}_{\ell 1}\left( x,i\right) &=&s\left( x,\left( 1-\left[ \ell \left(
x\right) ;\ell _{i}\left( x\right) =1\right] \right) \circ \left(
1-q_{U}\left( x\right) \right) \right) , \\
\widehat{s}_{\ell 0}\left( x,i\right) &=&s\left( x,\left( 1-\left[ \ell \left(
x\right) ;\ell _{i}\left( x\right) =0\right] \right) \circ \left(
1-q_{U}\left( x\right) \right) \right) .
\end{eqnarray*}%
The first expression captures when worker $i$ is laid off, and the second
when worker $i$ is not.

\item[\textbf{(C-X)}] \textbf{Exit.} A firm $x$ exits when $x\in \mathcal{E}$%
. This set consists of the states in which the total outside value of the
firm and its workers is larger than the joint value of operation:
\begin{equation*}
\mathcal{E}=\left\{ x\Big|\vartheta +n\left( s\left( x,\kappa \left(
x\right) \right) \right) \cdot U>\Omega \left( s\left( x,\kappa \left(
x\right) \right) \right) \right\} .
\end{equation*}

\item[\textbf{(C-V)}] \textbf{Vacancies.} The expected return to a matched
vacancy $R(x)$ depends only on the joint value, and so the firm's optimal
vacancy policy $v(x)$ depends only on the joint value. The policy $v(x)$
solves
\begin{equation*}
\max_{v}\:\:q(\theta)vR(x)-c\left( v,x\right),
\end{equation*}
where the expected return to a matched vacancy is
\begin{eqnarray*}
R(x) &=& \phi \underbrace{\Big.\left[ \bm{\Omega }\left( h_{U}\left( x\right)
\right) -\Omega \left( x\right) -U\right] \cdot \mathbb{I}_{\left\{ x\in
\mathcal{A}\right\} }}_{\text{Return from unemployed worker match}} \\
&+&\left( 1-\phi \right)\underbrace{\Big.\int_{x\in \mathcal{Q}^{E}\left(
x^{\prime },i^{\prime }\right) }\left\{ \left[ \bm{\Omega }\left(
h_{E}\left( x^{\prime },i^{\prime },x\right) \right) -\Omega \left( x\right) %
\right] -\left[ \Omega \left( x^{\prime }\right) -\bm{\Omega }\left(
q_{E}\left( x^{\prime },i^{\prime },x\right) \right) \right] \right\}
dH_n\left( x^{\prime },i^{\prime }\right)}_{\text{Expected return from
employed worker match}} .
\end{eqnarray*}

\item[\textbf{(C-E)}] \textbf{Entry.} A firm enters if and only if
\begin{equation*}
\int \bm{\Omega }\left( x_{0}\right) d\Pi _{0}(z)\geq c_{0}+n_{0}U.
\end{equation*}
\end{enumerate}

\paragraph{Summarizing (C).}

The substantive result is that all firm and worker decisions and employed
workers' values can be expressed in terms of joint value $\Omega$ and
exogenous worker outside option $U$.

\subsection{Proof of Conditions (C)}

\label{sec:proofC}

\subsubsection{Proof of C-UE and C-RT ($UE$ Hires and $UE$ Threats) \label{ue}}

In this subsection, we consider a meeting between a firm $x$ and an
unemployed worker. Following \textbf{A-IN} and \textbf{A-EN}, the firm internally renegotiates according to a zero-sum game with its incumbent workers and makes a take-leave offer  to the new worker. Intuitively, having the
worker ``at the door'' is identical to having her hired at value $U$ for the
firm and for all incumbent workers: the firm can always make new take-leave
offers to its incumbents after hiring the new worker. Hence, we expect the
firm to make one take-leave offer to the new worker and its incumbents at
the time of the meeting, and not make a new, different offer to is
incumbents afer hiring has taken place.

We start by showing this equivalence formally. To do so, when meeting an
unemployed worker, we let the firm conduct internal renegotiation with its
incumbent workers and make an offer to the new worker. Then, we let a
second round of internal offers take place after the hiring. We introduce
some notation to keep track of values throughout the internal and external
negotiations. To fix ideas, we denote by (IR1) the first round of internal
negotiation, pre-external negotiation. We denote by (IR2) the second round
of internal negotiation, post-hire.

Post-hire and post-internal negotiation (IR2) values are denoted with double
stars. Post-internal-negotiation (IR1) but pre-external-negotiation values
are denoted with stars.
\begin{align*}
\Omega^{\ast\ast} &
:=J^{\ast\ast}+\sum_{j=1}^{n(x)}V_{j}^{\ast\ast}+V_{i}^{\ast\ast} &
\Omega^{\ast} & :=J^{\ast}+\sum_{j=1}^{n(x)}V_{j}^{\ast} &
\Omega & :=J+\sum_{j=1}^{n\left(x\right)}V_{j}
\end{align*}

Proceeding by backward induction, under \textbf{A-EN} the firm makes a
take-it-or-leave-it offer to the unemployed worker, therefore
\begin{equation*}
V_{i}^{\ast\ast}=U
\end{equation*}

We now divide the proof in several steps. We start by proving that for all incumbent workers $j=1...n(x)$, $V_{j}^{\ast\ast}=V_{j}^{\ast}$. We then use \textbf{A-IN} to argue that $\Omega^* = \Omega$. Once these claims have been proven, we move on
to proving \textbf{C-UE} ($UE$ Hires) and the part of for threats from
unemployment \textbf{C-RT} ($UE$ Threats). Finally, we show that our microfoundations for the renegotiation game deliver \textbf{A-IN}.

\paragraph{Claim 1:}

For all incumbents workers $j=1...n(x)$, we have $V_{j}^{\ast\ast}=V_{j}^{%
\ast}$.\newline

We proceed by backwards induction using our assumptions \textbf{A-EN} and \textbf{A-IN}. Immediately after (IR1) has taken place, only the following events can
happen:

\begin{enumerate}
\item Hire/not-hire

\begin{itemize}
\item Either the worker is hired from unemployment (H),

\item Or the worker is not hired from unemployment (NH)
\end{itemize}

\item Possible new round of internal negotiation (IR2). This possible second
round of internal negotiation (now including the newly hired worker) leads
to values $V_j^{**}$.
\end{enumerate}

We focus on subgame perfect equilibria in this multi-stage game. Therefore,
after (IR1), workers perfectly anticipate what the outcome of the
hire/not-hire stage will be. That is, after (IR1), they know perfectly what
hiring decision (H or NH) the firm will make. Now suppose that internal
renegotiation (IR2) actually happens after the hire/not-hire decision, that
is, that for some incumbent worker $j \in \{1,...,n(x)\}$, $V_j^{**} \neq
V_j^*$. The firm has no incentives to accept a change in the new
worker's value to anything above $U$, so by \textbf{A-MC} her value does not
change in the second round (IR2).

We construct the rest of the proof by contradiction. Consider for a
contradiction an incumbent worker $j$ whose value changed in (IR2). Because
of \textbf{A-MC}, her value can change only in the following cases:

\begin{itemize}
\item The firm has a credible threat to fire worker $j$, in which case $%
V_j^{**} < V_j^*$

\item Worker $j$ has a credible threat to quit, in which case $V_j^{**} >
V_j^*$
\end{itemize}

In addition, those credible threats can lead to a different outcome than in
(IR1), and thus $V_j^{**} \neq V_j^*$, only if the threat on either side was
not available in (IR1). If that same threat was available in the first round
(IR1), then the outcome of the bargaining (IR1) would have been $V_j^{**}$.

Recall that both incumbent worker $j$ and the firm understand and anticipate
which hire/not-hire decision the firm will make after the first round (IR1).
They also understand and anticipate that, in case of hire, the value of the
new worker will remain $U$ in the second round (IR2).

Therefore, the firm can \textit{credibly threaten} to hire the new worker in
the first round \textit{if and only if} it actually hires her after the
first round (IR1) is over. This implies that the firm can credibly threaten
worker to fire $j$ in the second round (IR2), by \textbf{A-LC}, \textit{if
and only if} it could credibly threaten her with hiring the new worker
\textit{in the first round of internal renegotiation (IR1).} This in turn
entails that any credible threat the firm can make in the second round
(IR2) was already available in the first round.

On the worker side, quitting into unemployment is a credible threat
when her value is below the value of unemployment. So this threat does not
change between the first round (IR1) and the second round (IR2), because the
equilibrium value to that worker will always be above the value of
unemployment.

In sum, the set of credible threats both to the firm and to worker $j$ does
not change between the initial round of internal renegotiation (IR1) and the
post-hiring-decision round (IR2). This finally implies that the outcome of
the initial round of internal renegotiation (IR1) for any incumbent $j$
remains unchanged in the second round (IR2), that is:
\begin{equation*}
V_j^{**} = V_j^*
\end{equation*}
which proves \textbf{Claim 1}.

We can now move on to proving \textbf{C-UE}.

\paragraph{Proof of C-UE.}

Using the definitions of $\Omega^{**}$ and $\Omega$, we can write
\begin{align*}
\Omega^{\ast\ast}-\Omega & =\left[J^{\ast\ast}+\sum_{j=1}^{n(x)}V_{j}^{\ast%
\ast}+V_{i}^{\ast\ast}\right]-\left[J+\sum_{j=1}^{n\left(x\right)}V_{j}%
\right]
\end{align*}
Now using $V_i^{**} =U$, we obtain
\begin{align*}
\Omega^{\ast\ast}-\Omega & =\left[J^{\ast\ast}+\sum_{j=1}^{n(x)}V_{j}^{\ast%
\ast}\right]-\left[J+\sum_{j=1}^{n\left(x\right)}V_{j}\right]+U
\end{align*}
Using \textbf{Claim 1:} $V_{j}^{**} = V_j^*$, and adding and subtracting $J^\ast$ we obtain
\begin{align*}
\Omega^{\ast\ast}-\Omega & =\left[J^{\ast\ast}-J^{\ast}\right]
+\left[J^{\ast}+\sum_{j=1}^{n(x)}V_{j}^{\ast}\right]
-\left[J+\sum_{j=1}^{n\left(x\right)}V_{j}\right]+U
\end{align*}
Subsituting in the definition of $\Omega$ and of $\Omega^*$,
\begin{align*}
\Omega^{\ast\ast}-\Omega & =\left[J^{\ast\ast}-J^{\ast}\right]%
+\left[\Omega^*-\Omega\right] +U
\end{align*}
Finally recall that internal renegotiation is (1) individually rational, and (2) is a zero-sum game, according to \textbf{A-IN}. Thus, all incumbent workers remain in the coalition after internal renegotiation, and the joint value is unchanged: $\Omega^* = \Omega$. Using $\Omega^* = \Omega$
\begin{align*}
\Omega^{\ast\ast}-\Omega & =\left[J^{\ast\ast}-J^{\ast}\right]+U
\end{align*}
which can be re-written
\begin{align*}
J^{\ast\ast}-J^{\ast} & =\left[\Omega^{\ast\ast}-\Omega\right]-U
\end{align*}
Now under \textbf{A-LC}, the firm will only hire if its value after hiring
is higher than its value after internal renegotiation: $J^{\ast\ast}-J^{%
\ast}\geq0$. This inequality requires
\begin{align*}
\Omega^{\ast\ast}-\Omega & \geq U \\
\boldsymbol{\Omega}\left(h_{U}\left(x\right)\right)-\Omega\left(x\right) & \geq U
\end{align*}
The firm does not hire when its value of hiring is below its value of
renegotiation $J^{**} < J^*$. This inequality implies
\begin{equation*}
\Omega^{\ast\ast}-\Omega < U
\end{equation*}
When the firm does not hire, we obtain using again \textbf{A-IN} and $\Omega^* = \Omega$:
\begin{equation*}
\Omega^{\ast\ast}-\Omega^* < U \newline
\end{equation*}
which finally implies
\begin{equation*}
\boldsymbol{\Omega}\left(h_{U}\left(x\right)\right)-\boldsymbol{\Omega}%
\left(t_U(x)\right) < U
\end{equation*}

Now, we argue that conditional on not hiring, $\Omega^{**} = \Omega^* = \Omega$, where in this
case $\Omega^{**}$ denotes the value of the coalition without hiring, and
thus does not include the value of the unemployed worker. Just as before, this is a direct consequence from \textbf{A-IN} and that the internal renegotiation game is zero-sum.

Therefore:
\begin{equation*}
\boldsymbol{\Omega}\left(t_U(x)\right) = \Omega(x)
\end{equation*}
We have therefore shown \textbf{C-UE} and part of \textbf{C-RT ($UE$ Hires
and $UE$ Threats)}: An unemployed worker that meets $x$ is hired when$x\in%
\mathcal{Q}^{U}$, where
\begin{equation*}
\mathcal{A}=\left\{ x\Big|\boldsymbol{\Omega}\left(h_{U}\left(x\right)\right)-%
\Omega\left(x\right)\geq U\right\}
\end{equation*}
and upon joining the firm, has value
\begin{equation*}
\boldsymbol{V}\left(h_{U}\left(x,i\right)\right)=U.
\end{equation*}
and
\begin{equation*}
\boldsymbol{\Omega}(t_U(x)) = \Omega(x).
\end{equation*}


\subsubsection{Proof of C-EE and C-RT ($EE$ Hires, $EE$ Threats and Retentions)}

The structure of the argument for EE hires, threats and retentions follows closely the steps of the argument for UE hires, threats and retentions. The only major difference is that the worker's outside option is endogenously determined when hired from employment. Consider firm $x$ that has met worker $i^{\prime}$ at firm $x^{\prime}$. We proceed in two steps.

\paragraph{Maximum value at incumbent firm.} We first seek to determine the maximum value that $x'$ may offer to its worker when it may be poached by firm $x$. Under \textbf{A-IN} and \textbf{A-EN}, upon meeting an employed worker, internal negotiation
may take place at the poaching firm $x$, and $x$ makes a
take-it-or-leave-it offer. Internal negotiation may take place at $%
x^{\prime}$ with all workers including $i^{\prime}$.

Proceeding by backward induction, we again define intermediate values but
here at $x^{\prime}$, noting that $q_E\left(x^{\prime},i^{\prime},x\right)$
gives the number of employees in $x^{\prime}$ if the worker leaves:
\begin{align*}
\Omega &
=J+\sum_{j=1}^{n\left(q_E\left(x^{\prime},i^{\prime},x\right)%
\right)}V_{j}+V_{i^{\prime}} &
\Omega^{\ast} &
=J^{\ast}+\sum_{j=1}^{n\left(q_E\left(x^{\prime},i^{\prime},x\right)%
\right)}V_{j}^{\ast}+V_{i^{\prime}}^{\ast} &
\Omega^{\ast\ast} &
=J^{\ast\ast}+\sum_{j=1}^{n\left(q_E\left(x^{\prime},i^{\prime},x\right)%
\right)}V_{j}^{\ast\ast}
\end{align*}
In the second equation we are describing the values of the firm in
renegotiation where $i^{\prime}$ stays with the firm, so $%
V_{i^{\prime}}^{\ast}$ is the outcome of internal negotiation. In the third
equation we consider the firm having lost the worker. Under \textbf{A-EN} the
firm will respond to an offer $\overline{V}$ from $x$ with
\begin{equation*}
V_{i^{\prime}}^{\ast}=\overline{V}
\end{equation*}
Following the same arguments as in \textbf{Claim 1} from section \ref{ue}, the same result obtains: under
\textbf{A-EN} and \textbf{A-IN}, the values accepted by the incumbent workers \emph{after the
internal renegotiation $(V_{j}^{\ast})_j$ }will be equal to the
values they receive \emph{after the} \emph{external negotiation $%
(V_{j}^{\ast\ast})_j$, }that is
\begin{equation*}
V_{j}^{\ast\ast}=V_{j}^{\ast}
\end{equation*}
Then following the same steps as in section \ref{ue}, and using again that $\Omega^* = \Omega$, we obtain that:
\begin{equation*}
\Omega^{\ast\ast}-\Omega=\left[J^{\ast\ast}-J^{\ast}\right]-\overline{V}
\end{equation*}
Now under \textbf{A-LC}, the firm $x^{\prime }$ will only try to keep the
worker if $J^{\ast }>J^{\ast \ast }$, which requires
\begin{align*}
\Omega -\Omega ^{\ast \ast }& \leq \overline{V} &
\boldsymbol{\Omega} \left( r(x^{\prime },i^{\prime},x\right) -\boldsymbol{\Omega }%
\left( q_{E}\left( x^{\prime },i^{\prime },x\right) \right) & \leq \overline{V}
\end{align*}
This determines the maximum value that $x^{\prime }$ can offer to the worker
to retain them.

\paragraph{Poaching.} Our second step is to check when worker $i'$ moves from $x'$ to $x$. The bargaining protocol implies that $x$ firm will offer $%
\overline{V}$ if it is making an offer, since it need not offer more. For firm $x$
the argument may proceed identically to the case of unemployment, simply
replacing $U$ with $\overline{V}.$ The result is that the firm will hire only if and only if
\begin{equation*}
\boldsymbol{\Omega }\left( h_{E}\left( x^{\prime },i^{\prime },x\right) \right)
-\Omega\left( x\right) \geq \overline{V}
\end{equation*}%
or, equivalently,
\begin{equation*}
\boldsymbol{\Omega }\left( h_{E}\left( x^{\prime },i^{\prime },x\right) \right)
-\Omega \left( x\right) \geq \boldsymbol{\Omega} \left(
r(x^{\prime},i^{\prime},x)\right) -\boldsymbol{\Omega }\left( q_{E}\left(
x^{\prime },i^{\prime },x\right) \right)
\end{equation*}
When firm $x$ does not hire, \textbf{A-IN} applies, and so
\begin{equation*}
\boldsymbol{\Omega}(t_E(x^{\prime},i^{\prime},x)) = \Omega(x)
\end{equation*}
Similarly, when firm $x'$ is not poached, \textbf{A-IN} applies, and so
\begin{equation*}
\boldsymbol{\Omega}(r(x^{\prime},i^{\prime},x)) = \Omega(x^{\prime})
\end{equation*}

The combination of these conditions deliver \textbf{C-UE} and part of
\textbf{C-RT} ($EE$ Hires, $EE$ Threats and Retention):

\begin{enumerate}
\item The quit set of an employed worker is determined by
\begin{equation*}
\mathcal{Q}^{E}\left( x^{\prime },i^{\prime }\right) =\left\{ x\Bigg|\boldsymbol{%
\Omega }\left( h_{E}\left( x^{\prime },i^{\prime },x\right) \right) -\Omega
\left( x\right) \geq \Omega \left( x^{\prime }\right) -\boldsymbol{\Omega }%
\left( q_{E}\left( x^{\prime },i^{\prime },x\right) \right) \right\}
\end{equation*}

\item The worker's value of being hired from employment from firm $x^{\prime
}$ is
\begin{equation*}
\boldsymbol{V}(h_{E}(x,x^{\prime },i^{\prime }))=\Omega \left( x^{\prime
}\right) -\boldsymbol{\Omega }\left( q_{E}\left( x^{\prime },i^{\prime
},x\right) \right)
\end{equation*}

\item Worker $i^{\prime }$s value of being retained at $x^{\prime }$ after
meeting $x$ is\footnote{%
Because offers are made at no cost, both firms always make an offer, even
when they know that they cannot retain/hire the worker in equilibrium. This
is exactly the same as in Postel-Vinay Robin (2002).}
\begin{equation*}
\boldsymbol{V}(r(x^{\prime },i^{\prime },x),i^{\prime })=\boldsymbol{\Omega }\left(
h_{E}\left( x^{\prime },i^{\prime },x\right) \right) -\Omega \left( x\right)
\end{equation*}

\item The joint value of the potential poaching firm $x$ when the worker is
not hired does not change:
\begin{equation*}
\boldsymbol{\Omega}(t_E(x^{\prime},i^{\prime},x)) = \Omega(x)
\end{equation*}

\item The joint value of the potential poached firm $x^{\prime}$ does not
change when the worker stays:
\begin{equation*}
\boldsymbol{\Omega}(r(x^{\prime},i^{\prime},x)) = \Omega(x^{\prime})
\end{equation*}
\end{enumerate}

\subsubsection{Proof of C-EU ($EU$ Quits and layoffs)}

We first derive our expression for $\mathcal{L}$ on the firm side, and next our expression for $\mathcal{Q}^U$ on the worker side.

\paragraph{Part 1: Firm side}

Consider a firm $x$ who is considering laying off worker $i$ for whom $%
q_{U,i}\left( x\right) =0$. As above, we start with definitions, noting that
$n(s(\cdot)) \equiv n\left( s\left( x,\left( 1-\left[ \ell \left( x\right) ;\ell _{i}\left(
x\right) =1\right] \right) \circ \left( 1-q_{U}\left( x\right) \right)
\right) \right) $ is the number of workers if $i$ is laid off:
\begin{align*}
\Omega & =J+\sum_{j=1}^{n\left( s\left( \cdot \right) \right) }V_{j}+V_{i} &
\Omega ^{\ast }& =J^{\ast }+\sum_{j=1}^{n\left( s\left( \cdot \right)
\right) }V_{j}^{\ast }+V_{i}^{\ast } &
\Omega ^{\ast \ast }& =J^{\ast \ast }+\sum_{j=1}^{n\left( s\left( \cdot
\right) \right) }V_{j}^{\ast \ast }
\end{align*}
Note that in the first line the coalition has still worker $i$ in it. In the
second line, the firm and the worker $i$ have negotiated (and internal
negotiation has determined $V_{i}^{\ast }$ which is what $i$ will get if
they stay in the firm). In the third line, the worker has been fired and
another round of negotiation has occurred among incumbents.

The same result as in \textbf{Claim 1} from section \ref{ue} obtains: under
\textbf{A-BP}, $V_{j}^{\ast \ast }=V_{j}^{\ast }$. Using this result and the above definitions as before,
\begin{align*}
\Omega ^{\ast \ast }-\Omega &=\left[ J^{\ast \ast }-J^{\ast }\right] +\left[ \Omega ^{\ast }-\Omega %
\right] -V_{i}^{\ast }
\end{align*}
Using again \textbf{A-IN} to conclude that $\Omega ^{\ast }=\Omega$, we obtain
\begin{equation*}
\Omega ^{\ast \ast }-\Omega =\left[ J^{\ast \ast }-J^{\ast }\right]
-V_{i}^{\ast }
\end{equation*}
Now under \textbf{A-LC}, the firm $x$ will only layoff the worker if $%
J^{\ast \ast }>J^{\ast }$, which requires
\begin{equation*}
\Omega -\Omega ^{\ast \ast }<V_{i}^{\ast }
\end{equation*}
As long as $V_{i}^{\ast }>U$ the worker would be willing to renegotiate and transfer value
to the firm to avoid being laid off, implying
\begin{equation*}
\Omega -\Omega ^{\ast \ast }<U.
\end{equation*}%
which we can re-write%
\begin{equation*}
\Omega \left( s\left( x,\left( 1-\left[ \ell \left( x\right) ;\ell
_{i}\left( x\right) =1\right] \right) \circ \left( 1-q_{U}\left( x\right)
\right) \right) ,i\right) +U>\Omega \left( s\left( x,\left( 1-\left[ \ell
\left( x\right) ;\ell _{i}\left( x\right) =0\right] \right) \circ \left(
1-q_{U}\left( x\right) \right) \right) ,i\right)
\end{equation*}%
where the LHS is $\Omega ^{\ast \ast }+U$ (under the layoff) and the RHS is $%
\Omega .$ This concludes the proof for the firm side.

\paragraph{Part 2: Worker side}

Consider worker $i$ in firm $x$ who is considering quitting to unemployment
for whom $\ell _{i}\left( x\right) =0$. As above, we start with definitions,
noting that $n(s(\cdot)) \equiv n\left( s\left( x,\left( 1-\ell \left( x\right) \right) \circ
\left( 1-\left[ q_{U,-i}\left( x\right) ;q_{U,i}\left( x\right) =1\right]
\right) \right) \right) $ is the number of workers if $i$ quits. As before, we define
\begin{align*}
\Omega & =J+\sum_{j=1}^{n\left( s\left( \cdot \right) \right) }V_{j}+V_{i} &
\Omega ^{\ast }& =J^{\ast }+\sum_{j=1}^{n\left( s\left( \cdot \right)
\right) }V_{j}^{\ast }+V_{i}^{\ast } &
\Omega ^{\ast \ast }& =J^{\ast \ast }+\sum_{j=1}^{n\left( s\left( \cdot
\right) \right) }V_{j}^{\ast \ast }
\end{align*}
The same result as in \textbf{Claim 1} from section \ref{ue} obtains $%
V_{j}^{\ast \ast }=V_{j}^{\ast }$. Using this result and the above definitions,
\begin{align*}
\Omega ^{\ast \ast }-\Omega &=\left[ J^{\ast \ast }-J^{\ast }\right] +\left[ \Omega ^{\ast }-\Omega %
\right] -V_{i}^{\ast }
\end{align*}
Again, $ \Omega ^{\ast }=\Omega $ from \textbf{A-IN}, so that
\begin{equation*}
\Omega ^{\ast \ast }-\Omega =\left[ J^{\ast \ast }-J^{\ast }\right]
-V_{i}^{\ast }
\end{equation*}
Now under \textbf{A-LC}, worker $i$ will quit into unemployment iff $%
V_{i}^{\ast }<U$, which requires
\begin{equation*}
J^{\ast \ast }-J^{\ast }+[\Omega -\Omega ^{\ast \ast }]<U
\end{equation*}%
As long as $J^{\ast \ast }<J^{\ast }$, the firm is willing to transfer value
to worker $i$ to retain her. Therefore, worker $i$ quits into unemployment
iff the previous inequality holds at $J^{\ast \ast }=J^{\ast }$, i.e.
\begin{equation*}
\Omega -\Omega ^{\ast \ast }<U
\end{equation*}
Therefore, the worker quits iff
\begin{align*}
& \Omega \left( s\left( x,\left( 1-\ell \left( x\right) \right) \circ \left(
1- \left[ q_{U,-i}\left( x\right) ;q_{U,i}\left( x\right) =1\right] \right)
\right) ,i\right) +U \\
& > \Omega \left( s\left( x,\left( 1-\ell \left( x\right) \right) \circ
\left( 1-\left[ q_{U,-i}\left( x\right) ;q_{U,i}\left( x\right) =0\right]
\right) \right) ,i\right)
\end{align*}%
which concludes the proof of the worker side. This delivers \textbf{C-EU}.


\subsubsection{Proof of C-X (Exit)}

Consider a firm $x$ who contemplates exit after all endogenous quits and
layoffs, thus when its employment is $n\left( s\left( x,\kappa \left(
x\right) \right) \right) $. As before we define values conditional on
exiting:
\begin{align*}
\Omega & =J+\sum_{j=1}^{n\left( s\left( \cdot \right) \right) }V_{j} &
\Omega ^{\ast }& =J^{\ast }+\sum_{j=1}^{n\left( s\left( \cdot \right)
\right) }V_{j}^{\ast } &
\Omega ^{\ast \ast }& =J^{\ast \ast }+0
\end{align*}%
The joint value after exit is simply the value of the firm,
since all other workers have left because of exit. Following similar calculations as before,
\begin{eqnarray*}
\Omega ^{\ast \ast }-\Omega&=&\left[ J^{\ast \ast
}-J^{\ast }\right] +[\Omega ^{\ast }-\Omega ]-\sum_{j=1}^{n\left( s\left(
\cdot \right) \right) }V_{j}^{\ast }
\end{eqnarray*}
Again, $ \Omega ^{\ast }=\Omega $ from \textbf{A-IN}, so that
\begin{equation*}
\Omega ^{\ast \ast }-\Omega =\left[ J^{\ast \ast }-J^{\ast }\right]
-\sum_{j=1}^{n\left( s\left( \cdot \right) \right) }V_{j}^{\ast }
\end{equation*}

The firm exits iff $J^{\ast \ast }\geq J^{\ast }$, that is, $\vartheta \geq
J^{\ast }$. This is equivalent to
\begin{equation*}
\Omega ^{\ast \ast }-\Omega \geq -\sum_{j=1}^{n\left( s\left( \cdot \right)
\right) }V_{j}^{\ast }
\end{equation*}

Using again that $\Omega ^{\ast \ast }=J^{\ast \ast }=\vartheta $, the firm
exits iff
\begin{equation*}
\vartheta +\sum_{j=1}^{n\left( s\left( \cdot \right) \right) }V_{j}^{\ast
}\geq \Omega
\end{equation*}

Since any worker is better off under $V_{i}^{\ast }\geq U$ than unemployed, all workers are willing to take a value cut down to $U$
if $\vartheta \geq \Omega - \sum_{j=1}^{n\left( s\left( \cdot \right) \right)}V_{j}^{\ast }$ because then the firm can credibly exit. This observation implies that the firm exits if and only if
\begin{equation*}
\vartheta -\Omega \left( s\left( x,\kappa \left( x\right) \right) \right)
+n\left( s\left( x,\kappa \left( x\right) \right) \right) U\geq 0
\end{equation*}
This last equality proves \textbf{C-X (Exit)}: the set of $x$ such that the firm exits is
given by
\begin{equation*}
\mathcal{E}=\left\{ x\Big|\vartheta +n\left( s\left( x,\kappa \left(
x\right) \right) \right) \cdot U\geq \Omega \left( s\left( x,\kappa \left(
x\right) \right) \right) \right\}
\end{equation*}


\subsubsection{Proof of C-V (Vacancies) \label{app:vacancies}}

We split the proof in two steps. First, we show that workers are collectively willing to transfer value to the firm in exchange for the joint value-maximizing vacancy policy function. Second, we show that a single worker can create a system of transfers that achieves the same outcome. These transfers are equivalent to wage renegotiation, which explains why we have subsumed them in the wage function $w(x,i)$ in the equations above. Similarly to wages, these transfers drop out from the expression for the joint value.

\paragraph{Part 1: Collective transfers}

In this step, we show that workers are collectively better off transferring
value to the firm in exchange of the firm posting the joint value-maximizing
amount of vacancies.

The vacancy posting decision $v^{J}$ that maximizes firm value is:
\begin{equation*}
\frac{c_{v}\left( v^J\left( x\right) ,n\left( x\right) \right) }{q}=\phi %
\left[ \boldsymbol{J}\left( h_{U}\left( x\right) \right) -J\left( x\right) %
\right] \cdot \mathbb{I}_{\left\{ x\in \mathcal{A}\right\} }+\left( 1-\phi
\right) \int_{x\in \mathcal{Q}^{E}\left( x^{\prime },i^{\prime }\right) }%
\left[ \boldsymbol{J}\left( h_{E}\left( x^{\prime },i^{\prime },x\right) \right)
-J\left( x\right) \right] dH_n\left( x^{\prime },i^{\prime }\right) .
\end{equation*}
Similarly, define $v^{\Omega}$ be the policy that maximizes the value of the
coalition, and $v^{\overline{V}}$ be the policy that maximizes the value of all
the employees. Let $\Omega^{\gamma}$,$J^{\gamma},\overline{V}^{\gamma}$ be the
value of the coalition, firm and all workers under the $v^{\gamma}$, for $%
\gamma\in\left\{ \Omega,J,\overline{V}\right\} $. We now prove our claim in
several steps.

\subparagraph{Part 1-(a) Collective value gains.}

The policy $v^{\Omega}$ will lead to $\overline{V}%
^{\Omega}\geq\overline{V}^{J} + [J^J - J^\Omega]$ where $J^J - J^\Omega \geq 0$.%
\newline

\textit{Proof:} By construction $\Omega^{\Omega}$ is greater than $\Omega^{J}
$: $\Omega^{\Omega}\geq\Omega^{J}$. By definition: $\Omega^{\Omega}=J^{%
\Omega}+\overline{V}^{\Omega}$, and $\Omega^{J}=J^{J}+\overline{V}^{J}$. Use those
definitions to obtain inequality $J^{\Omega}+\overline{V}^{\Omega} \geq J^{J}+\overline{%
V}^{J}$, which can be re-arranged into $\overline{V}^{\Omega}-\overline{V}^{J} \geq
J^{J}-J^{\Omega} $. Since $J^{J}$ is the value under the optimal policy for $%
J$, then $J^{J}\geq J^{\Omega}$. The above then implies that
\begin{equation*}
\overline{V}^{\Omega}-\overline{V}^{J}\geq J^{J}-J^{\Omega}\geq0
\end{equation*}
This inequality implies that workers would be prepared to transfer $T=J^{J}-J^{\Omega}
\geq 0$ to the firm in order for the firm to pursue policy $v^{\Omega}$
instead of $v^{J}$. This concludes the proof of \textbf{Part 1-(a).}

\subparagraph{Part 1-(b) Infeasibility of $\overline{V}^{\overline{V}}$ .}

There does not exist an incentive-compatible transfer from workers to firm
that will lead to $\overline{V}^{\overline{V}}$.\newline

\textit{Proof:} Suppose workers consider transferring even more to induce
the firm to follow policy $v^{\overline{V}}$ that maximizes their value. By
construction $\Omega^{\Omega}\geq\Omega^{\overline{V}}$. Using definitions for
each of these, then $J^{\Omega}+\overline{V}^{\Omega}\geq J^{\overline{V}}+\overline{V}^{%
\overline{V}}$ . Rearranging this: $ J^{\Omega}-J^{\overline{V}}\geq\overline{V}^{\overline{V}}-%
\overline{V}^{\Omega}$. Since $\overline{V}^{\overline{V}}$ is the value under the optimal
policy for $\overline{V}$, then $\overline{V}^{\overline{V}}\geq\overline{V}^{\Omega}$. The
above then implies that
\begin{equation*}
J^{\Omega}-J^{\overline{V}}\geq\overline{V}^{\overline{V}}-\overline{V}^{\Omega}\geq0
\end{equation*}
Taking $v^{\Omega}$ as a baseline, the above implies that a change to $v^{%
\overline{V}}$ causes a loss of $J^{\Omega}-J^{\overline{V}}$ to the firm, which is
more than the gain of $\overline{V}^{\overline{V}}-\overline{V}^{\Omega}$ to the workers.
This implies that workers could transfer all of their gains under $v^{\overline{V}}$ to the
firm, but the firm would still not choose $v^{\overline{V}}$ over $v^{\Omega}$.
This concludes the proof of \textbf{Part 1-(b).}

\subparagraph{Part 1-(c) Optimality of $\overline{V}^\Omega$.}

There does not exist an incentive-compatible transfer from workers to firm
that will lead to $\overline{V}^{\ast}\in\left(\overline{V}^{\Omega},\overline{V}^{\overline{V}%
}\right)$.\newline

\textit{Proof:} Call such a policy $v^{\overline{V}\ast}.$ Then: $%
\Omega^{\Omega}\geq\Omega^{\overline{V}\ast}$ , and by definition,
\begin{eqnarray*}
J^{\Omega}-J^{\overline{V}^{\ast}} & \geq & \overline{V}^{\overline{V}^{\ast}}-\overline{V}%
^{\Omega}
\end{eqnarray*}
Since by definition $\overline{V}^{\ast}\in\left(\overline{V}^{\Omega},\overline{V}^{\overline{V}%
}\right)$ , then $\overline{V}^{\overline{V}^{\ast}}-\overline{V}^{\Omega}\geq0$. Therefore
\begin{equation*}
J^{\Omega}-J^{\overline{V}^{\ast}}\geq\overline{V}^{\overline{V}^{\ast}}-\overline{V}%
^{\Omega}\geq0
\end{equation*}
Taking $v^{\Omega}$ as a baseline, the above implies that a change to $v^{%
\overline{V}^{\ast}}$ causes a loss of $J^{\Omega}-J^{\overline{V}^{\ast}}$ to the
firm, which is more than the gain of $\overline{V}^{\overline{V}^{\ast}}-\overline{V}%
^{\Omega}$ to the workers. This concludes the proof of \textbf{Part 1-(c).}

\subparagraph{Part 1-(d) Conclusion.}

In summary, it is optimal for workers to transfer exactly $T=J^{J}-J^{\Omega}
$ to the firm, in order for the firm to pursue $v^{\Omega}$ instead of $v^J$%
. Further transfers to the firm would be required to have the firm pursue a
better policy for workers, but this is exceedingly costly to the firm and
the workers are unwilling to make a transfer to cover these costs. This
concludes the proof of \textbf{Step 1: Collective transfers.}

\paragraph{Part 2: Individual transfers}

In this step, we show that a single, randomly drawn worker can construct a
system of transfers that induces the firm to post $v^\Omega$ instead of $v^J$%
, while leaving all agents better off.

Within $dt$, consider the single, randomly drawn worker $j_0$. Consider the
following system of transfers. Worker $j_0$ makes a transfer $J^J-J^\Omega$
to the firm, in exchange of what (i) the firm posts $v^\Omega$ instead of $%
v^J$, and (ii) the worker gets a wage increase that gives her all the
differential surplus $\overline{V}^\Omega-\overline{V}^J$.

Following the same steps as in \textbf{Part 1: Collective transfers}, the
firm gets $J^\Omega + [J^J-J^\Omega] = J^J$ and is hence indifferent.
Similarly, workers $j \neq j_0$ do not get any value change, and are thus
indifferent Finally, worker $j_0$ gets a value increase of
\begin{equation*}
[\overline{V}^\Omega-\overline{V}^J ]- [J^J-J^\Omega] \geq 0
\end{equation*}
where the inequality similarly follows from \textbf{Part 1: Collective
transfers}. This concludes the proof of \textbf{Part 2: Individual transfers}%
.

\paragraph{Conclusion.}

The previous arguments show that a single worker has an incentive to and can
induce the firm to post $v^\Omega$. Notice also that the same argument holds
starting from any vacancy policy function $\widetilde{v} \neq v^J$ together with
a value of the firm $\widetilde{J}$. Thus, even if some worker induces the firm
to post a different vacancy policy function which is not $v^\Omega$ any
other worker has an incentive to induce the firm to post $v^\Omega$.
Therefore, in equilibrium, the firm posts $v^\Omega$, which concludes the
proof of \textbf{C-V}.


\subsection{Applying Conditions (C)}

Having established that \textbf{Assumption (A)} can be used to prove \textbf{%
Conditions (C)}, we now apply conditions \textbf{(C)} to the Bellman
equation for the joint value. The goal of this section is to show that for $x\in \mathcal{E}^c$ the complement of
the exit set, we can considerably simplify the recursion for the joint
value:
\begin{eqnarray}
\label{eq:JointValueArbitraryState}
\rho \Omega \left( x\right) &=&y\left( z(x),n\left( x\right) \right)
-c\left( v\left( x\right) ,n\left( x\right) ,z(x)\right) \nonumber \\
\text{Destructions} &&-\delta \sum\limits_{i=1}^{n\left( x\right) }\left[
\Omega \left( x\right) -\boldsymbol{\Omega }\left( d(x,i)\right) -U\right] \nonumber \\
\text{$UE$ Hires} &&+qv\left( x\right) \phi \left[ \boldsymbol{\Omega }\left(
h_{U}\left( x\right) \right) -\Omega \left( x\right) -U\right] \cdot \mathbb{%
I}_{\left\{ x\in \mathcal{A}\right\} } \nonumber \\
\text{$EE$ Hires} &&+qv\left( x\right) \left( 1-\phi \right) \int_{x\in
\mathcal{Q}^{E}\left( x^{\prime },i^{\prime }\right) }\left[ \left[ \boldsymbol{%
\Omega }\left( h_{E}\left( x^{\prime },i^{\prime },x\right) \right) -\Omega
\left( x\right) \right] -\left[ \Omega \left( x^{\prime }\right) -\boldsymbol{%
\Omega }\left( q_{E}\left( x^{\prime },i^{\prime },x\right) \right) \right] %
\right] dH_n\left( x^{\prime },i^{\prime }\right) \nonumber \\
\text{Shocks} &&+\Gamma[\boldsymbol{\Omega},\Omega]
\end{eqnarray}
with the sets
\begin{align}
\label{eq:SetsArbitraryState}
\mathcal{Q}^{U} &=\Big\{ (x,i)\Big|\Omega \left( s\left( x,\left( 1-\ell
\left( x\right) \right) \circ \left( 1-\left[ q_{U,-i}\left( x\right)
;q_{U,i}\left( x\right) =1\right] \right) \right) ,i\right) +U  \nonumber \\
& \hskip1.5cm >\Omega \left( s\left( x,\left( 1-\ell \left( x\right) \right)
\circ \left( 1-\left[ q_{U,-i}\left( x\right) ;q_{U,i}\left( x\right) =0%
\right] \right) \right) ,i\right) \Big\} \\
\mathcal{L} &=\Big\{ (x,i)\Big|\Omega \left( s\left( x,\left( 1-\left[ \ell
\left( x\right) ;\ell _{i}\left( x\right) =1\right] \right) \circ \left(
1-q_{U}\left( x\right) \right) \right) ,i\right) +U  \nonumber \\
& \hskip1.5cm >\Omega \left( s\left( x,\left( 1-\left[ \ell \left( x\right)
;\ell _{i}\left( x\right) =0\right] \right) \circ \left( 1-q_{U}\left(
x\right) \right) \right) ,i\right) \Big\} \\
\mathcal{E} &=\left\{ x\Big|\vartheta +n\left( s(x,\kappa(x))\right) \cdot
U\geq \Omega (s(x,\kappa(x)))\right\} \\
\mathcal{A}& =\left\{ x\Big|\boldsymbol{\Omega }\left( h_{U}\left( x\right)
\right) -\Omega \left( x\right) \geq U\right\} \\
\mathcal{Q}^{E}\left( x^{\prime },i^{\prime }\right) & =\left\{ x\Bigg|%
\boldsymbol{\Omega }\left( h_{E}\left( x^{\prime },i^{\prime },x\right) \right)
-\Omega \left( x\right) \geq \Omega \left( x^{\prime }\right) -\boldsymbol{%
\Omega }\left( q_{E}\left( x^{\prime },i^{\prime },x\right) \right) \right\}
\end{align}%
and---as per \textbf{(C-V)---}the vacancy policy $v\left( x\right) $ is
given by the solution to the following:
\begin{eqnarray}
\label{eq:VacanciesArbitraryState}
\frac{c_{v}\left( v\left( x\right) ,n\left( x\right) \right) }{q}&=&\phi %
\left[ \boldsymbol{\Omega }\left( h_{U}\left( x\right) \right) -\Omega \left(
x\right) \right] \cdot \mathbb{I}_{\left\{ x\in \mathcal{A}\right\} } \nonumber \\
& +&\left( 1-\phi \right) \int_{x\in \mathcal{Q}^{E}\left( x^{\prime
},i^{\prime }\right) }\left[ \left[ \boldsymbol{\Omega }\left( h_{E}\left(
x^{\prime },i^{\prime },x\right) \right) -\Omega \left( x\right) \right] -%
\left[ \Omega \left( x^{\prime }\right) -\boldsymbol{\Omega }\left( q_{E}\left(
x^{\prime },i^{\prime },x\right) \right) \right] \right]dH_n\left( x^{\prime
},i^{\prime }\right)
\end{eqnarray}%
In continuous time, the exit decision is captured by $x\in \mathcal{E}$. The
Bellman equation above holds exactly for $x\in \mathcal{E}^{c}$. Exit is
accounted for in the \textquotedblleft bold\textquotedblright\ continuation
values, which all include the possible exit decision should the firm's state
fall into $\mathcal{E}$ after an event.


We first proceed one term at the time, working
through (B.4.1) exogenous destructions, (B.4.2) retentions, (B.4.3) $EE$ (poached)
quits, (B.4.4) $UE $ hires, (B.4.5) $UE$ threats, (B.4.6) $EE$ (poached) hires, and (B.4.7) $%
EE$ threats.

\subsubsection{Exogenous destructions}

\begin{align*}
\text{Destructions}
& =\sum\limits _{i=1}^{n\left(x\right)}\delta\left[\boldsymbol{J}%
\left(d(x,i)\right)+\sum_{j=1}^{n\left(d(x,i)\right)}\boldsymbol{V}%
\left(d(x,i),j\right)+U-\Omega\left(x\right)\right] =\sum\limits _{i=1}^{n\left(x\right)}\delta\left[\boldsymbol{\Omega}%
\left(d(x,i)\right)+U-\Omega\left(x\right)\right] \\
\end{align*}
where we simply have used the definition $\boldsymbol{\Omega}\left(d(x,i)%
\right):=\boldsymbol{J}\left(d(x,i)\right)+\sum_{j=1}^{n\left(d(x,i)\right)}%
\boldsymbol{V}\left(d(x,i),j\right)$.

\subsubsection{Retentions}

\begin{align*}
\text{Retentions} & =\lambda^{E}\sum\limits
_{i=1}^{n\left(x\right)}\int_{x^{\prime}\notin\mathcal{Q}^E\left(x,i\right)}%
\left[\boldsymbol{J}\left(r\left(x,i,x^{\prime}\right)\right)+\sum_{j=i}^{n%
\left(x\right)}\boldsymbol{V}\left(r\left(x,i,x^{\prime}\right),j\right)-\Omega%
\left(x\right)\right]dH_v\left(x^{\prime}\right) \\
& =\lambda^{E}\sum\limits _{i=1}^{n\left(x\right)}\int_{x^{\prime}\notin%
\mathcal{Q}^E\left(x,i\right)}\left[\boldsymbol{\Omega}\left(r\left(x,i,x^{%
\prime}\right)\right)-\Omega\left(x\right)\right]dH_v\left(x^{\prime}\right)
\end{align*}
where we simply have used the definition $\boldsymbol{\Omega}\left(r%
\left(x,i,x^{\prime}\right)\right) =\boldsymbol{J}\left(r\left(x,i,x^{\prime}%
\right)\right)+\sum_{j=i}^{n\left(x\right)}\boldsymbol{V}\left(r\left(x,i,x^{%
\prime}\right),j\right)$. Now using the result in \textbf{C-RT} that
\begin{equation*}
\boldsymbol{\Omega}\left(r\left(x,i,x^{\prime}\right)\right) = \Omega(x^{\prime})
\end{equation*}
we obtain that
\begin{equation*}
\text{Retentions} = 0
\end{equation*}

\subsubsection{$EE$ Quits}

\begin{equation*}
\text{$EE$ Quits}=\lambda ^{E}\sum\limits_{i=1}^{n\left( x\right)
}\int_{x^{\prime }\in \mathcal{Q}^{E}\left( x,i\right) }\left[ \boldsymbol{J}%
\left( q_{E}\left( x,i,x^{\prime }\right) \right) +\boldsymbol{V}\left(
q_{E}\left( x,i,x^{\prime }\right) ,i\right) +\sum_{j\neq i}^{n\left(
x\right) }\boldsymbol{V}\left( q_{E}\left( x,i,x^{\prime }\right) ,j\right)
-\Omega \left( x\right) \right] dH_v\left( x^{\prime }\right)
\end{equation*}%
Now by definition
\begin{align*}
\boldsymbol{\Omega }\left( q_{E}\left( x,i,x^{\prime }\right) \right) & =\boldsymbol{%
J}\left( q_{E}\left( x,i,x^{\prime }\right) \right) +\sum_{j=1}^{n\left(
q_{E}\left( x,i,x^{\prime }\right) \right) }\boldsymbol{V}\left( q_{E}\left(
x,i,x^{\prime }\right) ,j\right)  =\boldsymbol{J}\left( q_{E}\left( x,i,x^{\prime }\right) \right) +\sum_{j\neq
i}^{n\left( x\right) }\boldsymbol{V}\left( q_{E}\left( x,i,x^{\prime }\right)
,j\right)
\end{align*}%
Using this last equality in the term in square brackets
\begin{equation*}
\text{$EE$ Quits}=\lambda ^{E}\sum\limits_{i=1}^{n\left( x\right)
}\int_{x^{\prime }\in \mathcal{Q}^{E}\left( x,i\right) }\left[ \Omega \left(
q_{E}\left( x,i,x^{\prime }\right) \right) -\Omega \left( x\right) +\boldsymbol{V%
}\left( q_{E}\left( x,i,x^{\prime }\right) ,i\right) \right] dH_v\left(
x^{\prime }\right)
\end{equation*}%
Using \textbf{C-EE}, the value going to the poached worker is \textbf{$%
\boldsymbol{V}\left( q_{E}\left( x,i,x^{\prime }\right) \right) =\Omega \left(
x\right) -\Omega \left( q_{E}\left( x,i,x^{\prime }\right) \right) $}.
Substituting this into the last equation, we observe that the term in the
square brackets is zero, and so
\begin{equation*}
\text{$EE$ Quits}=0
\end{equation*}

\subsubsection{$UE$ Hires}

\begin{equation*}
\text{$UE$ Hires}=qv\left( x\right) \phi \left[ \boldsymbol{J}\left( h_{U}\left(
x\right) \right) +\sum\limits_{i=1}^{n\left( x\right) }\boldsymbol{V}\left(
h_{U}\left( x\right) ,i\right) -\Omega \left( x\right) \right] \cdot \mathbb{%
I}_{\left\{ x\in \mathcal{A}\right\} }
\end{equation*}%
Now, by definition
\begin{align*}
\boldsymbol{\Omega }\left( h_{U}\left( x\right) \right) & =\boldsymbol{J}\left(
h_{U}\left( x\right) \right) +\sum\limits_{i=1}^{n\left( h_{U}\left(
x\right) \right) }\boldsymbol{V}\left( h_{U}\left( x\right) ,i\right) =\boldsymbol{J}\left( h_{U}\left( x\right) \right) +\sum\limits_{i=1}^{n\left(
x\right) }\boldsymbol{V}\left( h_{U}\left( x\right) ,i\right) +\boldsymbol{V}\left(
h_{U}\left( x\right) ,i\right)
\end{align*}
and so, re-arranging,
\begin{align*}
\boldsymbol{J}\left( h_{U}\left( x\right) \right) +\sum\limits_{i=1}^{n\left(
x\right) }\boldsymbol{V}\left( h_{U}\left( x\right) ,i\right) & =\boldsymbol{\Omega }%
\left( h_{U}\left( x\right) \right) -\boldsymbol{V}\left( h_{U}\left( x\right)
,i\right)
\end{align*}%
Substituting this last equation into the term in the square brackets of the
first equation,
\begin{equation*}
\text{$UE$ Hires}=qv\left( x\right) \phi \left[ \boldsymbol{\Omega }\left(
h_{U}\left( x\right) \right) -\Omega \left( x\right) -\boldsymbol{V}\left(
h_{U}\left( x\right) ,i\right) \right] \cdot \mathbb{I}_{\left\{ x\in
\mathcal{A}\right\} }
\end{equation*}%
Following \textbf{C-UE}, the value going to the hired worker is $\boldsymbol{V}%
\left( h_{U}\left( x\right) ,i\right) =U$. Substituting in:
\begin{equation*}
\text{$UE$ Hires}=qv\left( x\right) \phi \left[ \boldsymbol{\Omega }\left(
h_{U}\left( x\right) \right) -\Omega \left( x\right) -U\right] \cdot \mathbb{%
I}_{\left\{ x\in \mathcal{A}\right\} }
\end{equation*}

\subsubsection{ $UE$ Threats}

\begin{eqnarray*}
\text{$UE$ Threats} & = & qv\left( x\right) \phi \left[ \boldsymbol{J}\left(
t_{U}\left( x\right) \right) +\sum\limits_{i=1}^{n\left( x\right) }\boldsymbol{V}%
\left( t_{U}\left( x\right) ,i\right) -\Omega \left( x\right) \right] \cdot
\mathbb{I}_{\left\{ x\notin \mathcal{A}\right\}}
\end{eqnarray*}
Using the definition of $\boldsymbol{\Omega}(t_U(x))$, we can re-write this term
as
\begin{eqnarray*}
\text{$UE$ Threats} & = & qv\left( x\right) \phi \left[ \boldsymbol{\Omega}%
\left( t_{U}\left( x\right) \right) - \Omega(x) \right] \cdot \mathbb{I}%
_{\left\{ x\notin \mathcal{A}\right\}}
\end{eqnarray*}
Now using our result in condition \textbf{C-UE} that $\boldsymbol{\Omega}\left(
t_{U}\left( x\right) \right) = \Omega(x)$, we can conclude that
\begin{equation*}
\text{$UE$ Threats} = 0
\end{equation*}

\subsubsection{$EE$ Hires}

\begin{equation*}
\text{$EE$ Hires}=qv\left( x\right) \left( 1-\phi \right) \int_{x\in
\mathcal{Q}^{E}\left( x^{\prime },i^{\prime }\right) }\left[ \boldsymbol{J}%
\left( h_{E}\left( x^{\prime },i^{\prime },x\right) \right)
+\sum\limits_{i=1}^{n\left( x\right) }\boldsymbol{V}\left( h_{E}\left( x^{\prime
},i^{\prime },x\right) ,i\right) -\Omega \left( x\right) \right] dH_n\left(
x^{\prime },i^{\prime }\right)
\end{equation*}%
Now by definition
\begin{align*}
\boldsymbol{\Omega }\left( h_{E}\left( x^{\prime },i^{\prime },x\right) \right)
& =\boldsymbol{J}\left( h_{E}\left( x^{\prime },i^{\prime },x\right) \right)
+\sum\limits_{i=1}^{n\left( h_{E}\left( x^{\prime },i^{\prime },x\right)
\right) }\boldsymbol{V}\left( h_{E}\left( x^{\prime },i^{\prime },x\right)
,i\right) \\
& =\left[ \boldsymbol{J}\left( h_{E}\left( x^{\prime },i^{\prime },x\right)
\right) +\sum\limits_{i=1}^{n\left( x\right) }\boldsymbol{V}\left( h_{E}\left(
x^{\prime },i^{\prime },x\right) ,i\right) \right] +\boldsymbol{V}\left(
h_{E}\left( x^{\prime },i^{\prime },x\right) ,i\right)
\end{align*}
which can be re-arranged into
\begin{align*}
\boldsymbol{J}\left( h_{E}\left( x^{\prime },i^{\prime },x\right) \right)
+\sum\limits_{i=1}^{n\left( x\right) }\boldsymbol{V}\left( h_{E}\left( x^{\prime
},i^{\prime },x\right) ,i\right) & =\boldsymbol{\Omega }\left( h_{E}\left(
x^{\prime },i^{\prime },x\right) \right) -\boldsymbol{V}\left( h_{E}\left(
x^{\prime },i^{\prime },x\right) ,i\right)
\end{align*}%
Using this in the term in the square brackets
\begin{equation*}
\text{$EE$ Hires}=qv\left( x\right) \left( 1-\phi \right) \int_{x\in
\mathcal{Q}^{E}\left( x^{\prime },i^{\prime }\right) }\left[ \boldsymbol{\Omega }%
\left( h_{E}\left( x^{\prime },i^{\prime },x\right) \right) -\Omega \left(
x\right) -\boldsymbol{V}\left( h_{E}\left( x^{\prime },i^{\prime },x\right)
,i\right) \right] dH_n\left( x^{\prime },i^{\prime }\right)
\end{equation*}%
Under \textbf{C-EE}, the value going to the hired worker is $\boldsymbol{V}%
\left( h_{E}\left( x^{\prime },i^{\prime },x\right) ,i\right) =\Omega \left(
x^{\prime }\right) -\boldsymbol{\Omega }\left( q_{E}\left( x^{\prime },i^{\prime
},x\right) \right) $. Substituting this in:
\begin{equation*}
\text{$EE$ Hires}=qv\left( x\right) \left( 1-\phi \right) \int_{x\in
\mathcal{Q}^{E}\left( x^{\prime },i^{\prime }\right) }\left[ \left[ \boldsymbol{%
\Omega }\left( h_{E}\left( x^{\prime },i^{\prime },x\right) \right) -\Omega
\left( x\right) \right] -\left[ \Omega \left( x^{\prime }\right) -\boldsymbol{%
\Omega }\left( q_{E}\left( x^{\prime },i^{\prime },x\right) \right) \right] %
\right] dH_n\left( x^{\prime },i^{\prime }\right)
\end{equation*}

\subsubsection{$EE$ Threats}

\begin{eqnarray*}
\text{$EE$ Threats} &=&qv\left( x\right) \left( 1-\phi \right) \int_{x\notin
\mathcal{Q}^{E}\left( x^{\prime },i^{\prime }\right) }\left[ \boldsymbol{J}%
\left( t_{E}\left( x^{\prime },i^{\prime },x\right) \right)
+\sum\limits_{i=1}^{n\left( x\right) }\boldsymbol{V}\left( t_{E}\left( x^{\prime
},i^{\prime },x\right) ,i\right) -J\left( x\right) -\overline{V}\left( x\right) %
\right] dH_n\left( x^{\prime },i^{\prime }\right)
\end{eqnarray*}
Using the definition of $\boldsymbol{\Omega}(t_E(x^{\prime},i^{\prime},x))$, we
obtain
\begin{eqnarray*}
\text{$EE$ Threats} &=&qv\left( x\right) \left( 1-\phi \right) \int_{x\notin
\mathcal{Q}^{E}\left( x^{\prime },i^{\prime }\right) }\left[ \boldsymbol{\Omega}%
\left( t_{E}\left( x^{\prime },i^{\prime },x\right) \right) - \Omega(x) %
\right] dH_n\left( x^{\prime },i^{\prime }\right)
\end{eqnarray*}
Now using the result in condition \text{C-RT} that $\boldsymbol{\Omega}\left(
t_{E}\left( x^{\prime },i^{\prime },x\right) \right) = \Omega(x) $, we
obtain that
\begin{equation*}
\text{$EE$ Threats} = 0
\end{equation*}


\subsection{Reducing the state space}

We have obtained the simplified recursion \eqref{eq:JointValueArbitraryState}-\eqref{eq:VacanciesArbitraryState}. Inspection of the system \eqref{eq:JointValueArbitraryState}-\eqref{eq:VacanciesArbitraryState} reveals that the only payoff-relevant states are $(z,n)$, and the details of the within-firm contractual structure do not affect allocations. Any extra information contained in $x$ beyond $(z,n)$ would be redundant given $(z,n)$.

Therefore, it is straightforward to see that we may express the exit and separation decisions as
\begin{eqnarray}
\bm{\Omega }(z,n)&=& \mathbb{I}_{\{(z,n)\in\mathcal{E}\}} \bigg\{%
\:\vartheta +nU \bigg\} + \mathbb{I}_{\{(z,n)\in\mathcal{Q}^U\}}\bigg\{\:%
\bm{\Omega}(z,n-1)+U\bigg\} +  \mathbb{I}_{\{(z,n)\notin \mathcal{Q}^U\cup%
\mathcal{E}\}}\Omega(z,n),  \label{eq:D_omegabold} \\
\text{where}\quad\mathcal{E} &=&\big\{n,z\big|\vartheta +nU> {\Omega }(z,n)%
\big\},  \notag \\
\mathcal{Q}^{U}&=&\big\{ z,n\big|\bm{\Omega }\left(z,n-1\right)+U
>\Omega\left( z,n\right)\big\}.  \notag
\end{eqnarray}
The first expression is the value of exit. A firm that does not exit,
chooses whether to separate with a worker or not. If separating with a
worker, the firm re-enters $(\ref{eq:D_omegabold})$ with $\bm{\Omega}%
(z,n-1)$, having dispatched with a worker with value $U$, and again choosing
whether to exit, fire another worker, or continue. Iterating on this
procedure delivers
\begin{equation}  \label{eq:D_omegabold2}
\bm{\Omega }(z,n)= \max \bigg\{\vartheta +nU\:, \max_{s\in[0,\dots,n]%
}\Omega(z,n-s)+sU \bigg\}.
\end{equation}

Second, the post-exit/separation decision joint value is given by the
Bellman equation
\begin{eqnarray*}
\rho \Omega \left( z,n\right) &=& \max_{v\geq 0} y\left( z,n\right) -c\left(
v,n,z\right)  \label{eq:D_omega2} \\
\text{Destruction} &+&\:\:\delta n\Big\{ \big(\bm{\Omega }\left(
z,n-1\right) +U\big) - \Omega\left(z,n\right)\Big\}  \notag \\
\text{$UE$ Hire} &+&\:\:\phi q(\theta)v \cdot\mathbb{I}_{\left\{ \left(
z,n\right) \in \mathcal{A}\right\} }\cdot\Big\{ \bm{\Omega }\left(
z,n+1\right) -\big(\Omega \left( z,n\right)+U\big)\Big\}  \notag \\
\text{$EE$ Hire} &+&\:\:\left( 1-\phi \right)q(\theta)v \int_{\left(
z,n\right) \in \mathcal{Q}^E\left( z^{\prime },n^{\prime }\right) }\Big\{%
\left[ \bm{\Omega }\left(z, n+1\right) -\Omega \left( z,n\right) \right]
-\left[ \Omega \left( z^{\prime },n^{\prime }\right) -\bm{\Omega }\left(
z^{\prime },n^{\prime }-1\right) \right] \Big\}dH_n\left( z^{\prime
},n^{\prime }\right)  \notag \\
\text{Shock} &+&\:\:{\Gamma}_{z}\left[ \bm{\Omega },\Omega \right]
\left( z,n\right),  \notag  \\
%\end{eqnarray*}
%\begin{eqnarray*}
\text{where}\quad\mathcal{A}&=&\left\{ z,n\big|\bm{\Omega }\left(
z,n+1\right) \geq \Omega \left( z,n\right)+U\right\},  \notag \\
\mathcal{Q}^{E}\left( z^{\prime },n^{\prime }\right) &=&\big\{ z,n\big|%
\bm{\Omega }\left(z, n+1\right) -\Omega \left( z,n\right) \geq \Omega
\left( z^{\prime },n^{\prime }\right) -\bm{\Omega }\left( z^{\prime
},n^{\prime }-1\right) \big\} .  \notag
\end{eqnarray*}
Finally, firms enter if and only if
\begin{equation}  \label{eq:entry-Omega}
\int \bm{\Omega }\left( z,0\right) d\Pi _{0}(z)\geq c_e.
\end{equation}%
This condition pins down the entry rate per unit of time.\footnote{%
Recall that $J_0 = -c_e + \int \bm{J}(x_0)d\Pi(z_0)$. Given $%
\Omega(z_0,0) = J(z_0,0)$, we have $J_0 = -c_e + \int \bm{\Omega}%
(z_0,0)d\Pi(z_0)$. Free-entry implies $J_0=0$, which delivers $(\ref%
{eq:entry-Omega})$.} Details of the proof are available upon request.

\subsection{Continuous workforce limit}

Up to this point the economy has featured a continuum of firms, but an
integer-valued workforce. We now take the continuous workforce limit by
defining the `size' of a worker---which is 1 in the integer case---and
taking the limit as this approaches zero. Specifically, denote the ``size''
of a worker by $\Delta$, such that $n= N \Delta$ where $N$ is the old
integer number of workers. Now define ${\Omega}^\Delta(z,n) :=
\Omega(z,n/\Delta)$, and likewise define $y^{\Delta}(z,n):=y(z,n/\Delta)$ and
$c^\Delta(v,n,z):=c(v/\Delta,n/\Delta,z)$. We also define ${b}^\Delta:=b/\Delta
$ and ${\vartheta}^{\Delta}:=\vartheta/\Delta$. These imply, for example,
that $\Omega(z,N)={\Omega}^\Delta(z,N\Delta)$. Substituting these terms into
$(\ref{eq:D_omegabold2})$ and $(\ref{eq:D_omega2})$, and taking the limit $%
\Delta\rightarrow 0$, while holding $n=N\Delta$ fixed, we would obtain a
version of $(\ref{eq:finalBellman})$ in which all functions have the $\Delta$
super-script notation. We also specialize the productivity to a diffusion
process $dz_t=\mu(z_t)dt+\sigma(z_t)dW_t$.

The result is the joint value representation of Section 3: a Hamilton-Jacobi-Bellman (HJB) equation for the joint value \emph{conditional on the firm and its workers operating}:
\begin{eqnarray}  \label{eq:finalBellman}
\rho \Omega\left(z,n\right) =\max_{v \geq 0} && y\left(z,n\right) -c\left(
v,n,z\right) \\
\text{Destruction} &&-\delta n [\Omega_n(z,n) - U]  \notag \\
\text{$UE$ Hire} &&+\phi q(\theta)v \left[ \Omega_n(z,n) -U \right]  \notag
\\
\text{$EE$ Hire} &&+(1-\phi)q(\theta) v \int \max\bigg\{\Omega_n(z,n) -
\Omega_n(n^{\prime},z^{\prime}) \ , \ 0 \bigg\}d{H_n}\left( z^{\prime
},n^{\prime}\right)  \notag \\
\text{Shock} &&+\mu(z) \Omega_z(z,n) + \frac{\sigma(z)^2}{2}
\Omega_{zz}(z,n).  \notag
\end{eqnarray}
Boundary conditions for the firm and its workers operating require the state to be interior to the exit and separation boundaries:
\begin{eqnarray*}
\text{Exit boundary:}&& \Omega(z,n) \geq \vartheta + n U,
\quad\quad\quad\quad\quad\quad\quad  \notag \\
\text{Layoff boundary:}&& \Omega_n(z,n) \geq U
\quad\quad\quad\quad\quad\quad\quad  \notag
\end{eqnarray*}
Note the absence of $\bm{\Omega}$ terms. Since the value we
 track is that of a hiring firm subject to boundary conditions, then $%
\bm{\Omega}=\Omega$. This admits the simplification of `Shock' terms
we noted when discussing $(\ref{eq:D_diffusion})$.\\

We proceed in three steps:

\begin{enumerate}
\item[(B.6.1)] Define worker size and the renormalization

\item[(B.6.2)] Take the limit as worker size goes to zero

\item[(B.6.3)] Introduce a continuous productivity process.
\end{enumerate}

\subsubsection{Define worker size and the renormalization}

We denote the ``size'' of a worker by $\Delta$.
That is, we currently have an integer work-force $n\in \{1,2,3,\dots\}$.
We now consider $m \in \{\Delta,2\Delta,3\Delta,\dots\}$.
So then $n = m/\Delta$.
We use this to make the following normalizations:
\begin{align*}
\omega(z,m) &= \Omega \left( \frac{m}{\Delta}, z \right) & \mathcal{Y}(z,m) &= y \left( \frac{m}{\Delta}, z \right) & \mathcal{C}(z,m) &= c \left( \frac{v}{\Delta} ,\frac{m}{\Delta}, z \right)
\end{align*}
These definition imply
\begin{align*}
\Omega(z,n) &= \omega(n\Delta,z) &
y(z,n) &= \mathcal{Y}(n\Delta,z) &
c(v,z,n) &= \mathcal{C}(v\Delta,n\Delta,z)
\end{align*}
In addition, the value of unemployment solves $\rho U = b$. Define
\begin{equation*}
\mathcal{U} = \frac{b}{\rho \Delta} = \frac{U}{\Delta}
\end{equation*}
and
\begin{equation*}
\theta = \frac{\vartheta}{\Delta}
\end{equation*}

\noindent Substituting these definitions into the Bellman equation, we
obtain
{\small
\begin{eqnarray*}
\hspace*{-1.9cm}\rho \omega \left( n\Delta ,z\right) =\max_{v\Delta\geq 0} && \mathcal{Y}
\left( n\Delta,z\right) -\mathcal{C}\left( v\Delta,n\Delta,z\right) \\
\text{Destructions} &&-\delta n\Delta \left[ \frac{\omega \left(
n\Delta,z\right) -\pmb{\omega}\left( n\Delta-\Delta,z\right)}{\Delta} -
\mathcal{U}\right] \\
\text{$UE$ Hires} &&+qv\Delta \phi \left[ \frac{\pmb{\omega}\left(
n\Delta+\Delta,z\right) -\omega \left( n\Delta,z\right)}{\Delta} -\mathcal{U}%
\right] \cdot \mathbb{I}_{\left\{ \left( n\Delta,z\right) \in \mathcal{A}%
\right\} } \\
\text{$EE$ Hires} &&+qv\Delta \left( 1-\phi \right) \int_{\left(
n\Delta,z\right) \in \mathcal{Q}^E\left( n^{\prime }\Delta,z^{\prime }\right) }%
\Bigg[\frac{ \pmb{\omega}\left( n\Delta+\Delta,z\right) -\omega \left(
n\Delta,z\right) }{\Delta} -\frac{ \omega \left( n^{\prime }\Delta,z^{\prime
}\right) -\pmb{\omega}\left( n^{\prime }\Delta-\Delta,z^{\prime }\right)%
}{\Delta} \Bigg]d\widetilde{H_n}\left( n^{\prime}\Delta,z^{\prime }\right) \\
\text{Shocks} &&+\Gamma_z\left[ \pmb{\omega},\omega \right]\left(
n\Delta,z\right)
\end{eqnarray*}}
with the set definitions
\begin{align*}
\mathcal{E} & =\Bigg\{ n\Delta,z\Bigg| \max_{k\Delta \in
\{0,...,n\Delta\}}\omega (k\Delta,z) +(n\Delta-k\Delta) \mathcal{U} < \theta
+ n \Delta \mathcal{U} \Bigg\} \\
\mathcal{A}& =\left\{ n \Delta,z\Bigg| \frac{\pmb{\omega}\left(
n\Delta+\Delta,z\right) -\omega \left( n\Delta,z\right)}{\Delta} \geq
\mathcal{U}\right\} \\
\mathcal{Q}^U & =\left\{ n\Delta,z\Bigg| \frac{\omega \left(
n\Delta,z\right) - \pmb{\omega}\left( n\Delta-\Delta,z\right)}{\Delta} \leq
\mathcal{U}\right\} \\
\mathcal{Q}^{E}\left( n^{\prime }\Delta,z^{\prime }\right) & =\left\{
n\Delta,z\Bigg|\frac{\pmb{\omega}\left( n\Delta+\Delta,z\right) -\omega
\left( n\Delta,z\right)}{\Delta} \geq \frac{\omega \left( n^{\prime
}\Delta,z^{\prime }\right) -\pmb{\omega}\left( n^{\prime
}\Delta-\Delta,z^{\prime }\right)}{\Delta} \right\}
\end{align*}%
and the definition:
\begin{eqnarray*}
\pmb{\omega}(n\Delta,z) &=& \max \Bigg\{ \max_{k\Delta \in
\{0,...,n\Delta\}} \omega(k\Delta,z) + (n\Delta-k\Delta) \mathcal{U} \ , \
\theta + n \Delta\mathcal{U} \Bigg\}
\end{eqnarray*}

\subsubsection{Continuous limit as worker size goes to zero}

Now we take the limit $\Delta \to 0$, holding $m = n \Delta$ fixed. We note $%
\widehat{v} = \lim_{\Delta \to 0} v \Delta$. We see derivatives appear. We
denote $\omega_m(z,m) = \frac{\partial \omega}{\partial m}(z,m)$. \newline

First, we note that the following limit obtains:
\begin{eqnarray*}
\pmb{\omega}(z,m) &=& \max \Bigg\{ \max_{k \in [0,m]} \omega(k,z) + (m-k)
\mathcal{U} \ , \ \theta + m \Delta\mathcal{U} \Bigg\}
\end{eqnarray*}
In particular, the exit set limits to
\begin{align*}
\mathcal{E} & =\Bigg\{ z,m\Bigg| \max_{k \in [0,m]} \omega(k,z) + (m-k)
\mathcal{U} < \theta + m \mathcal{U} \Bigg\}
\end{align*}
In equilibrium, the $\pmb{\omega}(z,m)$ terns on the right-hand-side of
the Bellman equation are the result of endogenous quits, layoffs and hires.
Because our continuous time assumption has been made \textit{before} we take
the limit to a continuous workforce limit, we need only consider those
changes in the workforce one at a time. Hence, for any $(z,m) \in \text{%
Interior}(\mathcal{E}^c \cap \mathcal{A})$, the \emph{interior} of the
continuation set, there is always $\overline{ \Delta} > 0$: such that for any $%
\Delta \leq \overline{\Delta}$:
\begin{equation*}
\pmb{\omega}(m\pm \Delta,z) = \omega(m\pm \Delta,z)
\end{equation*}
Using this observation in the Bellman equation, we can obtain derivatives on
the right-hand-side. We obtain, for pairs $(z,n)$ in the interior of the
continuation set $(z,n) \in \text{Interior}(\mathcal{E}^c \cap \mathcal{A})$%
:
\begin{eqnarray*}
\rho \omega \left( z,m\right) =\max_{\widehat{v}\geq 0} && \mathcal{Y} \left(
z,m\right) -\mathcal{C}\left( \widehat{v},z,m\right) \\
\text{Destructions} &&-\delta m [\omega_m(z,m) - \mathcal{U}] \\
\text{$UE$ Hires} &&+q\widehat{v} \phi \left[ \omega_m(z,m) -\mathcal{U}\right]
\cdot \mathbb{I}_{\left\{ \left( z,m\right) \in \mathcal{A}\right\} } \\
\text{$EE$ Hires} &&+q\widehat{v} \left( 1-\phi \right) \int_{\left( z,m\right)
\in \mathcal{Q}^E\left( m^{\prime},z^{\prime }\right) }\Bigg[\omega_m(z,m) -
\omega_m(m^{\prime},z^{\prime}) \Bigg]d\widetilde{H_n}\left( m^{\prime},z^{\prime
}\right) \\
\text{Shocks} &&+\Gamma_z\left[\pmb{\omega},\omega\right](z,n)
\end{eqnarray*}%
with the set definitions
\begin{align*}
\mathcal{E} & =\Bigg\{ z,m\Bigg| \max_{k \in [0,m]} \omega(k,z) + (n-k)
\mathcal{U} < \theta + m \mathcal{U} \Bigg\} &
\mathcal{A}& =\left\{ z,m\Bigg| \omega_m(z,m) \geq \mathcal{U}\right\} \\
\mathcal{Q}^U & =\left\{ z,m\Bigg| \omega_m(z,m) \leq \mathcal{U}\right\} =
\overline{A} \text{ , the complement of $\mathcal{A}$} \\
\mathcal{Q}^{E}\left( z^\prime,m^{\prime }\right) & =\left\{ z,m\Bigg|\omega_m(z,m)
-  \omega_m(m^{\prime},z^{\prime}) \geq 0\right\}
\end{align*}%
and the definition
\begin{eqnarray*}
\pmb{\omega}(z,m) = \max \Bigg\{ \max_{k\in [0,m]} \omega(k,z) + (m-k)%
\mathcal{U} \ , \ \theta + m\mathcal{U} \Bigg\}
\end{eqnarray*}
Note that now, the only place where $\pmb{\omega}$ enters in the Bellman
equation is the contribution of shocks. To replace it with $\omega$, we need
to apply the same argument to $z$ as the one we applied to $n$. We thus need
to specialize to a continuous productivity process.

\subsubsection{Continuous productivity process}

We now specialize to a continuous productivity process, as this makes the
formulation of the problem very economical. It allows to simplify the
contribution of productivity shocks and get rid of the remaining ``bold''
notation. We suppose that productivity follows a diffusion process:
\begin{equation*}
dz_t = \mu(z_t) dt + \sigma(z_t) dW_t
\end{equation*}
In this case, for any $(z,m)$ in the interior of the continuation set,
productivity shocks in the interval $[t,t+dt]$ cannot move the firm
towards a region in which it would endogenously separate or exit, when $dt$
is small enough. In this case, we can write the following, where we have also replaced the $\mathcal{Q}^E$ set with the max operator:
\begin{eqnarray*}
\rho \omega \left( z,m\right) =\max_{v \geq 0} && \mathcal{Y} \left(
z,m\right) -\mathcal{C}\left( v,z,m\right) \\
\text{Destructions} &&-\delta m [\omega_m(z,m) - \mathcal{U}] \\
\text{$UE$ Hires} &&+qv \phi \left[ \omega_m(z,m) -\mathcal{U}\right] \\
\text{$EE$ Hires} &&+q v \left( 1-\phi \right) \int \max\Bigg\{\omega_m(z,m)
- \omega_m(z^{\prime},m^{\prime}) \ , \ 0 \Bigg\}d\widetilde{H_n}\left(
m^{\prime},z^{\prime }\right) \\
\text{Shocks} &&+\mu(z) \omega_z(z,m) + \frac{\sigma(z)^2}{2}
\omega_{zz}(z,m) \\
&\text{s.t.}& \\
\text{No Exit}&& \omega(z,m) \geq \theta + m \mathcal{U} \\
\text{No Separations}&& \omega_m(z,m) \geq \mathcal{U}
\end{eqnarray*}
To make the notation more comparable, we slightly abuse notation and use the
same letters as before, but now for the continuous workforce case. We obtain
finally:
\begin{eqnarray*}
\rho \Omega \left(z,n\right) =\max_{v \geq 0} && y\left(z,n\right) -c\left(
v,z,n\right) \\
\text{Destructions} &&-\delta n [\Omega_n(z,n) - U] \\
\text{$UE$ Hires} &&+q v \phi \left[ \Omega_n(z,n) -U \right] \\
\text{$EE$ Hires} &&+q v \left( 1-\phi \right) \int \max\Bigg[\Omega_n(z,n)
- \Omega_n(z^{\prime},n^{\prime}) \ , \ 0 \Bigg]d\widetilde{H_n}\left(
z^{\prime},n^{\prime}\right) \\
\text{Shocks} &&+\mu(z) \Omega_z(z,n) + \frac{\sigma(z)^2}{2}
\Omega_{zz}(z,n) \\
&\text{s.t.}& \\
\text{No Exit}&& \Omega(z,n) \geq \vartheta + n U \\
\text{No Separations}&& \Omega_n(z,n) \geq U
\end{eqnarray*}
When the coalition hits $\Omega_n(z,n) = U$, it endogenous separates worker
to stay on that frontier. It exits when it hits the frontier $\Omega(z,n) =
\vartheta + n U$.

In addition to these ``value-pasting'' boundary conditions, optimality
implies necessary ``smooth-pasting'' boundary conditions (see Stokey 2009): $%
\Omega_z(z,n) = 0$ if the firm actually exits at $(z,n)$ following
productivity shocks, and $\Omega_n(z,n) = 0$ if the firm actually exits at $%
(z,n)$ following changes in size. These are necessary and sufficient for the
definition of our problem (Brekke and {\O}ksendal 1991). Its general formulation
terms of optimal switching between three regimes (operation, layoffs, exit)
on the entire positive quadrant, can be made as a system of
Hamilton-Jacobi-Bellman-Variational-Inequality (see Pham 2009), which we
present here for completeness :
\begin{eqnarray*}
&& \max \Bigg\{ - \rho \Omega \left(z,n\right) + \max_{v \geq 0} -\delta n
[\Omega_n(z,n) - U] +q v \phi \left[ \Omega_n(z,n) -U \right] \\
&& \hskip1.2cm +q v \left( 1-\phi \right) \int \max\Bigg[\Omega_n(z,n) -
\Omega_n(z^{\prime},n^{\prime}) \ , \ 0 \Bigg]d\widetilde{H_n}\left(
z^{\prime},n^{\prime}\right) +\mu(z) \Omega_z(z,n) + \frac{\sigma(z)^2}{2}
\Omega_{zz}(z,n) \ ; \\
&& \hskip1.3cm \vartheta + n U - \Omega(z,n) \ ; \ \max_{k \in [0,n]}
\Omega(z,k) + (n-k) U - \Omega(z,n) \Bigg\} = 0 \hskip5mm , \hskip5mm
\forall (z,n) \in \mathbb{R}_+^2
\end{eqnarray*}\vspace{.7cm}

\newpage
\section{Computational details}\label{appx:computation}

%%% LyX 2.2.3 created this file.  For more info, see http://www.lyx.org/.
%%% Do not edit unless you really know what you are doing.
%\documentclass[english]{article}
%\usepackage[OT1]{fontenc}
%\usepackage[latin9]{inputenc}
%\usepackage{geometry}
%\geometry{verbose,tmargin=1in,bmargin=1in,lmargin=1in,rmargin=1in}
%\usepackage{units}
%\usepackage{amsmath}
%\usepackage{babel}
%\begin{document}

\subsection{Numerical Solution to Surplus HJB Equation}

\subsubsection{Simplifying the Bellman equation}
\begin{itemize}
\item Recall that the stochastic process for $z_{t}$ is
\[
dz_{t}=\mu\left(z_{t}\right)dt+\sigma\left(z_{t}\right)dW_{t}
\]
\item We want to solve:
\[
\begin{aligned}\widetilde{\rho}S(z,n) & =\max_{v\geq0}\:y(z,n)-nb-c_{f}-\delta nS_{n}\left(z,n\right)\\
 & \hfill+\underbrace{q\left[\phi S_{n}\left(z,n\right)+(1-\phi)\int_{0}^{S_{n}\left(z,n\right)}\left[S_{n}\left(z,n\right)-S_{n}^{\prime}\right]dH_{n}\left(S_{n}^{\prime}\right)\right]}_{\text{Expected benefit per vacancy :=\ensuremath{\mathcal{H}\left(S_{n}\right)}}}v-c\left(v,n\right)\\
 & \hfill+\mu(z)S_{z}(z,n)+\frac{\sigma^{2}(z)}{2}S_{zz}(z,n)
\end{aligned}
\]
where $\widetilde{\rho}=\left(\rho+\delta_{x}\right)$ to account
for exogneous exit at rate $\delta_{x}$, which we use in the quantitative
model.
\item The expected benefit per vacancy, $\mathcal{H}\left(S_{n}\right)$
is a function only of marginal surplus $S_{n}$, and integrating by
parts is:
\[
\begin{aligned}\mathcal{H}\left(S_{n}\right) & =q\left[\phi S_{n}+\left(1-\phi\right)\widehat{H}_{n}\left(S_{n}\right)\right]\end{aligned}
\]
where
\[
\widehat{H}_{n}\left(S_{n}\right)=\int_{0}^{S_{n}}H_{n}(s)ds\quad,\quad H_{n}\left(S_{n}\right)=\int_{0}^{S_{n}}h_{n}\left(s\right)ds\quad,\quad h_{n}\left(S_{n}\right)=\frac{n\left(S_{n}\right)}{\texttt{n}}h\left(S_{n}\right)
\]
\item As in our quantitative exercise, let $c(v,n)=\frac{\bar{c}}{1+\gamma}\left(\frac{v}{n}\right)^{\gamma}v$.
\item The first order condition for vacancies is then as follows, with associated
vacancy rate:
\begin{eqnarray*}
\mathcal{H}\left(S_{n}\right) & = & \bar{c}\left(\frac{v}{n}\right)^{\gamma}\\
\frac{v\left(z,n\right)}{n} & = & \frac{1}{\overline{c}^{1/\gamma}}\mathcal{H}\left(S_{n}\left(z,n\right)\right)^{\frac{1}{\gamma}}.
\end{eqnarray*}
\item The terms that depend on vacancies in the Bellman equation can therefore
be simplified:
\begin{align*}
\mathcal{H}\left(S_{n}\left(z,n\right)\right)v-c(v,n) & =\xi\mathcal{H}\left(S_{n}\left(z,n\right)\right)^{\frac{1+\gamma}{\gamma}}n\:,\\
\xi & :=\frac{1}{\overline{c}^{1/\gamma}}\frac{\gamma}{1+\gamma}.
\end{align*}
\item Substituting this back into the Bellman equation, and re-arranging
we have flow payoffs, terms that depend on the drift of $n$ and terms
that depend on the dynamics of $z$:
\begin{equation}
\begin{aligned}\widetilde{\rho}S\left(z,n\right) & =y\left(z,n\right)-nb-c_{f}\\
 & \qquad+\left[\xi\frac{\mathcal{H}\left(S_{n}\left(z,n\right)\right)^{\frac{1+\gamma}{\gamma}}}{S_{n}\left(z,n\right)}-\delta\right]nS_{n}\left(z,n\right)\\
 & \qquad+\mu(z)S_{z}(z,n)+\frac{\sigma^{2}(z)}{2}S_{zz}(z,n).
\end{aligned}
\label{eq:Surplus}
\end{equation}
\item This is subject to boundary conditions and the previous definitions:
\[
\begin{aligned}S\left(z,n\right) & \geq0\\
S_{n}\left(z,n\right) & \geq0\\
\mathcal{H}\left(S_{n}\left(z,n\right)\right) & =q\left[\phi S_{n}\left(z,n\right)+(1-\phi)\widehat{H}_{n}\left(S_{n}\left(z,n\right)\right)\right]\\
\widehat{H}_{n}\left(S_{n}\left(z,n\right)\right) & =\int_{0}^{S_{n}(z,n)}H_{n}(s)ds
\end{aligned}
\]
\end{itemize}

\subsubsection{Change of variables}
\begin{itemize}
\item Define the following objects
\begin{eqnarray*}
\mu_{n}\left(z,n\right) & := & \xi\frac{\mathcal{H}\left(S_{n}\left(z,n\right)\right)^{\frac{1+\gamma}{\gamma}}}{S_{n}\left(z,n\right)}-\delta\\
\pi\left(z,n\right) & := & y\left(z,n\right)-nb-c_{f}
\end{eqnarray*}
\item Up to this point, the stochastic process for $z_{t}$ has been a general
geometric Brownian motion with drift and volatility $\mu\left(z_{t}\right)$and
$\sigma\left(z_{t}\right)$, respectively:
\[
dz_{t}=\mu\left(z_{t}\right)dt+\sigma\left(z_{t}\right)dW_{t}
\]
\item In the quantitative model, we consider a random walk in logs:
\begin{align*}
d\log z_{t} & =\mu dt+\sigma dW_{t}
\end{align*}
\item Ito's Lemma implies that
\begin{align*}
dz_{t} & =\underbrace{\left[\mu+\frac{\sigma^{2}}{2}\right]z_{t}}_{\mu\left(z_{t}\right)}dt+\underbrace{\sigma z_{t}}_{\sigma\left(z_{t}\right)}dW_{t}
\end{align*}
\item Substituting these into the Bellman equation
\begin{align*}
\widetilde{\rho}S\left(z,n\right) & =\pi\left(z,n\right)+\mu_{n}\left(z,n\right)nS_{n}\left(z,n\right)+\mu(z)S_{z}(z,n)+\frac{\sigma^{2}(z)}{2}S_{zz}(z,n)\\
\widetilde{\rho}S\left(z,n\right) & =\pi\left(z,n\right)+\mu_{n}\left(z,n\right)nS_{n}\left(z,n\right)+\left[\mu+\frac{\sigma^{2}}{2}\right]zS_{z}(z,n)+\frac{\sigma^{2}}{2}z^{2}S_{zz}(z,n)
\end{align*}
\item Now consider a change of variables. Let $\widetilde{z}=\log z$, $\widetilde{n}=\log n$.
Now define $\widetilde{S}\left(\widetilde{z},\widetilde{n}\right)=S\left(e^{\widetilde{z}},e^{\widetilde{n}}\right)=S\left(z,n\right)$,
$\widetilde{\pi}\left(\widetilde{z},\widetilde{n}\right)=\pi\left(e^{\widetilde{z}},e^{\widetilde{n}}\right)$
, and $\widetilde{\mu}_{n}\left(\widetilde{z},\widetilde{n}\right)=\mu_{n}\left(e^{\widetilde{z}},e^{\widetilde{n}}\right)$
\item Note that
\begin{align*}
\widetilde{\pi}\left(\widetilde{z},\widetilde{n}\right) & =y\left(e^{\widetilde{z}},e^{\widetilde{n}}\right)-e^{\widetilde{n}}b-c_{f}=\widetilde{y}\left(\widetilde{z},\widetilde{n}\right)-e^{\widetilde{n}}b-c_{f}\\
\widetilde{y}\left(\widetilde{z},\widetilde{n}\right) & =\left(e^{\widetilde{z}}\right)\times\left(e^{\alpha\widetilde{n}}\right)
\end{align*}
\item Applying the chain rule to $S\left(z,n\right)=\widetilde{S}\left(\log z,\log n\right)$,
and re-arranging:
\begin{align*}
S_{n}\left(z,n\right)n & =\widetilde{S}_{\widetilde{n}}\left(\widetilde{z},\widetilde{n}\right)\\
S_{z}\left(z,n\right)z & =\widetilde{S}_{\widetilde{z}}\left(\widetilde{z},\widetilde{n}\right)\\
z^{2}S_{zz}\left(z,n\right) & =\widetilde{S}_{\widetilde{z}\widetilde{z}}\left(\widetilde{z},\widetilde{n}\right)-\widetilde{S}_{\widetilde{z}}\left(\widetilde{z},\widetilde{n}\right)
\end{align*}
\item Substituting these into the Bellman equation
\begin{equation}
\widetilde{\rho}\widetilde{S}\left(\widetilde{z},\widetilde{n}\right)=\widetilde{\pi}\left(\widetilde{z},\widetilde{n}\right)+\mu_{n}\left(S_{n}\left(e^{\widetilde{z}},e^{\widetilde{n}}\right)\right)\widetilde{S}_{\widetilde{n}}\left(\widetilde{z},\widetilde{n}\right)+\mu\widetilde{S}_{\widetilde{z}}\left(\widetilde{z},\widetilde{n}\right)+\frac{\sigma^{2}}{2}\widetilde{S}_{\widetilde{z}\widetilde{z}}\left(\widetilde{z},\widetilde{n}\right)\label{eq:SurplusTilde}
\end{equation}
\item The boundary conditions are the same, since $S\left(z,n\right)=\widetilde{S}\left(\widetilde{z},\widetilde{n}\right)$,
and $S_{n}\left(z,n\right)\geq0$, which since $n\geq0$, is true
if and only if $S_{n}\left(z,n\right)n\geq0$, which is equivalent
to $\widetilde{S}_{\widetilde{n}}\left(\widetilde{z},\widetilde{n}\right)\geq0$.
\end{itemize}

\subsubsection{Implicit method}
\begin{itemize}
\item We solve (\ref{eq:SurplusTilde}) using an implicit method.
\item Let $\Delta$ denote step-size and $\tau$ the iteration of the algorithm.
\item Then given $\widetilde{S}^{\tau-1}\left(\widetilde{z},\widetilde{n}\right)$,
the implicit method gives an update:
\[
\frac{1}{\Delta}\left[\widetilde{S}^{\tau}\left(\widetilde{z},\widetilde{n}\right)-\widetilde{S}^{\tau-1}\left(\widetilde{z},\widetilde{n}\right)\right]+\widetilde{\rho}S^{\tau}\left(\widetilde{z},\widetilde{n}\right)=\widetilde{\pi}\left(\widetilde{z},\widetilde{n}\right)+\mu_{n}\left(S_{n}\left(e^{\widetilde{z}},e^{\widetilde{n}}\right)\right)\widetilde{S}_{\widetilde{n}}^{\tau}\left(\widetilde{z},\widetilde{n}\right)+\mu\widetilde{S}_{\widetilde{z}}^{\tau}\left(\widetilde{z},\widetilde{n}\right)+\frac{\sigma^{2}}{2}\widetilde{S}_{\widetilde{z}\widetilde{z}}^{\tau}\left(\widetilde{z},\widetilde{n}\right)
\]
\item Rearranging this expression:
\begin{equation}
\left(\frac{1}{\Delta}+\widetilde{\rho}\right)\widetilde{S}^{\tau}\left(\widetilde{z},\widetilde{n}\right)-\mu_{n}\left(S_{n}\left(e^{\widetilde{z}},e^{\widetilde{n}}\right)\right)\widetilde{S}_{\widetilde{n}}^{\tau}\left(\widetilde{z},\widetilde{n}\right)-\mu\widetilde{S}_{\widetilde{z}}^{\tau}\left(\widetilde{z},\widetilde{n}\right)-\frac{\sigma^{2}}{2}\widetilde{S}_{\widetilde{z}\widetilde{z}}^{\tau}\left(\widetilde{z},\widetilde{n}\right)=\widetilde{\pi}\left(\widetilde{z},\widetilde{n}\right)+\frac{1}{\Delta}\widetilde{S}^{\tau-1}\left(\widetilde{z},\widetilde{n}\right)\label{eq:implicit2}
\end{equation}
\item We now discretize $\widetilde{n}$ on an evenly spaced $N_{\widetilde{n}}\times1$
grid, $\widetilde{\boldsymbol{n}}=\left(\widetilde{n}_{0},\widetilde{n}_{0}+\Delta_{\widetilde{n}},\widetilde{n}_{0}+2\Delta_{\widetilde{n}},\widetilde{n}_{0}+\left(N_{\widetilde{n}}-1\right)\Delta_{\widetilde{n}}\right)$,
and $\widetilde{z}$ on an evenly spaced $N_{\widetilde{z}}\times1$
grid, $\widetilde{\boldsymbol{z}}=\left(\widetilde{z}_{0},\widetilde{z}_{0}+\Delta_{\widetilde{z}},\widetilde{z}_{0}+2\Delta_{\widetilde{z}},\widetilde{z}_{0}+\left(N_{\widetilde{z}}-1\right)\Delta_{\widetilde{z}}\right)$.
\item Stack these according to:
\[
\widetilde{\boldsymbol{s}}=\left[\boldsymbol{i}_{N_{\widetilde{n}}}\otimes\widetilde{\boldsymbol{z}},\widetilde{\boldsymbol{n}}\otimes\boldsymbol{i}_{N_{\widetilde{z}}}\right]=\left(\begin{array}{c}
\widetilde{z}_{1},\tilde{n}_{1}\\
\widetilde{z}_{2},\tilde{n}_{1}\\
\vdots\\
\widetilde{z}_{N_{\widetilde{z}}},\tilde{n}_{1}\\
\vdots\\
\widetilde{z}_{1},\tilde{n}_{N_{\widetilde{n}}}\\
\vdots\\
\widetilde{z}_{N_{\widetilde{z}}},\tilde{n}_{N_{\widetilde{n}}}
\end{array}\right).
\]
\item Discretized, we can write (\ref{eq:implicit2}) as $N_{\widetilde{z}}\times N_{\widetilde{n}}$
equations:
\begin{equation}
\left(\frac{1}{\Delta}+\widetilde{\rho}\right)\widetilde{S}^{\tau}-\mu_{n}\widetilde{S}_{\widetilde{n}}^{\tau}-\mu\widetilde{S}_{\widetilde{z}}^{\tau}-\frac{\sigma^{2}}{2}\widetilde{S}_{\widetilde{z}\widetilde{z}}^{\tau}=\widetilde{\pi}+\frac{1}{\Delta}\widetilde{S}^{\tau-1}\label{eq:surplusiteration5}
\end{equation}
\item Let $D_{\tilde{n}}$ be the $\left(N_{\widetilde{z}}\times N_{\widetilde{n}}\right)\times\left(N_{\widetilde{z}}\times N_{\widetilde{n}}\right)$
square matrix that, when pre-multiplying $\widetilde{S}^{\tau}$,
gives an approximation of $\widetilde{S}_{\widetilde{n}}^{\tau}$.
Analogously, define $D_{\widetilde{z}}$ and $D_{\widetilde{z}\widetilde{z}}$:
\begin{align*}
\widetilde{S}_{\widetilde{n}}^{\tau} & =D_{\tilde{n}}\widetilde{S}^{\tau}\\
\widetilde{S}_{\widetilde{z}}^{\tau} & =D_{\widetilde{z}}\widetilde{S}^{\tau}\\
\widetilde{S}_{\widetilde{z}\widetilde{z}}^{\tau} & =D_{\widetilde{z}\widetilde{z}}\widetilde{S}^{\tau}
\end{align*}
\item Using these we can write (\ref{eq:surplusiteration5}) as
\[
\left[\left(\frac{1}{\Delta}+\widetilde{\rho}\right)-\widetilde{\mu}_{n}D_{\tilde{n}}-\left(\mu D_{\widetilde{z}}+\frac{\sigma^{2}}{2}D_{\widetilde{z}\widetilde{z}}\right)\right]\widetilde{S}^{\tau}=\widetilde{\pi}+\frac{1}{\Delta}\widetilde{S}^{\tau-1}
\]
\item Define
\begin{align*}
\mathcal{N}^{v} & =\widetilde{\mu}_{n}D_{\tilde{n}}\\
\mathcal{Z}^{v} & =\mu D_{\widetilde{z}}+\frac{\sigma^{2}}{2}D_{\widetilde{z}\widetilde{z}}
\end{align*}
\item Then we have
\[
\left[\left(\frac{1}{\Delta}+\widetilde{\rho}\right)-\mathcal{N}^{v}-\mathcal{Z}^{v}\right]\widetilde{S}^{\tau}=\widetilde{\pi}+\frac{1}{\Delta}\widetilde{S}^{\tau-1}
\]
\item Therefore the implicit method works by updating $S^{\tau}$ through
\begin{align}
\mathbf{B}S^{\tau} & =\Pi+\frac{1}{\Delta}S^{\tau-1}\nonumber \\
\mathbf{B} & =\left(\frac{1}{\Delta}+\widetilde{\rho}\right)-\mathcal{N}^{v}-\mathcal{Z}^{v}\label{eq:surplusiteration3}
\end{align}
\end{itemize}

\subsubsection{Derivative matrices}
\begin{itemize}
\item To compute the derivative matrices $D_{\tilde{n}}$, $D_{\tilde{z}}$,
and $D_{\tilde{z}\tilde{z}}$, we follow an upwind scheme.
\item That is, we use a forward approximation when the drift of the state
variable is positive, and a backward approximation when the drift
of the state is negative.
\item In the simple case of $D_{\tilde{z}}$, since our estimation delivers
$\mu<0$, $D_{\tilde{z}}$ is built considering a backward difference
such that, for example, the derivative at point $S_{z}\left(z_{3},n_{2}\right)$
is computed as
\[
\widetilde{S}_{z}\left(\tilde{z}_{3},\widetilde{n}_{2}\right)=\frac{\widetilde{S}_{z}\left(\tilde{z}_{3},\widetilde{n}_{2}\right)-\widetilde{S}_{z}\left(\tilde{z}_{2},\widetilde{n}_{2}\right)}{\Delta_{\widetilde{z}}}.
\]
\item This requires that $D_{\tilde{z}}$ is as follows, where $\boldsymbol{I}_{N_{\widetilde{n}}}$
is an $N_{\widetilde{n}}\times N_{\widetilde{n}}$ identity matrix:
\[
D_{\tilde{z}}=\left(\begin{array}{cccccc}
\nicefrac{-1}{\Delta_{\widetilde{z}}} & \nicefrac{1}{\Delta_{\widetilde{z}}} & 0 & \cdots & \cdots & 0\\
\nicefrac{-1}{\Delta_{\widetilde{z}}} & \nicefrac{1}{\Delta_{\widetilde{z}}} & 0 & \ddots & \ddots & \vdots\\
0 & \nicefrac{-1}{\Delta_{\widetilde{z}}} & \nicefrac{1}{\Delta_{\widetilde{z}}} & 0 & \ddots & \vdots\\
\vdots & 0 & \nicefrac{-1}{\Delta_{\widetilde{z}}} & \nicefrac{1}{\Delta_{\widetilde{z}}} & \ddots & \vdots\\
\vdots & \ddots & \ddots & \ddots & \ddots & 0\\
0 & \cdots & \cdots & 0 & \nicefrac{-1}{\Delta_{\widetilde{z}}} & \nicefrac{1}{\Delta_{\widetilde{z}}}
\end{array}\right)_{N_{\widetilde{z}}\times N_{\widetilde{z}}}\otimes\boldsymbol{I}_{N_{\widetilde{n}}}
\]
which gives a backward approximation for any $i$ except for $i=1$
in which case it uses a forward approximation.
\item For the case of the second derivative with respect to $z$, we use
a central approximation by building the following matrix\footnote{For any $j$ the $\left(i,j\right)$ entry of $\widetilde{S}_{\widetilde{z}\widetilde{z}}=D_{\widetilde{z}}\widetilde{S}$
reads $\frac{\left[\widetilde{S}\left(\widetilde{z}_{i+1},\widetilde{n}_{j}\right)-\widetilde{S}\left(z_{i},\widetilde{n}_{j}\right)\right]-\left[\widetilde{S}\left(\widetilde{z}_{i},\widetilde{n}_{j}\right)-\widetilde{S}\left(\widetilde{z}_{i-1},\widetilde{n}_{j}\right)\right]}{\Delta_{\widetilde{z}}^{2}}$
for any $i$ except for $i=1$ in which case it reads $\frac{\widetilde{S}\left(\widetilde{z}_{2},\widetilde{n}_{j}\right)-\widetilde{S}\left(\widetilde{z}_{1},\widetilde{n}_{j}\right)}{\Delta_{\widetilde{z}}^{2}}$
and for $i=N_{\tilde{z}}$ in which case it reads $\frac{\widetilde{S}\left(\widetilde{z}_{N_{z}},\widetilde{n}_{j}\right)-\widetilde{S}\left(\widetilde{z}_{N_{\widetilde{z}}-1},\widetilde{n}_{j}\right)}{\Delta_{\widetilde{z}}^{2}}$.}
\[
D_{\tilde{z}\tilde{z}}=\left(\begin{array}{cccccc}
\nicefrac{-1}{\Delta_{\widetilde{z}}^{2}} & \nicefrac{1}{\Delta_{\widetilde{z}}^{2}} & 0 & \cdots & \cdots & 0\\
\nicefrac{1}{\Delta_{\tilde{z}}^{2}} & \nicefrac{-2}{\Delta_{\tilde{z}}^{2}} & \nicefrac{1}{\Delta_{\tilde{z}}^{2}} & 0 & \ddots & \vdots\\
0 & \nicefrac{1}{\Delta_{\tilde{z}}^{2}} & \nicefrac{-2}{\Delta_{\tilde{z}}^{2}} & \ddots & \ddots & \vdots\\
\vdots & 0 & \ddots & \ddots & \nicefrac{1}{\Delta_{\tilde{z}}^{2}} & 0\\
\vdots & \ddots & \ddots & \nicefrac{1}{\Delta_{\tilde{z}}^{2}} & \nicefrac{-2}{\Delta_{\tilde{z}}^{2}} & \nicefrac{1}{\Delta_{\tilde{z}}^{2}}\\
0 & \cdots & \cdots & 0 & \nicefrac{1}{\Delta_{\tilde{z}}^{2}} & \nicefrac{-1}{\Delta_{\tilde{z}}^{2}}
\end{array}\right)_{N_{\widetilde{z}}\times N_{\widetilde{z}}}\otimes\boldsymbol{I}_{N_{\widetilde{n}}}
\]
\item Finally, care must be taken in the case of $D_{\tilde{n}}$ since
the drift of $n$ endogenously depends on $\left(\widetilde{z},\widetilde{n}\right)$,
and thus the upwind scheme dictates that the direction of the finite
difference depends on $\left(\widetilde{z},\widetilde{n}\right)$.
We construct both $D_{\tilde{n}}^{f}$ that considers a forward approximation,
$D_{\tilde{n}}^{b}$ for a backward approximation, and use the former
for positive values of $\widetilde{\mu}_{n}\left(\widetilde{z},\widetilde{n}\right)$
and the latter for negative values,
\end{itemize}

\subsection*{F.2 Numerical Solution to Kolmogorov Forward Equation \label{subsec:F.2-KFE}}
\begin{itemize}
\item Substituting these into the Bellman equation
\begin{equation}
\widetilde{\rho}\widetilde{S}\left(\widetilde{z},\widetilde{n}\right)=\widetilde{\pi}\left(\widetilde{z},\widetilde{n}\right)+\widetilde{\mu}_{n}\left(\widetilde{z},\widetilde{n}\right)\widetilde{S}_{\widetilde{n}}\left(\widetilde{z},\widetilde{n}\right)+\mu\widetilde{S}_{\widetilde{z}}\left(\widetilde{z},\widetilde{n}\right)+\frac{\sigma^{2}}{2}\widetilde{S}_{\widetilde{z}\widetilde{z}}\left(\widetilde{z},\widetilde{n}\right)\label{eq:SurplusTilde-1}
\end{equation}
\item We continue to work in the transformed variables.
\item Note that
\[
\frac{dn/n}{dt}=\frac{d\log n}{dt}=\frac{d\widetilde{n}}{dt}
\]
\item Let $h\left(\widetilde{z},\widetilde{n}\right)$ be the stationary
distribution of firms in the economy.
\item Recall $\widetilde{S}\left(\widetilde{z},\widetilde{n}\right)=S\left(z,n\right)$,
so operating firms have $\widetilde{S}\left(\widetilde{z},\widetilde{n}\right)\geq0$.
\item Then $h\left(\widetilde{z},\widetilde{n}\right)$ satisfies
\begin{align*}
0 & =-\frac{\partial}{\partial\widetilde{n}}\left(\frac{d\widetilde{n}\left(\widetilde{z},\widetilde{n}\right)}{dt}\Bigg|_{\widetilde{S}\left(\widetilde{z},\widetilde{n}\right)\geq0}h\left(\widetilde{z},\widetilde{n}\right)\right)+\left(\frac{d\widetilde{n}\left(\widetilde{z},\widetilde{n}\right)}{dt}\Bigg|_{\widetilde{S}\left(\widetilde{z},\widetilde{n}\right)<0}h\left(\widetilde{z},\widetilde{n}\right)\right)\\
 & \quad\qquad-\mu h_{\widetilde{z}}\left(\widetilde{z},\widetilde{n}\right)+\frac{\sigma^{2}}{2}h_{\widetilde{z}\widetilde{z}}\left(\widetilde{z},\widetilde{n}\right)-\delta_{x}h\left(\widetilde{z},\widetilde{n}\right)+\texttt{m}_{0}\pi_{0}\left(\widetilde{z}\right)\Delta\left(\widetilde{n}\right)
\end{align*}
Note that firm exit from negative surplus is not treated as part of
the drift in $n$, hence two $\frac{d\widetilde{n}}{dt}$ terms: firm
exit takes mass away from certain states without returning it anywhere
in particular, while layoffs shift mass from one state to another.
\item We can vectorize this in the same way as above, and obtain
\begin{align*}
0 & =-D_{\widetilde{n}}\dot{\widetilde{n}}h-\mu D_{\tilde{z}}h+\frac{\sigma{}^{2}}{2}D_{\tilde{z}\tilde{z}}h-\mathcal{X}h-\delta_{x}h+h_{0}
\end{align*}
\item Where (i) $h$ is the stacked as above, (ii) $\dot{\widetilde{n}}$
is a $\left(N_{\widetilde{z}}\times N_{\widetilde{n}}\right)\times\left(N_{\widetilde{z}}\times N_{\widetilde{n}}\right)$
diagonal matrix with $\left.\frac{d\widetilde{n}\left(\widetilde{z},\widetilde{n}\right)}{dt}\right|_{\widetilde{S}\left(\widetilde{z},\widetilde{n}\right)\geq0}$
as its entries, (iii) $\mathcal{X}$ is a $\left(N_{\widetilde{z}}\times N_{\widetilde{n}}\right)\times\left(N_{\widetilde{z}}\times N_{\widetilde{n}}\right)$
diagonal matrix with $-\frac{d\widetilde{n}\left(\widetilde{z},\widetilde{n}\right)}{dt}\Bigg|_{\widetilde{S}\left(\widetilde{z},\widetilde{n}\right)<0}$
as its entries, and $h_{0}$ is the stacked $\left(N_{\widetilde{z}}\times N_{\widetilde{n}}\right)\times1$
vector version of $\texttt{m}_{0}\pi_{0}\left(z\right)\Delta\left(n\right)$.
\item The derivative matrices follow the same purpose as before, however
note that $D_{\tilde{z}}$ must use a \emph{forward difference} and
$D_{\widetilde{n}}$ must use upwinding to determine the approximation,
depending on the sign of $\dot{\widetilde{n}}$ (forward approximation
for negative drift and backward approximation for negative drift).
\item This expression can be rearranged to yield
\begin{equation}
\underbrace{\left(\underbrace{\left(-D_{\widetilde{n}}\dot{\widetilde{n}}\right)}_{\mathcal{N}^{d}}+\underbrace{\left(-\mu D_{\tilde{z}}+\frac{\sigma{}^{2}}{2}D_{zz}\right)}_{\mathcal{Z}^{d}}-\mathcal{X}-\delta_{x}\boldsymbol{I}_{N_{\widetilde{z}}\times N_{\widetilde{n}}}\right)}_{\mathbf{L}}h=-h_{0}.\label{eq:KFEnumerical}
\end{equation}
\end{itemize}

\subsection*{F.3 Stationary equilibrium algorithm}
\begin{itemize}
\item The algorithm consists of three steps, implemented in MATLAB by \texttt{SolveBEMV.m},
which is called by the master file \texttt{MAIN.m}.
\end{itemize}

\subsubsection*{Step 0 - Construct initial guess}
\begin{itemize}
\item Start by constructing a $N_{\widetilde{z}}\times N_{\widetilde{n}}$
grid for log productivity and log size.
\item Let $\pi\left(\widetilde{z},\widetilde{n}\right)=y\left(e^{\widetilde{z}},e^{\widetilde{n}}\right)-e^{\widetilde{n}}b-c_{f}$
denote the stacked $\left(N_{\widetilde{z}}\times N_{\widetilde{n}}\right)\times1$
vector of flow payoffs on this grid.
\item Guess an initial surplus $\widetilde{S}^{0}$ on this grid (a $\left(N_{\widetilde{z}}\times N_{\widetilde{n}}\right)\times1$
column vector); a distribution of firms over productivity and size
$h^{0}$ (a $\left(N_{\widetilde{z}}\times N_{\widetilde{n}}\right)\times1$
column vector); aggregate finding rates $q^{0}$ and $\lambda^{0};$
and an efficiency-weighted share of unemployed searchers, $\theta^{0}.$
\item Bundle together these aggregates $X^{0}=\left\{ q^{0},\phi^{0},h^{0},S^{0}\right\} $.
\item Construct marginal surplus. Construct exit regions, separation regions
and the vacancy policy. File \texttt{InitialGuess.m} does this.
\item Set $t=1$.
\end{itemize}

\subsubsection*{Step I - Given aggregate states, iterate to convergence on the coalition's
problem}
\begin{itemize}
\item Given $X^{t-1}=\left\{ q^{t-1},\phi^{t-1},h^{t-1},S^{t-1}\right\} $,
solve the coalition's problem to obtain $S^{t}.$
\item The equation we use is
\item We compute $\mathbf{B}$\textemdash which depends on the distribution
of marginal surplus, and other elements of $X^{t-1}$\textemdash using
$X^{t-1}$ and keep it fixed. Denote this $\mathbf{B}_{t-1}$.
\item Set $\tau=0$. Set $S^{t,\tau}=S^{t-1}$. Iterate using (\ref{eq:surplusiteration3}),
until convergence to $S^{t\ast}$, where $||S^{t,\tau}-S^{t,\tau-1}||<\varepsilon_{S}$.
\[
S^{t,\tau+1}=\mathbf{B}^{-1}\left(\Pi+\frac{1}{\Delta}S^{t,\tau}\right)
\]
These iterations are performed using \texttt{IndividualBehavior.m},
and the solution is assigned as the updated $S^{t}.$ At each step
we update $S_{n}^{t,\tau+1}$ using $S_{n}=D_{n}S$.
\item We also obtain from the converged solution the updated separation,
exit and vacancy policies.
\item Note that the step size $\Delta$ cannot be too large, otherwise the
problem will not converge.
\end{itemize}

\subsubsection*{Step II - Given individual behavior, iterate to convergence on aggregate
states}
\begin{itemize}
\item Given updated individual behavior in outer iteration $t$, obtain
through iteration in an inner loop $\tau$ the distribution of firms
$h^{t}$, the aggregate finding rates $q^{t}$ and $p^{t}$, the share
$\phi^{t}$, the distribution of workers over marginal surplus $H_{n}^{t}$,
the distribution of vacancies over marginal surplus $H_{v}^{t}$ and
the entry rate of firms $m_{0}^{t}$
\item File \texttt{AggregateBehavior.m} proceeds to do this in four steps.
\item Initiate each aggregate object with the previous outer iteration solution,
$x^{t-1,0}=x^{t-1}$. Then:
\item \textit{Step II-a. }Update the distribution of workers over marginal
surplus to $H_{n}^{t-1,\tau}$ and its integral $\hat{H}_{n}^{t-1,\tau}$
given a distribution of firms $h^{t-1,\tau-1}$ and marginal surplus
$S_{n^{\prime}}^{t}$ where the latter was obtained in \textbf{Step
I} above. This is done by file \texttt{Cdf\_Gn.m}.
\begin{itemize}
\item First consider the employment-weighted pdf:
\[
h_{n}^{t-1,\tau}\left(z,n\right)=\frac{n}{\texttt{n}}h^{t-1,\tau-1}\left(z,n\right)
\]
Where aggregate employment $\texttt{n}$ can be normalized so that
$h_{n}^{t-1,\tau}\left(z,n\right)$ integrates to 1.
\item We then sort $h_{n}$ by marginal surplus. Note that $S_{n}\left(z,n\right)n=\widetilde{S}_{\widetilde{n}}\left(\widetilde{z},\widetilde{n}\right)$,
and that we solved the Bellman equation in $\widetilde{S}_{\widetilde{n}}\left(\widetilde{z},\widetilde{n}\right)$,
therefore we have to obtain $S_{n}\left(z,n\right)$ by
\[
S_{n}\left(z,n\right)=\widetilde{S}_{\widetilde{n}}\left(\widetilde{z},\widetilde{n}\right)/n=\widetilde{S}_{\widetilde{n}}\left(\widetilde{z},\widetilde{n}\right)/e^{\widetilde{n}}.
\]
\item Then sort $h_{n}$ by marginal surplus and compute
\begin{eqnarray*}
H_{n}^{t-1,\tau}\left(S_{n}^{t}\left(z,n\right)\right) & = & \int_{0}^{S_{n}^{t}\left(z,n\right)}h_{n}^{t-1,\tau}\left(s\right)ds\\
\hat{H}_{n}^{t-1,\tau}\left(S_{n}^{t}(z,n)\right) & = & \int_{0}^{S_{n}^{t}(z,n)}H_{n}^{1-t,\tau}(s)ds
\end{eqnarray*}
\end{itemize}
\item \textit{Step II-b.} Update the distribution of vacancies over marginal
surplus $H_{v}^{t-1,\tau}$ given the vacancy policy $v^{t}$, $h^{t-1,\tau-1}$,
$H_{n}^{t-1,\tau},\ q^{t-1,\tau-1}$, $\phi^{t-1,\tau-1}$ and the
entry rate $m_{0}^{t-1,\tau}$. This is done by file \texttt{Cdf\_Gv.m}.
\begin{itemize}
\item First construct vacancies per worker for the entrants, which is the
necessary amount of vacancies you need to post to get $n_{0}$ workers:
\[
v_{e}^{t-1,\tau}\left(z,n\right)=\frac{n}{q^{t-1,\tau-1}\left[\phi^{t-1,\tau-1}+\left(1-\phi^{t-1,\tau-1}\right)H_{n}^{t-1,\tau}\left(S_{n}^{t}\left(z,n\right)\right)\right]}m_{0}^{t-1,\tau-1}\pi_{0}\left(z\right)\Delta\left(n\right)
\]
\item Now consider the distribution weighted by vacancies, both of incumbents
(using the policy function) and entrants:
\begin{align*}
h_{v}^{t-1,\tau}\left(z,n\right) & =\frac{v^{t}\left(z,n\right)h^{t-1,\tau-1}\left(z,n\right)+v_{e}^{t-1,\tau}\left(z,n\right)}{\texttt{v}}
\end{align*}
Where aggregate vacancies $\texttt{v}$ can be normalized so that
$h_{v}^{t-1,\tau}\left(z,n\right)$ integrates to 1.
\item Then simply, sort $h_{v}$ by marginal surplus and
\[
H_{v}^{t-1,\tau}\left(S_{n}^{t}\left(z,n\right)\right)=\int_{0}^{S_{n}^{t}\left(z,n\right)}h_{v}^{t-1,\tau}\left(s\right)ds
\]
\end{itemize}
\item \textit{Step II-c.} Update the distribution of firms $h^{t-1,\tau}$
and get the entry rate $m_{0}^{t-1,\tau}$ following the Kolmogorov
forward equation in steady-state given $H_{n}^{t-1,\tau},\ H_{v}^{t-1,\tau},\ q^{t-1,\tau-1},\ p^{t-1,\tau-1},\ \phi^{t-1,\tau-1}$
and $m_{0}^{t-1,\tau}$. This is executed by file \texttt{Distribution.m}.
\begin{itemize}
\item Using (\ref{eq:KFEnumerical}), this can be solved in one line.
\[
h^{t-1,\tau}=-\mathbf{L}^{-1}h_{0}
\]
\item To compute the entry rate, note that in the stationary equilibrium,
the entry rate must be equal to the exit rate of firms, which can
be obtained from $m_{0}^{t-1,\tau}=\mathbf{L}h^{t-1,\tau}$.
\end{itemize}
\item \textit{Step II-d.} Update the finding rates $q^{t-1,\tau},p^{t-1,\tau}$
and share $\phi^{t-1,\tau}$ that are consistent with the vacancy
policy $v^{t}$ and the distribution of firms $h^{t-1,\tau}$. This
is done by file \texttt{FindingRates.m}.
\begin{itemize}
\item First get the total units of search efficiency in the labor market
and total vacancies to construct:
\begin{eqnarray*}
\texttt{u}^{t-1,\tau} & = & \bar{\texttt{n}}-\texttt{m}\int\int ndH^{t-1,\tau}\left(z,n\right)\\
\texttt{s}^{t-1,\tau} & = & \texttt{u}^{t-1,\tau}+\xi\texttt{u}^{t-1,\tau}\\
\phi^{t-1,\tau} & = & \frac{\texttt{u}^{t-1,\tau}}{\texttt{s}^{t-1,\tau}}
\end{eqnarray*}
\item Then, get total vacancies and use the matching function:
\begin{eqnarray*}
\texttt{v}^{t-1,\tau} & = & \texttt{m}\int\int v^{t}\left(z,n\right)+v_{e}^{t-1,\tau}\left(z,n\right)dH^{t-1,\tau}\left(z,n\right)\\
\theta^{t-1,\tau} & = & \frac{\texttt{v}^{t-1,\tau}}{\texttt{s}^{t-1,\tau}}\\
q^{t-1,\tau} & = & A\left(\theta^{t-1,\tau}\right)^{\beta-1}\\
p^{t-1,\tau} & = & A\left(\theta^{t-1,\tau}\right)^{\beta}
\end{eqnarray*}
\end{itemize}
\item Iterate over the four sub-steps \textit{Step II-a} - \textit{Step
II-d} until convergence and assign the updated aggregate states $q^{t},\ p^{t},\ \phi^{t},\ h^{t},\ H_{n}^{t}$
and $H_{v}^{t}$.
\end{itemize}
\medskip{}

We subsequently iterate on \textbf{Step I - Step II} until both the
surplus function and the aggregate states have converged.
%\end{document}


\bigskip\bigskip

\noindent \textbf{\Large References}\bigskip

\noindent \textsc{Brekke K., and B. {\O}ksendal (1991):} ``The High Contact Principle as a Sufficiency Condition for Optimal Stopping,'' in  D. Lund, B. {\O}ksendal (Eds.), \emph{Stochastic Models and Option Values: Applications to Resources, Environment, and Investment Problems}. Contributions to Economic Analysis, Vol. 200. North-Holland, Amsterdam.\medskip

\noindent \textsc{Pham, H. (2009):} \emph{Continuous-time Stochastic Control and Optimization with Financial Applications.} Springer. \medskip

\noindent \textsc{Stokey, N. (2009):} \emph{The Economics of Inaction: Stochastic Control Models with Fixed Costs.} Princeton University Press.\medskip



\end{document}


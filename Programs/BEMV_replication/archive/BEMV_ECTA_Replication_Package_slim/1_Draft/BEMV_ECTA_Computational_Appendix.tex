%%% LyX 2.2.3 created this file.  For more info, see http://www.lyx.org/.
%%% Do not edit unless you really know what you are doing.
%\documentclass[english]{article}
%\usepackage[OT1]{fontenc}
%\usepackage[latin9]{inputenc}
%\usepackage{geometry}
%\geometry{verbose,tmargin=1in,bmargin=1in,lmargin=1in,rmargin=1in}
%\usepackage{units}
%\usepackage{amsmath}
%\usepackage{babel}
%\begin{document}

\subsection{Numerical Solution to Surplus HJB Equation}

\subsubsection{Simplifying the Bellman equation}
\begin{itemize}
\item Recall that the stochastic process for $z_{t}$ is
\[
dz_{t}=\mu\left(z_{t}\right)dt+\sigma\left(z_{t}\right)dW_{t}
\]
\item We want to solve:
\[
\begin{aligned}\widetilde{\rho}S(z,n) & =\max_{v\geq0}\:y(z,n)-nb-c_{f}-\delta nS_{n}\left(z,n\right)\\
 & \hfill+\underbrace{q\left[\phi S_{n}\left(z,n\right)+(1-\phi)\int_{0}^{S_{n}\left(z,n\right)}\left[S_{n}\left(z,n\right)-S_{n}^{\prime}\right]dH_{n}\left(S_{n}^{\prime}\right)\right]}_{\text{Expected benefit per vacancy :=\ensuremath{\mathcal{H}\left(S_{n}\right)}}}v-c\left(v,n\right)\\
 & \hfill+\mu(z)S_{z}(z,n)+\frac{\sigma^{2}(z)}{2}S_{zz}(z,n)
\end{aligned}
\]
where $\widetilde{\rho}=\left(\rho+\delta_{x}\right)$ to account
for exogneous exit at rate $\delta_{x}$, which we use in the quantitative
model.
\item The expected benefit per vacancy, $\mathcal{H}\left(S_{n}\right)$
is a function only of marginal surplus $S_{n}$, and integrating by
parts is:
\[
\begin{aligned}\mathcal{H}\left(S_{n}\right) & =q\left[\phi S_{n}+\left(1-\phi\right)\widehat{H}_{n}\left(S_{n}\right)\right]\end{aligned}
\]
where
\[
\widehat{H}_{n}\left(S_{n}\right)=\int_{0}^{S_{n}}H_{n}(s)ds\quad,\quad H_{n}\left(S_{n}\right)=\int_{0}^{S_{n}}h_{n}\left(s\right)ds\quad,\quad h_{n}\left(S_{n}\right)=\frac{n\left(S_{n}\right)}{\texttt{n}}h\left(S_{n}\right)
\]
\item As in our quantitative exercise, let $c(v,n)=\frac{\bar{c}}{1+\gamma}\left(\frac{v}{n}\right)^{\gamma}v$.
\item The first order condition for vacancies is then as follows, with associated
vacancy rate:
\begin{eqnarray*}
\mathcal{H}\left(S_{n}\right) & = & \bar{c}\left(\frac{v}{n}\right)^{\gamma}\\
\frac{v\left(z,n\right)}{n} & = & \frac{1}{\overline{c}^{1/\gamma}}\mathcal{H}\left(S_{n}\left(z,n\right)\right)^{\frac{1}{\gamma}}.
\end{eqnarray*}
\item The terms that depend on vacancies in the Bellman equation can therefore
be simplified:
\begin{align*}
\mathcal{H}\left(S_{n}\left(z,n\right)\right)v-c(v,n) & =\xi\mathcal{H}\left(S_{n}\left(z,n\right)\right)^{\frac{1+\gamma}{\gamma}}n\:,\\
\xi & :=\frac{1}{\overline{c}^{1/\gamma}}\frac{\gamma}{1+\gamma}.
\end{align*}
\item Substituting this back into the Bellman equation, and re-arranging
we have flow payoffs, terms that depend on the drift of $n$ and terms
that depend on the dynamics of $z$:
\begin{equation}
\begin{aligned}\widetilde{\rho}S\left(z,n\right) & =y\left(z,n\right)-nb-c_{f}\\
 & \qquad+\left[\xi\frac{\mathcal{H}\left(S_{n}\left(z,n\right)\right)^{\frac{1+\gamma}{\gamma}}}{S_{n}\left(z,n\right)}-\delta\right]nS_{n}\left(z,n\right)\\
 & \qquad+\mu(z)S_{z}(z,n)+\frac{\sigma^{2}(z)}{2}S_{zz}(z,n).
\end{aligned}
\label{eq:Surplus}
\end{equation}
\item This is subject to boundary conditions and the previous definitions:
\[
\begin{aligned}S\left(z,n\right) & \geq0\\
S_{n}\left(z,n\right) & \geq0\\
\mathcal{H}\left(S_{n}\left(z,n\right)\right) & =q\left[\phi S_{n}\left(z,n\right)+(1-\phi)\widehat{H}_{n}\left(S_{n}\left(z,n\right)\right)\right]\\
\widehat{H}_{n}\left(S_{n}\left(z,n\right)\right) & =\int_{0}^{S_{n}(z,n)}H_{n}(s)ds
\end{aligned}
\]
\end{itemize}

\subsubsection{Change of variables}
\begin{itemize}
\item Define the following objects
\begin{eqnarray*}
\mu_{n}\left(z,n\right) & := & \xi\frac{\mathcal{H}\left(S_{n}\left(z,n\right)\right)^{\frac{1+\gamma}{\gamma}}}{S_{n}\left(z,n\right)}-\delta\\
\pi\left(z,n\right) & := & y\left(z,n\right)-nb-c_{f}
\end{eqnarray*}
\item Up to this point, the stochastic process for $z_{t}$ has been a general
geometric Brownian motion with drift and volatility $\mu\left(z_{t}\right)$and
$\sigma\left(z_{t}\right)$, respectively:
\[
dz_{t}=\mu\left(z_{t}\right)dt+\sigma\left(z_{t}\right)dW_{t}
\]
\item In the quantitative model, we consider a random walk in logs:
\begin{align*}
d\log z_{t} & =\mu dt+\sigma dW_{t}
\end{align*}
\item Ito's Lemma implies that
\begin{align*}
dz_{t} & =\underbrace{\left[\mu+\frac{\sigma^{2}}{2}\right]z_{t}}_{\mu\left(z_{t}\right)}dt+\underbrace{\sigma z_{t}}_{\sigma\left(z_{t}\right)}dW_{t}
\end{align*}
\item Substituting these into the Bellman equation
\begin{align*}
\widetilde{\rho}S\left(z,n\right) & =\pi\left(z,n\right)+\mu_{n}\left(z,n\right)nS_{n}\left(z,n\right)+\mu(z)S_{z}(z,n)+\frac{\sigma^{2}(z)}{2}S_{zz}(z,n)\\
\widetilde{\rho}S\left(z,n\right) & =\pi\left(z,n\right)+\mu_{n}\left(z,n\right)nS_{n}\left(z,n\right)+\left[\mu+\frac{\sigma^{2}}{2}\right]zS_{z}(z,n)+\frac{\sigma^{2}}{2}z^{2}S_{zz}(z,n)
\end{align*}
\item Now consider a change of variables. Let $\widetilde{z}=\log z$, $\widetilde{n}=\log n$.
Now define $\widetilde{S}\left(\widetilde{z},\widetilde{n}\right)=S\left(e^{\widetilde{z}},e^{\widetilde{n}}\right)=S\left(z,n\right)$,
$\widetilde{\pi}\left(\widetilde{z},\widetilde{n}\right)=\pi\left(e^{\widetilde{z}},e^{\widetilde{n}}\right)$
, and $\widetilde{\mu}_{n}\left(\widetilde{z},\widetilde{n}\right)=\mu_{n}\left(e^{\widetilde{z}},e^{\widetilde{n}}\right)$
\item Note that
\begin{align*}
\widetilde{\pi}\left(\widetilde{z},\widetilde{n}\right) & =y\left(e^{\widetilde{z}},e^{\widetilde{n}}\right)-e^{\widetilde{n}}b-c_{f}=\widetilde{y}\left(\widetilde{z},\widetilde{n}\right)-e^{\widetilde{n}}b-c_{f}\\
\widetilde{y}\left(\widetilde{z},\widetilde{n}\right) & =\left(e^{\widetilde{z}}\right)\times\left(e^{\alpha\widetilde{n}}\right)
\end{align*}
\item Applying the chain rule to $S\left(z,n\right)=\widetilde{S}\left(\log z,\log n\right)$,
and re-arranging:
\begin{align*}
S_{n}\left(z,n\right)n & =\widetilde{S}_{\widetilde{n}}\left(\widetilde{z},\widetilde{n}\right)\\
S_{z}\left(z,n\right)z & =\widetilde{S}_{\widetilde{z}}\left(\widetilde{z},\widetilde{n}\right)\\
z^{2}S_{zz}\left(z,n\right) & =\widetilde{S}_{\widetilde{z}\widetilde{z}}\left(\widetilde{z},\widetilde{n}\right)-\widetilde{S}_{\widetilde{z}}\left(\widetilde{z},\widetilde{n}\right)
\end{align*}
\item Substituting these into the Bellman equation
\begin{equation}
\widetilde{\rho}\widetilde{S}\left(\widetilde{z},\widetilde{n}\right)=\widetilde{\pi}\left(\widetilde{z},\widetilde{n}\right)+\mu_{n}\left(S_{n}\left(e^{\widetilde{z}},e^{\widetilde{n}}\right)\right)\widetilde{S}_{\widetilde{n}}\left(\widetilde{z},\widetilde{n}\right)+\mu\widetilde{S}_{\widetilde{z}}\left(\widetilde{z},\widetilde{n}\right)+\frac{\sigma^{2}}{2}\widetilde{S}_{\widetilde{z}\widetilde{z}}\left(\widetilde{z},\widetilde{n}\right)\label{eq:SurplusTilde}
\end{equation}
\item The boundary conditions are the same, since $S\left(z,n\right)=\widetilde{S}\left(\widetilde{z},\widetilde{n}\right)$,
and $S_{n}\left(z,n\right)\geq0$, which since $n\geq0$, is true
if and only if $S_{n}\left(z,n\right)n\geq0$, which is equivalent
to $\widetilde{S}_{\widetilde{n}}\left(\widetilde{z},\widetilde{n}\right)\geq0$.
\end{itemize}

\subsubsection{Implicit method}
\begin{itemize}
\item We solve (\ref{eq:SurplusTilde}) using an implicit method.
\item Let $\Delta$ denote step-size and $\tau$ the iteration of the algorithm.
\item Then given $\widetilde{S}^{\tau-1}\left(\widetilde{z},\widetilde{n}\right)$,
the implicit method gives an update:
\[
\frac{1}{\Delta}\left[\widetilde{S}^{\tau}\left(\widetilde{z},\widetilde{n}\right)-\widetilde{S}^{\tau-1}\left(\widetilde{z},\widetilde{n}\right)\right]+\widetilde{\rho}S^{\tau}\left(\widetilde{z},\widetilde{n}\right)=\widetilde{\pi}\left(\widetilde{z},\widetilde{n}\right)+\mu_{n}\left(S_{n}\left(e^{\widetilde{z}},e^{\widetilde{n}}\right)\right)\widetilde{S}_{\widetilde{n}}^{\tau}\left(\widetilde{z},\widetilde{n}\right)+\mu\widetilde{S}_{\widetilde{z}}^{\tau}\left(\widetilde{z},\widetilde{n}\right)+\frac{\sigma^{2}}{2}\widetilde{S}_{\widetilde{z}\widetilde{z}}^{\tau}\left(\widetilde{z},\widetilde{n}\right)
\]
\item Rearranging this expression:
\begin{equation}
\left(\frac{1}{\Delta}+\widetilde{\rho}\right)\widetilde{S}^{\tau}\left(\widetilde{z},\widetilde{n}\right)-\mu_{n}\left(S_{n}\left(e^{\widetilde{z}},e^{\widetilde{n}}\right)\right)\widetilde{S}_{\widetilde{n}}^{\tau}\left(\widetilde{z},\widetilde{n}\right)-\mu\widetilde{S}_{\widetilde{z}}^{\tau}\left(\widetilde{z},\widetilde{n}\right)-\frac{\sigma^{2}}{2}\widetilde{S}_{\widetilde{z}\widetilde{z}}^{\tau}\left(\widetilde{z},\widetilde{n}\right)=\widetilde{\pi}\left(\widetilde{z},\widetilde{n}\right)+\frac{1}{\Delta}\widetilde{S}^{\tau-1}\left(\widetilde{z},\widetilde{n}\right)\label{eq:implicit2}
\end{equation}
\item We now discretize $\widetilde{n}$ on an evenly spaced $N_{\widetilde{n}}\times1$
grid, $\widetilde{\boldsymbol{n}}=\left(\widetilde{n}_{0},\widetilde{n}_{0}+\Delta_{\widetilde{n}},\widetilde{n}_{0}+2\Delta_{\widetilde{n}},\widetilde{n}_{0}+\left(N_{\widetilde{n}}-1\right)\Delta_{\widetilde{n}}\right)$,
and $\widetilde{z}$ on an evenly spaced $N_{\widetilde{z}}\times1$
grid, $\widetilde{\boldsymbol{z}}=\left(\widetilde{z}_{0},\widetilde{z}_{0}+\Delta_{\widetilde{z}},\widetilde{z}_{0}+2\Delta_{\widetilde{z}},\widetilde{z}_{0}+\left(N_{\widetilde{z}}-1\right)\Delta_{\widetilde{z}}\right)$.
\item Stack these according to:
\[
\widetilde{\boldsymbol{s}}=\left[\boldsymbol{i}_{N_{\widetilde{n}}}\otimes\widetilde{\boldsymbol{z}},\widetilde{\boldsymbol{n}}\otimes\boldsymbol{i}_{N_{\widetilde{z}}}\right]=\left(\begin{array}{c}
\widetilde{z}_{1},\tilde{n}_{1}\\
\widetilde{z}_{2},\tilde{n}_{1}\\
\vdots\\
\widetilde{z}_{N_{\widetilde{z}}},\tilde{n}_{1}\\
\vdots\\
\widetilde{z}_{1},\tilde{n}_{N_{\widetilde{n}}}\\
\vdots\\
\widetilde{z}_{N_{\widetilde{z}}},\tilde{n}_{N_{\widetilde{n}}}
\end{array}\right).
\]
\item Discretized, we can write (\ref{eq:implicit2}) as $N_{\widetilde{z}}\times N_{\widetilde{n}}$
equations:
\begin{equation}
\left(\frac{1}{\Delta}+\widetilde{\rho}\right)\widetilde{S}^{\tau}-\mu_{n}\widetilde{S}_{\widetilde{n}}^{\tau}-\mu\widetilde{S}_{\widetilde{z}}^{\tau}-\frac{\sigma^{2}}{2}\widetilde{S}_{\widetilde{z}\widetilde{z}}^{\tau}=\widetilde{\pi}+\frac{1}{\Delta}\widetilde{S}^{\tau-1}\label{eq:surplusiteration5}
\end{equation}
\item Let $D_{\tilde{n}}$ be the $\left(N_{\widetilde{z}}\times N_{\widetilde{n}}\right)\times\left(N_{\widetilde{z}}\times N_{\widetilde{n}}\right)$
square matrix that, when pre-multiplying $\widetilde{S}^{\tau}$,
gives an approximation of $\widetilde{S}_{\widetilde{n}}^{\tau}$.
Analogously, define $D_{\widetilde{z}}$ and $D_{\widetilde{z}\widetilde{z}}$:
\begin{align*}
\widetilde{S}_{\widetilde{n}}^{\tau} & =D_{\tilde{n}}\widetilde{S}^{\tau}\\
\widetilde{S}_{\widetilde{z}}^{\tau} & =D_{\widetilde{z}}\widetilde{S}^{\tau}\\
\widetilde{S}_{\widetilde{z}\widetilde{z}}^{\tau} & =D_{\widetilde{z}\widetilde{z}}\widetilde{S}^{\tau}
\end{align*}
\item Using these we can write (\ref{eq:surplusiteration5}) as
\[
\left[\left(\frac{1}{\Delta}+\widetilde{\rho}\right)-\widetilde{\mu}_{n}D_{\tilde{n}}-\left(\mu D_{\widetilde{z}}+\frac{\sigma^{2}}{2}D_{\widetilde{z}\widetilde{z}}\right)\right]\widetilde{S}^{\tau}=\widetilde{\pi}+\frac{1}{\Delta}\widetilde{S}^{\tau-1}
\]
\item Define
\begin{align*}
\mathcal{N}^{v} & =\widetilde{\mu}_{n}D_{\tilde{n}}\\
\mathcal{Z}^{v} & =\mu D_{\widetilde{z}}+\frac{\sigma^{2}}{2}D_{\widetilde{z}\widetilde{z}}
\end{align*}
\item Then we have
\[
\left[\left(\frac{1}{\Delta}+\widetilde{\rho}\right)-\mathcal{N}^{v}-\mathcal{Z}^{v}\right]\widetilde{S}^{\tau}=\widetilde{\pi}+\frac{1}{\Delta}\widetilde{S}^{\tau-1}
\]
\item Therefore the implicit method works by updating $S^{\tau}$ through
\begin{align}
\mathbf{B}S^{\tau} & =\Pi+\frac{1}{\Delta}S^{\tau-1}\nonumber \\
\mathbf{B} & =\left(\frac{1}{\Delta}+\widetilde{\rho}\right)-\mathcal{N}^{v}-\mathcal{Z}^{v}\label{eq:surplusiteration3}
\end{align}
\end{itemize}

\subsubsection{Derivative matrices}
\begin{itemize}
\item To compute the derivative matrices $D_{\tilde{n}}$, $D_{\tilde{z}}$,
and $D_{\tilde{z}\tilde{z}}$, we follow an upwind scheme.
\item That is, we use a forward approximation when the drift of the state
variable is positive, and a backward approximation when the drift
of the state is negative.
\item In the simple case of $D_{\tilde{z}}$, since our estimation delivers
$\mu<0$, $D_{\tilde{z}}$ is built considering a backward difference
such that, for example, the derivative at point $S_{z}\left(z_{3},n_{2}\right)$
is computed as
\[
\widetilde{S}_{z}\left(\tilde{z}_{3},\widetilde{n}_{2}\right)=\frac{\widetilde{S}_{z}\left(\tilde{z}_{3},\widetilde{n}_{2}\right)-\widetilde{S}_{z}\left(\tilde{z}_{2},\widetilde{n}_{2}\right)}{\Delta_{\widetilde{z}}}.
\]
\item This requires that $D_{\tilde{z}}$ is as follows, where $\boldsymbol{I}_{N_{\widetilde{n}}}$
is an $N_{\widetilde{n}}\times N_{\widetilde{n}}$ identity matrix:
\[
D_{\tilde{z}}=\left(\begin{array}{cccccc}
\nicefrac{-1}{\Delta_{\widetilde{z}}} & \nicefrac{1}{\Delta_{\widetilde{z}}} & 0 & \cdots & \cdots & 0\\
\nicefrac{-1}{\Delta_{\widetilde{z}}} & \nicefrac{1}{\Delta_{\widetilde{z}}} & 0 & \ddots & \ddots & \vdots\\
0 & \nicefrac{-1}{\Delta_{\widetilde{z}}} & \nicefrac{1}{\Delta_{\widetilde{z}}} & 0 & \ddots & \vdots\\
\vdots & 0 & \nicefrac{-1}{\Delta_{\widetilde{z}}} & \nicefrac{1}{\Delta_{\widetilde{z}}} & \ddots & \vdots\\
\vdots & \ddots & \ddots & \ddots & \ddots & 0\\
0 & \cdots & \cdots & 0 & \nicefrac{-1}{\Delta_{\widetilde{z}}} & \nicefrac{1}{\Delta_{\widetilde{z}}}
\end{array}\right)_{N_{\widetilde{z}}\times N_{\widetilde{z}}}\otimes\boldsymbol{I}_{N_{\widetilde{n}}}
\]
which gives a backward approximation for any $i$ except for $i=1$
in which case it uses a forward approximation.
\item For the case of the second derivative with respect to $z$, we use
a central approximation by building the following matrix\footnote{For any $j$ the $\left(i,j\right)$ entry of $\widetilde{S}_{\widetilde{z}\widetilde{z}}=D_{\widetilde{z}}\widetilde{S}$
reads $\frac{\left[\widetilde{S}\left(\widetilde{z}_{i+1},\widetilde{n}_{j}\right)-\widetilde{S}\left(z_{i},\widetilde{n}_{j}\right)\right]-\left[\widetilde{S}\left(\widetilde{z}_{i},\widetilde{n}_{j}\right)-\widetilde{S}\left(\widetilde{z}_{i-1},\widetilde{n}_{j}\right)\right]}{\Delta_{\widetilde{z}}^{2}}$
for any $i$ except for $i=1$ in which case it reads $\frac{\widetilde{S}\left(\widetilde{z}_{2},\widetilde{n}_{j}\right)-\widetilde{S}\left(\widetilde{z}_{1},\widetilde{n}_{j}\right)}{\Delta_{\widetilde{z}}^{2}}$
and for $i=N_{\tilde{z}}$ in which case it reads $\frac{\widetilde{S}\left(\widetilde{z}_{N_{z}},\widetilde{n}_{j}\right)-\widetilde{S}\left(\widetilde{z}_{N_{\widetilde{z}}-1},\widetilde{n}_{j}\right)}{\Delta_{\widetilde{z}}^{2}}$.}
\[
D_{\tilde{z}\tilde{z}}=\left(\begin{array}{cccccc}
\nicefrac{-1}{\Delta_{\widetilde{z}}^{2}} & \nicefrac{1}{\Delta_{\widetilde{z}}^{2}} & 0 & \cdots & \cdots & 0\\
\nicefrac{1}{\Delta_{\tilde{z}}^{2}} & \nicefrac{-2}{\Delta_{\tilde{z}}^{2}} & \nicefrac{1}{\Delta_{\tilde{z}}^{2}} & 0 & \ddots & \vdots\\
0 & \nicefrac{1}{\Delta_{\tilde{z}}^{2}} & \nicefrac{-2}{\Delta_{\tilde{z}}^{2}} & \ddots & \ddots & \vdots\\
\vdots & 0 & \ddots & \ddots & \nicefrac{1}{\Delta_{\tilde{z}}^{2}} & 0\\
\vdots & \ddots & \ddots & \nicefrac{1}{\Delta_{\tilde{z}}^{2}} & \nicefrac{-2}{\Delta_{\tilde{z}}^{2}} & \nicefrac{1}{\Delta_{\tilde{z}}^{2}}\\
0 & \cdots & \cdots & 0 & \nicefrac{1}{\Delta_{\tilde{z}}^{2}} & \nicefrac{-1}{\Delta_{\tilde{z}}^{2}}
\end{array}\right)_{N_{\widetilde{z}}\times N_{\widetilde{z}}}\otimes\boldsymbol{I}_{N_{\widetilde{n}}}
\]
\item Finally, care must be taken in the case of $D_{\tilde{n}}$ since
the drift of $n$ endogenously depends on $\left(\widetilde{z},\widetilde{n}\right)$,
and thus the upwind scheme dictates that the direction of the finite
difference depends on $\left(\widetilde{z},\widetilde{n}\right)$.
We construct both $D_{\tilde{n}}^{f}$ that considers a forward approximation,
$D_{\tilde{n}}^{b}$ for a backward approximation, and use the former
for positive values of $\widetilde{\mu}_{n}\left(\widetilde{z},\widetilde{n}\right)$
and the latter for negative values,
\end{itemize}

\subsection*{F.2 Numerical Solution to Kolmogorov Forward Equation \label{subsec:F.2-KFE}}
\begin{itemize}
\item Substituting these into the Bellman equation
\begin{equation}
\widetilde{\rho}\widetilde{S}\left(\widetilde{z},\widetilde{n}\right)=\widetilde{\pi}\left(\widetilde{z},\widetilde{n}\right)+\widetilde{\mu}_{n}\left(\widetilde{z},\widetilde{n}\right)\widetilde{S}_{\widetilde{n}}\left(\widetilde{z},\widetilde{n}\right)+\mu\widetilde{S}_{\widetilde{z}}\left(\widetilde{z},\widetilde{n}\right)+\frac{\sigma^{2}}{2}\widetilde{S}_{\widetilde{z}\widetilde{z}}\left(\widetilde{z},\widetilde{n}\right)\label{eq:SurplusTilde-1}
\end{equation}
\item We continue to work in the transformed variables.
\item Note that
\[
\frac{dn/n}{dt}=\frac{d\log n}{dt}=\frac{d\widetilde{n}}{dt}
\]
\item Let $h\left(\widetilde{z},\widetilde{n}\right)$ be the stationary
distribution of firms in the economy.
\item Recall $\widetilde{S}\left(\widetilde{z},\widetilde{n}\right)=S\left(z,n\right)$,
so operating firms have $\widetilde{S}\left(\widetilde{z},\widetilde{n}\right)\geq0$.
\item Then $h\left(\widetilde{z},\widetilde{n}\right)$ satisfies
\begin{align*}
0 & =-\frac{\partial}{\partial\widetilde{n}}\left(\frac{d\widetilde{n}\left(\widetilde{z},\widetilde{n}\right)}{dt}\Bigg|_{\widetilde{S}\left(\widetilde{z},\widetilde{n}\right)\geq0}h\left(\widetilde{z},\widetilde{n}\right)\right)+\left(\frac{d\widetilde{n}\left(\widetilde{z},\widetilde{n}\right)}{dt}\Bigg|_{\widetilde{S}\left(\widetilde{z},\widetilde{n}\right)<0}h\left(\widetilde{z},\widetilde{n}\right)\right)\\
 & \quad\qquad-\mu h_{\widetilde{z}}\left(\widetilde{z},\widetilde{n}\right)+\frac{\sigma^{2}}{2}h_{\widetilde{z}\widetilde{z}}\left(\widetilde{z},\widetilde{n}\right)-\delta_{x}h\left(\widetilde{z},\widetilde{n}\right)+\texttt{m}_{0}\pi_{0}\left(\widetilde{z}\right)\Delta\left(\widetilde{n}\right)
\end{align*}
Note that firm exit from negative surplus is not treated as part of
the drift in $n$, hence two $\frac{d\widetilde{n}}{dt}$ terms: firm
exit takes mass away from certain states without returning it anywhere
in particular, while layoffs shift mass from one state to another.
\item We can vectorize this in the same way as above, and obtain
\begin{align*}
0 & =-D_{\widetilde{n}}\dot{\widetilde{n}}h-\mu D_{\tilde{z}}h+\frac{\sigma{}^{2}}{2}D_{\tilde{z}\tilde{z}}h-\mathcal{X}h-\delta_{x}h+h_{0}
\end{align*}
\item Where (i) $h$ is the stacked as above, (ii) $\dot{\widetilde{n}}$
is a $\left(N_{\widetilde{z}}\times N_{\widetilde{n}}\right)\times\left(N_{\widetilde{z}}\times N_{\widetilde{n}}\right)$
diagonal matrix with $\left.\frac{d\widetilde{n}\left(\widetilde{z},\widetilde{n}\right)}{dt}\right|_{\widetilde{S}\left(\widetilde{z},\widetilde{n}\right)\geq0}$
as its entries, (iii) $\mathcal{X}$ is a $\left(N_{\widetilde{z}}\times N_{\widetilde{n}}\right)\times\left(N_{\widetilde{z}}\times N_{\widetilde{n}}\right)$
diagonal matrix with $-\frac{d\widetilde{n}\left(\widetilde{z},\widetilde{n}\right)}{dt}\Bigg|_{\widetilde{S}\left(\widetilde{z},\widetilde{n}\right)<0}$
as its entries, and $h_{0}$ is the stacked $\left(N_{\widetilde{z}}\times N_{\widetilde{n}}\right)\times1$
vector version of $\texttt{m}_{0}\pi_{0}\left(z\right)\Delta\left(n\right)$.
\item The derivative matrices follow the same purpose as before, however
note that $D_{\tilde{z}}$ must use a \emph{forward difference} and
$D_{\widetilde{n}}$ must use upwinding to determine the approximation,
depending on the sign of $\dot{\widetilde{n}}$ (forward approximation
for negative drift and backward approximation for negative drift).
\item This expression can be rearranged to yield
\begin{equation}
\underbrace{\left(\underbrace{\left(-D_{\widetilde{n}}\dot{\widetilde{n}}\right)}_{\mathcal{N}^{d}}+\underbrace{\left(-\mu D_{\tilde{z}}+\frac{\sigma{}^{2}}{2}D_{zz}\right)}_{\mathcal{Z}^{d}}-\mathcal{X}-\delta_{x}\boldsymbol{I}_{N_{\widetilde{z}}\times N_{\widetilde{n}}}\right)}_{\mathbf{L}}h=-h_{0}.\label{eq:KFEnumerical}
\end{equation}
\end{itemize}

\subsection*{F.3 Stationary equilibrium algorithm}
\begin{itemize}
\item The algorithm consists of three steps, implemented in MATLAB by \texttt{SolveBEMV.m},
which is called by the master file \texttt{MAIN.m}.
\end{itemize}

\subsubsection*{Step 0 - Construct initial guess}
\begin{itemize}
\item Start by constructing a $N_{\widetilde{z}}\times N_{\widetilde{n}}$
grid for log productivity and log size.
\item Let $\pi\left(\widetilde{z},\widetilde{n}\right)=y\left(e^{\widetilde{z}},e^{\widetilde{n}}\right)-e^{\widetilde{n}}b-c_{f}$
denote the stacked $\left(N_{\widetilde{z}}\times N_{\widetilde{n}}\right)\times1$
vector of flow payoffs on this grid.
\item Guess an initial surplus $\widetilde{S}^{0}$ on this grid (a $\left(N_{\widetilde{z}}\times N_{\widetilde{n}}\right)\times1$
column vector); a distribution of firms over productivity and size
$h^{0}$ (a $\left(N_{\widetilde{z}}\times N_{\widetilde{n}}\right)\times1$
column vector); aggregate finding rates $q^{0}$ and $\lambda^{0};$
and an efficiency-weighted share of unemployed searchers, $\theta^{0}.$
\item Bundle together these aggregates $X^{0}=\left\{ q^{0},\phi^{0},h^{0},S^{0}\right\} $.
\item Construct marginal surplus. Construct exit regions, separation regions
and the vacancy policy. File \texttt{InitialGuess.m} does this.
\item Set $t=1$.
\end{itemize}

\subsubsection*{Step I - Given aggregate states, iterate to convergence on the coalition's
problem}
\begin{itemize}
\item Given $X^{t-1}=\left\{ q^{t-1},\phi^{t-1},h^{t-1},S^{t-1}\right\} $,
solve the coalition's problem to obtain $S^{t}.$
\item The equation we use is
\item We compute $\mathbf{B}$\textemdash which depends on the distribution
of marginal surplus, and other elements of $X^{t-1}$\textemdash using
$X^{t-1}$ and keep it fixed. Denote this $\mathbf{B}_{t-1}$.
\item Set $\tau=0$. Set $S^{t,\tau}=S^{t-1}$. Iterate using (\ref{eq:surplusiteration3}),
until convergence to $S^{t\ast}$, where $||S^{t,\tau}-S^{t,\tau-1}||<\varepsilon_{S}$.
\[
S^{t,\tau+1}=\mathbf{B}^{-1}\left(\Pi+\frac{1}{\Delta}S^{t,\tau}\right)
\]
These iterations are performed using \texttt{IndividualBehavior.m},
and the solution is assigned as the updated $S^{t}.$ At each step
we update $S_{n}^{t,\tau+1}$ using $S_{n}=D_{n}S$.
\item We also obtain from the converged solution the updated separation,
exit and vacancy policies.
\item Note that the step size $\Delta$ cannot be too large, otherwise the
problem will not converge.
\end{itemize}

\subsubsection*{Step II - Given individual behavior, iterate to convergence on aggregate
states}
\begin{itemize}
\item Given updated individual behavior in outer iteration $t$, obtain
through iteration in an inner loop $\tau$ the distribution of firms
$h^{t}$, the aggregate finding rates $q^{t}$ and $p^{t}$, the share
$\phi^{t}$, the distribution of workers over marginal surplus $H_{n}^{t}$,
the distribution of vacancies over marginal surplus $H_{v}^{t}$ and
the entry rate of firms $m_{0}^{t}$
\item File \texttt{AggregateBehavior.m} proceeds to do this in four steps.
\item Initiate each aggregate object with the previous outer iteration solution,
$x^{t-1,0}=x^{t-1}$. Then:
\item \textit{Step II-a. }Update the distribution of workers over marginal
surplus to $H_{n}^{t-1,\tau}$ and its integral $\hat{H}_{n}^{t-1,\tau}$
given a distribution of firms $h^{t-1,\tau-1}$ and marginal surplus
$S_{n^{\prime}}^{t}$ where the latter was obtained in \textbf{Step
I} above. This is done by file \texttt{Cdf\_Gn.m}.
\begin{itemize}
\item First consider the employment-weighted pdf:
\[
h_{n}^{t-1,\tau}\left(z,n\right)=\frac{n}{\texttt{n}}h^{t-1,\tau-1}\left(z,n\right)
\]
Where aggregate employment $\texttt{n}$ can be normalized so that
$h_{n}^{t-1,\tau}\left(z,n\right)$ integrates to 1.
\item We then sort $h_{n}$ by marginal surplus. Note that $S_{n}\left(z,n\right)n=\widetilde{S}_{\widetilde{n}}\left(\widetilde{z},\widetilde{n}\right)$,
and that we solved the Bellman equation in $\widetilde{S}_{\widetilde{n}}\left(\widetilde{z},\widetilde{n}\right)$,
therefore we have to obtain $S_{n}\left(z,n\right)$ by
\[
S_{n}\left(z,n\right)=\widetilde{S}_{\widetilde{n}}\left(\widetilde{z},\widetilde{n}\right)/n=\widetilde{S}_{\widetilde{n}}\left(\widetilde{z},\widetilde{n}\right)/e^{\widetilde{n}}.
\]
\item Then sort $h_{n}$ by marginal surplus and compute
\begin{eqnarray*}
H_{n}^{t-1,\tau}\left(S_{n}^{t}\left(z,n\right)\right) & = & \int_{0}^{S_{n}^{t}\left(z,n\right)}h_{n}^{t-1,\tau}\left(s\right)ds\\
\hat{H}_{n}^{t-1,\tau}\left(S_{n}^{t}(z,n)\right) & = & \int_{0}^{S_{n}^{t}(z,n)}H_{n}^{1-t,\tau}(s)ds
\end{eqnarray*}
\end{itemize}
\item \textit{Step II-b.} Update the distribution of vacancies over marginal
surplus $H_{v}^{t-1,\tau}$ given the vacancy policy $v^{t}$, $h^{t-1,\tau-1}$,
$H_{n}^{t-1,\tau},\ q^{t-1,\tau-1}$, $\phi^{t-1,\tau-1}$ and the
entry rate $m_{0}^{t-1,\tau}$. This is done by file \texttt{Cdf\_Gv.m}.
\begin{itemize}
\item First construct vacancies per worker for the entrants, which is the
necessary amount of vacancies you need to post to get $n_{0}$ workers:
\[
v_{e}^{t-1,\tau}\left(z,n\right)=\frac{n}{q^{t-1,\tau-1}\left[\phi^{t-1,\tau-1}+\left(1-\phi^{t-1,\tau-1}\right)H_{n}^{t-1,\tau}\left(S_{n}^{t}\left(z,n\right)\right)\right]}m_{0}^{t-1,\tau-1}\pi_{0}\left(z\right)\Delta\left(n\right)
\]
\item Now consider the distribution weighted by vacancies, both of incumbents
(using the policy function) and entrants:
\begin{align*}
h_{v}^{t-1,\tau}\left(z,n\right) & =\frac{v^{t}\left(z,n\right)h^{t-1,\tau-1}\left(z,n\right)+v_{e}^{t-1,\tau}\left(z,n\right)}{\texttt{v}}
\end{align*}
Where aggregate vacancies $\texttt{v}$ can be normalized so that
$h_{v}^{t-1,\tau}\left(z,n\right)$ integrates to 1.
\item Then simply, sort $h_{v}$ by marginal surplus and
\[
H_{v}^{t-1,\tau}\left(S_{n}^{t}\left(z,n\right)\right)=\int_{0}^{S_{n}^{t}\left(z,n\right)}h_{v}^{t-1,\tau}\left(s\right)ds
\]
\end{itemize}
\item \textit{Step II-c.} Update the distribution of firms $h^{t-1,\tau}$
and get the entry rate $m_{0}^{t-1,\tau}$ following the Kolmogorov
forward equation in steady-state given $H_{n}^{t-1,\tau},\ H_{v}^{t-1,\tau},\ q^{t-1,\tau-1},\ p^{t-1,\tau-1},\ \phi^{t-1,\tau-1}$
and $m_{0}^{t-1,\tau}$. This is executed by file \texttt{Distribution.m}.
\begin{itemize}
\item Using (\ref{eq:KFEnumerical}), this can be solved in one line.
\[
h^{t-1,\tau}=-\mathbf{L}^{-1}h_{0}
\]
\item To compute the entry rate, note that in the stationary equilibrium,
the entry rate must be equal to the exit rate of firms, which can
be obtained from $m_{0}^{t-1,\tau}=\mathbf{L}h^{t-1,\tau}$.
\end{itemize}
\item \textit{Step II-d.} Update the finding rates $q^{t-1,\tau},p^{t-1,\tau}$
and share $\phi^{t-1,\tau}$ that are consistent with the vacancy
policy $v^{t}$ and the distribution of firms $h^{t-1,\tau}$. This
is done by file \texttt{FindingRates.m}.
\begin{itemize}
\item First get the total units of search efficiency in the labor market
and total vacancies to construct:
\begin{eqnarray*}
\texttt{u}^{t-1,\tau} & = & \bar{\texttt{n}}-\texttt{m}\int\int ndH^{t-1,\tau}\left(z,n\right)\\
\texttt{s}^{t-1,\tau} & = & \texttt{u}^{t-1,\tau}+\xi\texttt{u}^{t-1,\tau}\\
\phi^{t-1,\tau} & = & \frac{\texttt{u}^{t-1,\tau}}{\texttt{s}^{t-1,\tau}}
\end{eqnarray*}
\item Then, get total vacancies and use the matching function:
\begin{eqnarray*}
\texttt{v}^{t-1,\tau} & = & \texttt{m}\int\int v^{t}\left(z,n\right)+v_{e}^{t-1,\tau}\left(z,n\right)dH^{t-1,\tau}\left(z,n\right)\\
\theta^{t-1,\tau} & = & \frac{\texttt{v}^{t-1,\tau}}{\texttt{s}^{t-1,\tau}}\\
q^{t-1,\tau} & = & A\left(\theta^{t-1,\tau}\right)^{\beta-1}\\
p^{t-1,\tau} & = & A\left(\theta^{t-1,\tau}\right)^{\beta}
\end{eqnarray*}
\end{itemize}
\item Iterate over the four sub-steps \textit{Step II-a} - \textit{Step
II-d} until convergence and assign the updated aggregate states $q^{t},\ p^{t},\ \phi^{t},\ h^{t},\ H_{n}^{t}$
and $H_{v}^{t}$.
\end{itemize}
\medskip{}

We subsequently iterate on \textbf{Step I - Step II} until both the
surplus function and the aggregate states have converged.
%\end{document}
